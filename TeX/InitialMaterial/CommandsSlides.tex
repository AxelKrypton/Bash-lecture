\newcommand<>{\tc}[2]{\textcolor#3{#1}{#2}}
\newcommand{\tikzmark}[1]{\tikz[overlay,remember picture, baseline=-0.5ex] \node at (0,0) (#1) {};}
\newcommand{\URLsymbol}[2][white]{%
    \begin{tikzpicture}[every path/.style={line width=3, rounded corners, #2}]
        \pgfmathsetmacro{\longSide}{0.9}
        \pgfmathsetmacro{\shortSide}{0.3}
        \draw[rotate=45, xshift=0.5*\longSide cm]       (0,0) rectangle (\longSide, \shortSide);
        \draw[rotate=45, halo=#1, #2!50]                   (0,0) rectangle (\longSide, \shortSide);
        \draw[rotate=45, halo=#1, xshift=0.5*\longSide cm] (0, 0.5*\shortSide) -- (0,0) -- (\longSide, 0) -- (\longSide, 0.5*\shortSide);
    \end{tikzpicture}
}
\NewDocumentCommand{\URL}{ O{black} m m O{BGLIGHT} }%
{%
    \raisebox{-0.4ex}{\resizebox{!}{2ex}{\URLsymbol[#4]{#1}}}{\href{#2}{\textcolor{#1}{#3}}}
}

\NewDocumentCommand{\addSection}{ m O{} m m }%
{%
    \setbeamertemplate{section page}[Iceland][#2]{#3}{#4}
    \section{#1}
}

\makeatletter
\newcommand*\keystroke[1]
{%
    \begin{tikzpicture}[baseline=($(key.base)!0.8!(key.south)$), very thin]%
        \pgfmathsetlengthmacro{\textHeight}{0.9*\f@size}
        \pgfmathsetlengthmacro{\textHeightPlus}{1.25*\textHeight}
        \pgfmathsetlengthmacro{\roundedSmall}{0.2}
        \pgfmathsetlengthmacro{\roundedLarge}{0.4}
        \pgfmathsetlengthmacro{\dl}{0.5\pgflinewidth}
        \node[font=\sffamily, inner xsep=2pt, inner ysep=1pt] (text) {\scalebox{1.2}[0.75]{\textsmaller[2]{#1\strut}}};
        \node[rounded corners=\roundedLarge, minimum size=\textHeight, anchor=north west] (key) at (text.north west){};
        \node[rounded corners=\roundedSmall, minimum size=\textHeightPlus] (frame) at ($(key.center)!0.05!(key.south)$){};
        \path coordinate (keyNW) at ($(key.north west)+(\roundedLarge,-\dl)$)
              coordinate (keyNE) at ($(key.north east)-(\roundedLarge,+\dl)$)
              coordinate (keyEN) at ($(key.north east)-(-\dl,\roundedLarge)$)
              coordinate (keyES) at ($(key.south east)+(+\dl,\roundedLarge)$)
              coordinate (keySE) at ($(key.south east)-(\roundedLarge,+\dl)$)
              coordinate (keySW) at ($(key.south west)+(\roundedLarge,-\dl)$)
              coordinate (keyWS) at ($(key.south west)+(+\dl,\roundedLarge)$)
              coordinate (keyWN) at ($(key.north west)-(-\dl,\roundedLarge)$)
              coordinate (frameNW) at ($(frame.north west)+(\roundedSmall,-\dl)$)
              coordinate (frameNE) at ($(frame.north east)-(\roundedSmall,+\dl)$)
              coordinate (frameEN) at ($(frame.north east)-(+\dl,\roundedSmall)$)
              coordinate (frameES) at ($(frame.south east)+(-\dl,\roundedSmall)$)
              coordinate (frameSE) at ($(frame.south east)-(\roundedSmall,-\dl)$)
              coordinate (frameSW) at ($(frame.south west)+(\roundedSmall,+\dl)$)
              coordinate (frameWS) at ($(frame.south west)+(+\dl,\roundedSmall)$)
              coordinate (frameWN) at ($(frame.north west)-(-\dl,\roundedSmall)$);
        \foreach \n in {NW,NE,EN,ES,SE,SW,WS,WN}{
            \draw[line cap=round, ultra thin] (key\n) -- (frame\n);
        }
        \begin{scope}[on background layer]
            \node[draw, rounded corners=\roundedSmall, minimum size=\textHeightPlus, fill=fg] at ($(key.center)!0.05!(key.south)$){};
            \node[draw, rounded corners=\roundedLarge, lower right=gray!20, lower left=gray!50, upper right=gray!50, upper left=gray!80, minimum size=\textHeight, anchor=north west] at (text.north west){};
            \fill [gray!70!bg] (keyNW) -- (keyNE) -- (frameNE) -- (frameNW) -- cycle;
            \fill [gray!50!bg] (keyWS) -- (keyWN) -- (frameWN) -- (frameWS) -- cycle;
            \fill [gray!30!bg] (keyES) -- (keyEN) -- (frameEN) -- (frameES) -- cycle;
            \fill [gray!10!bg]  (keySW) -- (keySE) -- (frameSE) -- (frameSW) -- cycle;
        \end{scope}
  \end{tikzpicture}%
}
\makeatother

\newcommand{\Remark}[2][1mm]
{%
    \hspace{#1}{\tiny\{~#2~\}}%
}

\newcommand{\FrameRemark}[2][1-]%
{%
    \begin{tikzpicture}[remember picture, overlay]
        \node[font=\tiny, anchor=south, visible on=<#1>] at (current page.south) {#2};
    \end{tikzpicture}
}

\newcommand{\MakeEnumerateBox}[1]%
{%
    \hbox{%
      \usebeamerfont*{item projected}%
      \usebeamercolor[bg]{item projected}%
      \vrule width2.25ex height1.85ex depth.4ex%
      \hskip-2.25ex%
      \hbox to2.25ex{%
        \hfil%
        \color{fg}#1%
        \hfil}%
    }%
}

%Tikz group of commands to make a pyramid layer
\newcommand{\pyramidLayer}[6][]{%
    \pgfmathsetmacro{\Thickness}{#3}
    \pgfmathsetmacro{\Long}{#4}
    \pgfmathsetmacro{\Short}{#4-0.5*#3}
    \path coordinate (middleUpperRight) at ($(#2)+(0,\Thickness)+(30:0.5*\Short)$)
          coordinate (middleLowerRight) at ($(#2)+(30:0.5*\Long)$)
          coordinate (upperRight) at ($(#2)+(0,\Thickness)+(30:\Short)$)
          coordinate (lowerRight) at ($(#2)+(30:\Long)$);
    \draw[fill=#5!50] ($(#2)+(0,\Thickness)+(30:\Short)$)  -- ++(210:\Short) -- ++(270:\Thickness) -- ++(30:\Long) -- cycle;
    \draw[fill=#5!20] ($(#2)+(0,\Thickness)+(150:\Short)$)  -- ++(330:\Short) -- ++(270:\Thickness) -- ++(150:\Long) -- cycle;
    \draw[fill=#5]    ($(#2)+(0,\Thickness)$) -- ++(30:\Short) -- ++(150:\Short) -- ++(210:\Short) -- cycle;
    \node at ($(middleUpperRight)!0.5!(middleLowerRight)$) {\rotatebox{30}{#6}};
    \node[text=#5, anchor=west] at ($(upperRight)!0.5!(lowerRight)+(0.5,0)$) {#1};
}

\renewcommand{\checkmark}[1][PS]{\tc{#1}{\ding{51}}}
\newcommand{\crossmark}[1][PT]{\tc{#1}{\ding{55}}}

\def\quizOverlay{1}
\newcounter{QuizNumber}
\resetcounteronoverlays{QuizNumber}% To avoid the counter being increased by overlays
\newenvironment{quiz}[2][1]{%
    \edef\tmpNumber{\numexpr #1 +1\relax}%
    \gdef\quizOverlay{\the\tmpNumber}%
    \stepcounter{QuizNumber}%
    \medskip%
    % mbox to avoid newline after hbox of enumerate
    \mbox{\MakeEnumerateBox{\theQuizNumber}}\enspace #2%
    \begin{itemize}%
}{%
    \end{itemize}%
    \bigskip%
}

\newcommand{\wrongChoice}[1]%
{%
    \item[\alt<\quizOverlay->{\crossmark}{$\bullet$}] #1
}

\newcommand{\correctChoice}[1]%
{%
    \item[\alt<\quizOverlay->{\checkmark}{$\bullet$}] #1
}
