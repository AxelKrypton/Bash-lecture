\newcommand{\tikzmark}[1]{\tikz[overlay,remember picture, baseline=-0.ex] \node at (0,0) (#1) {};}

\newcommand{\URLsymbol}[2][white]{%
    \begin{tikzpicture}[every path/.style={line width=3, rounded corners, #2}]
        \pgfmathsetmacro{\longSide}{0.9}
        \pgfmathsetmacro{\shortSide}{0.3}
        \draw[rotate=45, xshift=0.5*\longSide cm]       (0,0) rectangle (\longSide, \shortSide);
        \draw[rotate=45, halo=#1, #2!50]                   (0,0) rectangle (\longSide, \shortSide);
        \draw[rotate=45, halo=#1, xshift=0.5*\longSide cm] (0, 0.5*\shortSide) -- (0,0) -- (\longSide, 0) -- (\longSide, 0.5*\shortSide);
    \end{tikzpicture}
}
\NewDocumentCommand{\URL}{ O{black} m m O{white} }%
{%
    \raisebox{-0.4ex}{\resizebox{!}{2ex}{\URLsymbol[#4]{#1}}}{\href{#2}{\textcolor{#1}{#3}}}
}

\newcommand\fancybox[3]{%
\tcbset{
    mybox/.style={
        enhanced,
        boxsep=0mm,
        opacityfill=0,
        overlay={
            \coordinate (X) at ([xshift=-1mm, yshift=-1.5mm]frame.north west);
            \node[align=right, text=#1, text width=4cm, anchor=north east] at (X) {#2};
            \draw[line width=0.8mm, color=#1] (frame.north west) -- (frame.south west);
            }
        }
    }
\begin{tcolorbox}[mybox]
    #3
\end{tcolorbox}
}

% Now that we have the exercise style, create counter and exercise delimiters
% NOTE: It is not possible to put a listing in a custom environment with other latex code
\newcounter{Exercise}
\makeatletter
\newcommand{\BeginExercise}[1][]
{%
    \refstepcounter{Exercise}
    \noindent\tikzmark{boxStart}\ifstrempty{#1}{}{%
        \textbf{#1}\\[0.3em]
    }
}

\newcommand{\EndExercise}[1][Navy]
{%
    \tikzmark{boxEnd}
    \begin{tikzpicture}[remember picture, overlay]
        %Adjust start-end coordinates
        \path coordinate (lineStart) at ($(boxStart)+(-1em,1.5*\f@size pt)$)
              coordinate (lineEnd)   at ($(lineStart |- boxEnd)+(0,-0.8*\f@size pt)$);
        \node[align=right, font=\large, text=#1, text width=3cm, base left = 1em of boxStart] {Exercise~\theExercise};
        \draw[line width=0.8mm, color=#1] (lineStart) -- (lineEnd);
    \end{tikzpicture}
    \par\vspace{8mm}
}
\makeatother
