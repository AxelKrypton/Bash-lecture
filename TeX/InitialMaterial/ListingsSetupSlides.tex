%Colors for listings
\colorlet{background-color}{gray!20}
\colorlet{basic-color}{black}
\colorlet{keywords-color}{Goldenrod}
\colorlet{comment-color}{red!95!black}
\colorlet{strings-color}{ForestGreen}
\colorlet{builtins-color}{MediumBlue!90!black}
\colorlet{functions-color}{NavyBlue}
\colorlet{variables-color}{DarkOrange}
\colorlet{environment-color}{Gray}
\colorlet{external-color}{SteelBlue}


%listings set
\lstdefinestyle{MyBash}{
backgroundcolor=\color{background-color}, % choose the background color; you must add \usepackage{color} or \usepackage{xcolor}
breakatwhitespace=false,            % sets if automatic breaks should only happen at whitespace
breaklines=true,                    % sets automatic line breaking
captionpos=b,                       % sets the caption-position to bottom
deletekeywords={...},               % if you want to delete keywords from the given language
escapeinside={@|}{|@},              % if you want to add LaTeX within your code
extendedchars=true,                 % lets you use non-ASCII characters; for 8-bits encodings only,
                                    % does not work with UTF-8
frame=single,                       % adds a frame around the code
framerule=0pt,                      % Width of the frame rule
framesep=3pt,                       % separation around text
linewidth=\textwidth,               % defines the base line width for listings
xleftmargin=6mm,                    % Margin left
xrightmargin=6mm,                   % Margin right
numbers=left,                       % where to put the line-numbers; possible values are (none, left, right)
numbersep=8pt,                      % how far the line-numbers are from the code
numberstyle=\tiny\color{black},     % the style that is used for the line-numbers
rulecolor=\color{black},            % if not set, the frame-color may be changed on line-breaks within not-black text
                                    % (e.g. comments (green here))
showspaces=false,                   % show spaces everywhere adding particular underscores; it overrides 'showstringspaces'
showstringspaces=false,             % underline spaces within strings only
showtabs=false,                     % show tabs within strings adding particular underscores
stepnumber=1,                       % the step between two line-numbers. If it's 1, each line will be numbered
tabsize=2,                          % sets default tabsize to 2 spaces
title=\lstname,                     % show the filename of files included with \lstinputlisting; also try caption instead of title
%
%Base style for this presentation 
keepspaces=true,                    % keeps spaces in text, useful for keeping indentation of code
                                    % (possibly needs columns=flexible)
language=bash,
basicstyle=\ttfamily\scriptsize\color{basic-color},
keywordstyle=\color{keywords-color},
stringstyle=\color{strings-color},
commentstyle=\color{comment-color},
morestring=[b][\color{strings-color}]{"},
morestring=[d][\color{strings-color}]{'},
moredelim=[is][\color{basic-color}]{|+}{+|}, % I will use this for terminal output
literate={`}{\textasciigrave}1, % https://tex.stackexchange.com/a/466224/128737
literate={~}{{\textasciitilde}}1,
% literate=% literate={<replace>}{<replacement text>}{<width>}
%   {\#define}{{{\color{CarnationPink}\#define}}}{6}
%   {\#include}{{{\color{CarnationPink}\#include}}}{7},
alsoletter=0123456789![]/\{\}.:+, % This to mark the symbols in keyword/emph[5] to be highlighted (otherkeywords does not work i.e. it highlights also in comments!) -> manual at page 45
morekeywords={if, then, else, elif, fi, case, esac, for, select, while, until, do, done, in, function, time, [[, ]], \{, \}, !, coproc}, %https://askubuntu.com/a/513712
emph=[1]{CreateListOfFiles, LevelOne, LevelTwo, LevelThree, Test, SecondsToTimeStringWithDays, ExampleFunction,
         ExampleFunction_implementation, ExtractColumnFromFile, CalculateSizeOfFiles, ReportOnLargestDirectories,
         CountTill5From, FifthElementOf, CreateAuxiliaryFiles, CleanAuxiliaryFiles},
emphstyle=[1]{\color{functions-color}}, %Functions
emph=[2]{variableName, invisibleVariable, prefix, day, today, song, aVar, bVar, langRegex, deadline, now,
         index, file, fgbg, color, reference, files, array, message, entry, index, dict, key, flag, line, inputTime,
         days, hours, minutes, seconds, globalVar, pid, extglobSet},
emphstyle=[2]{\color{variables-color}}, %Variables
emph=[4]{PATH, SHELL, IFS, BASH_ALIASES, BASH_REMATCH, PS3, REPLY, HOME, LANGUAGE, EDITOR, PIPESTATUS, PWD, FUNCNEST,
         DIRSTACK, PWD, OLDPWD, SHELLOPTS, BASHOPTS},
emphstyle=[4]{\color{environment-color}}, %Environment variables
emph=[5]{alias, bg, bind, break, builtin, cd, command, compgen, complete, continue, declare, dirs, disown, echo, enable, eval,
         exec, exit, export, false, fc, fg, getopts, hash, help, history, jobs, kill, let, local, logout, popd, printf, pushd, pwd,
         read, readonly, return, set, shift, shopt, source, suspend, test, times, trap, true, type, typeset, ulimit, umask,
         % case, if, until, while  % <--- these built-in are keywords and I leave them highlighted as such
         unalias, unset, wait, :, ., [, ]},
emphstyle=[5]{\color{builtins-color}}, %Shell built-in
emph=[6]{man, apropos, ls, rm, g++, chmod, cp, awk, sed, cut, perl, args, date, grep, sleep, tput, seq, cat, wc, sort, tail,
         head, sdiff, tar, mktemp, mkdir, ps, emacs, systemd, timeout, parallel, xargs, gnuplot, pdflatex, vi, ping, bash,
         egrep, shuf, stat, find, fgrep, bc, tr, paste},
emphstyle=[6]{\color{external-color}}, %(External) commands
emph=[7]{},
emphstyle=[7]{\color{variables-color}}, %Class for local variables (usually with bad names)
emph=[8]{},
emphstyle=[8]{\color{builtins-color}}, %Class for local commands (usually with bad names)
%
%Additional customizations
belowskip=-7mm,
aboveskip=3pt,
autogobble=true, % lstautogobble needed!
}

\lstnewenvironment{Bash}[1][] %I will rarely use this because putting a $ in it as prompt breaks down TeXclipse highlight syntax!
    {\lstset{style=MyBash, #1}}
    {}

\def\bash{\lstinline[style=MyBash, basicstyle=\ttfamily\color{black}]}

%This additional style is to just print odd numbers (NOTE: style keyword can be repeated and it is cumulative!)
\lstdefinestyle{oddnumbers}{
    stepnumber=2,
    firstnumber=0,
    numberstyle={\tiny\color{black}\ifodd\value{lstnumber}\relax\else\refstepcounter{lstnumber}\fi\tiny\color{black}\ifodd\value{lstnumber}\relax\else\refstepcounter{lstnumber}\fi}
}

%This additional style is to just print odd numbers (NOTE: style keyword can be repeated and it is cumulative!)
\lstdefinestyle{smaller}{
    basicstyle={\linespread{1.2}\ttfamily\ssmall\color{basic-color}}
}
