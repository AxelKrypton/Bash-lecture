\usepackage{listings}
\def\transpPerc{100}
%listings set
\lstdefinestyle{MyBash}{
% backgroundcolor=\color{white},    % choose the background color; you must add \usepackage{color} or \usepackage{xcolor}
breakatwhitespace=false,            % sets if automatic breaks should only happen at whitespace
breaklines=true,                    % sets automatic line breaking
captionpos=b,                       % sets the caption-position to bottom
deletekeywords={...},               % if you want to delete keywords from the given language
escapeinside={@|}{|@},                % if you want to add LaTeX within your code
extendedchars=true,                 % lets you use non-ASCII characters; for 8-bits encodings only,
                                    % does not work with UTF-8
frame=none  ,                       % adds a frame around the code
numbers=none,                       % where to put the line-numbers; possible values are (none, left, right)
numbersep=5pt,                      % how far the line-numbers are from the code
numberstyle=\tiny\color{black},     % the style that is used for the line-numbers
rulecolor=\color{black},            % if not set, the frame-color may be changed on line-breaks within not-black text
                                    % (e.g. comments (green here))
showspaces=false,                   % show spaces everywhere adding particular underscores; it overrides 'showstringspaces'
showstringspaces=false,             % underline spaces within strings only
showtabs=false,                     % show tabs within strings adding particular underscores
stepnumber=2,                       % the step between two line-numbers. If it's 1, each line will be numbered
stringstyle=\color{OliveGreen},     % string literal style
tabsize=2,                          % sets default tabsize to 2 spaces
title=\lstname,                     % show the filename of files included with \lstinputlisting; also try caption instead of title
%
%Base style for this presentation 
keepspaces=true,                    % keeps spaces in text, useful for keeping indentation of code
                                    % (possibly needs columns=flexible)
keywordstyle=\color{Cyan},          % keyword style
language=C++,
basicstyle=\ttfamily\scriptsize\color{black},
keywordstyle=\color{OliveGreen},
stringstyle=\color{Magenta},
commentstyle=\color{red},
moredelim=[is][\color{ForestGreen}]{|+}{+|},
literate=% literate={<replace>}{<replacement text>}{<width>}
  {\#define}{{{\color{CarnationPink}\#define}}}{6}
  {\#include}{{{\color{CarnationPink}\#include}}}{7},
morekeywords={},
emph=[1]{},
emphstyle=[1]{\color{NavyBlue}}, %Functions
emph=[2]{},
emphstyle=[2]{\color{Orange}}, %Variables
emph=[3]{if, else, elif, fi, while, for, case, do, esac, done},
emphstyle=[3]{\color{violet}}, %Loops, if, etc.
emph=[4]{return, exit},
emphstyle=[4]{\color{ProcessBlue}}, %Logical keywords
emph=[6]{PATH, SHELL},
emphstyle=[6]{\color{Gray}}, %Environment variables
}

\lstnewenvironment{Bash}[1][]
    {\lstset{style=MyBash, belowskip=-7mm, aboveskip=0pt,#1}}
    {}

\def\bash{\lstinline[style=MyBash, basicstyle=\ttfamily\color{black}]}


\makeatletter
\newenvironment{CenteredBox}{% 
\begin{Sbox}}{% Save the content in a box
\end{Sbox}\centerline{\parbox{\wd\@Sbox}{\TheSbox}}}% And output it centered
\makeatother

