%-------------------------------%
%  Author: Alessandro Sciarra   %
%    Date: 16 Sep 2019          %
%-------------------------------%

\begin{exercise}[Inspirational]{Understanding the power of parameter expansion}
    Go through the following list of tasks and explore the parameter expansion syntax.
    Feel free to simply play in the terminal or write one or more scripts.
    \begin{enumerate}
        \item The \bash|EDITOR| environment variable is used to specify the user preferred text editor.
              Suppose to have a variable \bash|filename| which store the name of a file.
              Then the command
              \begin{lstlisting}[style=MyBash]
                  ${EDITOR} "${filename}"
              \end{lstlisting}
              would open the file to edit.
              How would you make this command safer, considering the possibility that either of the variable (or both) is unset or null?
              Use a default value for \bash|EDITOR| and abort if \bash|filename| is unset.
        \item The environment \bash|PATH| variable contains the locations where the OS looks for commands as colon-separated list.
              Print the highest- and lowest-priority locations.
              How would you, instead, exclude the highest- and the lowest-priority locations?
        \item It is common to have to deal with strings that contain some information separated by a delimiter.
              Consider for example the string \bash|"b5.6789_s9876_thermalizeFromHot"| which identify a LQCD simulation.
              The characters \texttt{'b'} and \texttt{'s'} refer to a beta and a seed value.
              Such prefixes are alphabetic strings, whose length might vary.
              Assuming the beta and the seed values format are fixed, how would you extract
              \begin{itemize}
                  \item the beta value?
                  \item the seed value?
                  \item the postfix after the last \texttt{'\_'}?
                  \item the beta and the seed prefixes?
              \end{itemize}
              Test your code on different strings like \bash|"beta6.0000_seed1111_continueWithNewChain"| or \bash|"beta6.1234_s1234_thermalizeFromConf"|.
        \item The \bash|printf| builtin has the following interesting feature.
              \begin{quote}
                \emph{%
                    The format is reused as necessary to consume all  of  the  arguments.
                    If the format requires more arguments than are supplied, the extra format specifications behave as if a zero value or null string, as appropriate, had  been supplied.
                }
              \end{quote}
              It is then easily possible to concatenate strings using a delimiter.
              \begin{lstlisting}[style=MyBash]
                  printf '%s_' First Second Third
              \end{lstlisting}
              Use this command to assign the resulting string to a variable, getting rid of the trailing underscore.
              Run \bash|help printf| to get inspired.
    \end{enumerate}
\end{exercise}