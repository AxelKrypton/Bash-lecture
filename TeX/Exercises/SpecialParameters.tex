%-------------------------------%
%  Author: Alessandro Sciarra   %
%    Date: 3 Jul 2019           %
%-------------------------------%

\begin{exercise}[Instructive]{The special parameters \bash{*} and \bash{\@} and their quoted versions}
    Consider the following script.
    \begin{lstlisting}[style=MyBash, numbers=left]
        #!/bin/bash
        printf '\nScript run with %d argument(s)\n' "$#"
        IFS=":${IFS}"
        printf 'Using "$@":'
        printf ' <%s>' "$@"   # or "$*" or $@ or $*
        printf '\n\n'
    \end{lstlisting}
    Use the manual or the web to understand how the command \bash|printf| works and to understand then the given script.
    Make the above script executable and complete it adding lines 4-6 for \bash{"$*"}, for \bash{$@} and for \bash{$*}.
    Create a new temporary folder, move into it and \bash|touch Day_{1..3}.dat| (what happens?).
    Run your script with the following arguments:
    \begin{lstlisting}[style=MyBash]
        '*.dat' $(whoami) "Hello World"
    \end{lstlisting}
    Have you understood the difference between \bash{"$@"}, \bash{"$*"}, \bash{$@} and \bash{$*}?
    Are there differences between the unquoted \bash{$@} and \bash{$*}?
\end{exercise}