%-------------------------------%
%  Author: Alessandro Sciarra   %
%    Date: 3 Jul 2019           %
%-------------------------------%

\begin{exercise}[Instructive]{Handling command line options}
    Write a Bash script which checks if the following requirements on its command line options are all fulfilled.
    \begin{itemize}
        \item The first argument is  either \bash{-n} (numbers mode) or {-f} (file mode).
        \item If the first argument is \bash{-n} the script must receive either \texttt{2} or \texttt{8} as second argument and a string among \bash{I}, \bash{II}, \bash{III} as third argument.
        \item If the first argument is \bash{-f} the script must receive a three-lowercase-characters string as second argument.
    \end{itemize}
    If the number mode is selected, the script must print all numbers in base $b$ (either 2 or 8, depending on the value of the second option) with one (\bash{I}), two (\bash{II}) or three (\bash{III}) digits.
    In file mode, the script should print all the files in the directory where it is run matching the extension specified as second option (e.g. all files ending with \bash{.txt} if the second option is \bash{txt}.)
    This exercise should make you miss better features you will learn about tomorrow\ldots
\end{exercise}