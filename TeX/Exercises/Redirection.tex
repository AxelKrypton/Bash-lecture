%-------------------------------%
%  Author: Alessandro Sciarra   %
%    Date: 15 Oct 2020          %
%-------------------------------%

\begin{exercise}[Instructive]{Discovering the redirection mechanism}
    From now on, you will bring your new bash scripts to a new level, e.g. always redirecting error messages to standard error.
    
    Suppose to have to produce at some point in a larger script an input file for some application.
    This might require e.g. some logic to add different lines to a file.
    A minimal example might be the following, although you have to imagine to have tens of printing statements and more complex structures.
    \begin{lstlisting}[style=myBash]
        echo "THERMALIZE=1"
        if (($1 % 4 == 0)); then echo "USE_HMC"; else echo "USE_RHMC"; fi
        echo "DO_BACKUP=0"
    \end{lstlisting}
    
    Think of how redirecting the output of the block of code above to an external file
    \begin{itemize}[nosep]
        \item using a compound command;
        \item using the \bash|exec| builtin;
        \item using a here document (maybe not ideal).
    \end{itemize}
\end{exercise}