%-------------------------------%
%  Author: Alessandro Sciarra   %
%    Date: 21 Oct 2020          %
%-------------------------------%

\begin{exercise}[Inspirational]{Plotting data in LaTeX from a Bash script}
    As final exercise of this course we would like to write a last script, which might let you discover some interesting feature for your work as physicist.
    We assume you have a \LaTeX\ distribution installed on the machine where you'll work.
    If this is not the case, just add the following \bash|pdflatex| function to your script to mimic it:
    \begin{lstlisting}[style=myBash]
        function @|\textcolor{functions-color}{\texttt{pdflatex}}|@()
        {
            echo "$@"; sleep 5
            touch 'Plot.'{log,aux,pdf}
        }
    \end{lstlisting}
    You will loose the graphical aspect of this exercise, but you will still be able to work on it.

    As you might know, \LaTeX\ has a wonderful package, \texttt{PGF}, which is provided with a math engine and on which lots of packages are based (e.g. \texttt{TikZ}, \texttt{forest}, \texttt{genealogytree}).
    Here we will use \texttt{pgfplots}, which allows to make nice-looking plots of e.g. data.
    Our goal is to create a Bash script which plots the data in a given file, using one column as $x$-values and another one for the $y$-values.
    The minimal user requirements are the following, but with a bit of creativity you can implement much more.
    \begin{enumerate}
        \item The input data filename must be given on the command line.
        \item It should be possible to specify the $x$ and $y$ columns to be plotted, using some default values.
        \item The script should create the plot as pdf file and give informative output to the user.
        \item An external \texttt{.tex} source file should be used, which could also be compiled by hand if needed.
    \end{enumerate}

    The flow should be clear.
    The command line options are parsed, the \texttt{.tex} file is used according to the user directives and it is finally compiled.
    \textbf{The first column in the data file is the number 0, the second is the number 1, etc.}

    The way to hand over information from Bash to \LaTeX\ world is not part of the exercise and the short story\footnote{%
        \textbf{For the curious reader:}
        In the \LaTeX\ code, the macro can be defined only \texttt{\textbackslash ifundefined} to provide a default value to compile the \texttt{.tex} file by hand.
        Invoking the \bash|pdflatex| compiler as above, it is possible to define some macros in advance which are then handed-over to the \LaTeX\ source code from outside.
        The \texttt{batchmode} of the \bash|pdflatex| compiler will suppress all but a handful of declarative lines, without stopping, even in the case of a syntax error.
    } is that you should use this command
    \begin{lstlisting}[style=myBash]
        pdflatex -interaction=batchmode \
        "\def\filename{...} \def\xColumn{...} \def\yColumn{...} \documentclass{standalone}
\usepackage[usenames, svgnames]{xcolor}
\usepackage{pgfplots}
\pgfplotsset{compat=1.14}

\ifdefined\filename%
\else%
    \def\filename{data.dat}% Name of the file
\fi%

\ifdefined\xColumn%
\else%
    \def\xColumn{0}% column in file with x-values
\fi%

\ifdefined\yColumn%
\else%
    \def\yColumn{1}% column in file with y-values
\fi%

\ifdefined\xLabel%
\else%
    \def\xLabel{$x$}% label for x-axis
\fi%

\ifdefined\yLabel%
\else%
    \def\yLabel{$y$}% label for y-axis
\fi%

\begin{document}
\begin{tikzpicture}
    \begin{axis}[xlabel=\xLabel, ylabel=\yLabel]
        \addplot table [x index=\xColumn, y index=\yColumn] {\filename};
    \end{axis}
\end{tikzpicture}
\end{document}
"
    \end{lstlisting}
    in your Bash script to compile the \texttt{.tex} file (the \texttt{...} should be the properly filled by your Bash script).
    \URL[Blue]{https://github.com/AxelKrypton/Bash\_lecture\_2020/blob/master/Bash/Exercise\_Sheet\_5/Plot.tex}{Here} you find the \LaTeX\ source code to be used (you can have a look to it, although not necessary).

    Now that you know how to hand over information to the \LaTeX\ compiler, you can focus on your Bash script.
    In particular, be sure to have examined and dealt with the following aspects.
    \begin{itemize}
        \item As you know, \LaTeX\ produces a bunch of additional files and it would be nice to remove them to leave the present folder in a tidy state.
        \item The clean-up should be always done, unless you have some reason not to do it.
              For example, how does you script behave when pressing \texttt{CTRL-C} during execution?
              It might be interesting to use the \texttt{-halt-on-error} of {pdflatex} to make it better handle an interruption.
        \item Which strategy did you use to do error handling? What happens when the \LaTeX\ compilation fails? Trigger a failure by hand to test your code.
    \end{itemize}
    You can produce an input data file using \bash|awk| via
    \begin{lstlisting}[style=myBash]
        awk 'BEGIN{for(i=0; i<10; i++){print i"\t "rand()"\t"2*i"\t"10*rand()}}'
    \end{lstlisting}
    and try to plot e.g. the fourth column as function of the third one.
    \vspace{1cm}
\end{exercise}