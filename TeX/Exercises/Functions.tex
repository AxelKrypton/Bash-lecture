%-------------------------------%
%  Author: Alessandro Sciarra   %
%    Date: 17 Sep 2019          %
%-------------------------------%

\begin{exercise}[Instructive]{The world of functions}
    In this exercise we will write few functions to practice and understand how they can be useful.
    \begin{enumerate}
        \item In the exercise session from yesterday, you learnt how to provide your script with command line options and, then, how to parse them.
              Write a function that takes over this responsibility.
              How do you invoke this function?
              Take the opportunity to structure further more your code, extracting more functions to avoid code duplication.
        \item Write a function which checks if an element is in a given array.
              How is input passed to this function?
              How would you signal to the caller the presence/absence of the element?
              Your function must work in general.
              Test it with
              \begin{lstlisting}[style=MyBash]
                  array=("first element"  $'to be\nfound'  "something")
                  string=$'to be\nfound'
              \end{lstlisting}
              or with \bash|string="notExisting"|.
        \item Write a function that, given two integers, calculates their \textbf{greatest common divisors} using the \URL[Blue]{https://en.wikipedia.org/wiki/Greatest\_common\_divisor\#Euclid's\_algorithm}{Euclid's algorithm}.
              This is a great opportunity to write a recursive function.
              Write another function which calculates the \textbf{least common multiple} of two integers, using the relation
              \[
                \mbox{lcm}(a,b)\cdot\mbox{gcd}(a,b)=a\cdot b \quad.
              \]
    \end{enumerate}
\end{exercise}


