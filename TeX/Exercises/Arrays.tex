%-------------------------------%
%  Author: Alessandro Sciarra   %
%    Date: 25 Jul 2019          %
%-------------------------------%

\begin{exercise}[Instructive]{The power of arrays}
    Arrays play a very important role in Bash scripting and they are too often forgotten.
    To warm up and also explore the array notation think about
    \begin{itemize}
        \item how to find the longest entry of an array?
        \item how to find the maximum of an array with \emph{numeric} entries?
        \item how to find only common entries in two arrays?
        \item how to sort an array?
        \item how to check if an array is sparse?
    \end{itemize}
    Once you feel comfortable with the array syntax, tackle the following problems.
    \begin{enumerate}[after=\vspace{-\baselineskip}]
        \item Write a Bash script to make a report of the files in the present folder, counting them by extension.
              In order to test your script, create a test folder where you can create some files via
              \begin{lstlisting}[style=MyBash, emph={[2]ext, index}, alsoletter={0123456789![]/\{\}.:}]
                  for ext in jpg png eps pdf txt odt tex; do
                      for ((index=1; index<=$(shuf -i 3-9 -n 1); index++)); do
                          touch file_${index}.${ext}
                      done
                  done; unset -v 'ext' 'index'
              \end{lstlisting}
        \item Write a script that given integers on the command line parameters, calculates their \textbf{greatest common divisors} and their \textbf{least common multiple}, using the prime factorisations.
              Take a look to the \bash|factor| command.
    \end{enumerate}
    \bigskip
\end{exercise}
