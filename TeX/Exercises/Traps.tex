%-------------------------------%
%  Author: Alessandro Sciarra   %
%    Date: 17 Sep 2019          %
%-------------------------------%

\begin{exercise}[Inspirational]{Using traps to clean up}
    The main purpose of this exercise is to understand how a trap works and how it can be useful.
    We will consider the most common case, cleaning up, but signal handling can be used for much more than that.
    
    Suppose you have an analysis tool that takes a file with only one column in input and produces a file in output.
    Suppose as well that your tool produces temporary files which are auxiliary for the analysis.
    We will mimic the analysis tool via \bash|sort -g| (redirecting the output) and the temporary files will be simply touched.
    Write a script to complete the following operations.
    \begin{enumerate}
        \item \bash|cut| the data file into several files, each having one column of the original file.
        \item For each single-column file
              \begin{itemize}
                  \item process it with the analysis tool (which will produce an output file);
                  \item create the auxiliary files, e.g.\ with \bash|touch| \texttt{column\_\$\{index\}\_\{0..9\}.aux} (or similar);
                  \item sleep for one second;
                  \item remove the auxiliary files.
              \end{itemize}
    \end{enumerate}
    You can produce a fake data file via \,\bash{seq 0 .001 1 | shuf | columns -c 11 -W 100}\, which has 11 columns, each with 91 lines.
    \begin{enumerate}
        \setcounter{enumi}{2}
        \item Run the script (ideally in a new folder) and abort it with CTRL-C after few seconds.
              Your folder should have been polluted by a bunch of files.
        \item Add a \bash|trap| to your script to clean up on exit.
    \end{enumerate}
\end{exercise}