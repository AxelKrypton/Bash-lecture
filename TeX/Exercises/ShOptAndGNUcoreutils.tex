%-------------------------------%
%  Author: Alessandro Sciarra   %
%    Date: 25 Sep 2020          %
%-------------------------------%

\begin{exercise}[Inspirational]{Shell options and GNU coreutils: Inquire about files}
    Write a bash script that
    \begin{enumerate}
        \item takes one or more file extensions on the command line;
        \item finds all the files from the invoking position with such an extension (also in sub-folders);
        \item prints a small report about each extension, specifying
              \begin{itemize}
                  \item the number of existing files;
                  \item the size of the largest file;
                  \item The total number of lines of all files.
              \end{itemize}
    \end{enumerate}

    You learnt today about \emph{shell options} and \emph{GNU core utilities}.
    Use this exercise to do some practice with them.
    In particular, write this script enabling \bash|nounset| and \bash|errexit| options via
    \begin{lstlisting}[style=myBash]
        set -o nounset -o errexit   # equivalent to 'set -u -e' or 'set -ue'
    \end{lstlisting}
    at the beginning of your script after the shebang.

    \textbf{Attention:} You might be surprised from some strange shell behaviour and this might be connected to the enabled shell options.
    Feel free to ask if something strange occurs.
    We will discuss this aspect in detail on Day 4.

    Make sure to test your script, ideally in a temporary folder created for this purpose.
    Did you consider the possibility no files of the given extension exist?
\end{exercise}
