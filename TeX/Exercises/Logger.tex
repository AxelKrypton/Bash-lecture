%-------------------------------%
%  Author: Alessandro Sciarra   %
%    Date: 17 Sep 2019          %
%-------------------------------%

\begin{exercise}[Inspirational]{A (colourful) logger}
    Now that you are more confident with functions, try to implement some functionality to easily log information to the user.
    Often such a functionality is called logger.
    We will focus on a basic version, but with a bit of imagination you can improve it.
    \begin{enumerate}
        \item Start with two basic functions which should simply print what they receive to the output, prefixing it with a label and using a given colour.
              Using them via
              \begin{lstlisting}[style=MyBash]
                  PrintInfo 'An informational message'
                  PrintError 'An error message'
              \end{lstlisting}
              should give\\[0.3ex]
              {
                \ttfamily
                \hspace*{5mm}\color{ForestGreen}\underline{INFO}: An informational message\\
                \hspace*{5mm}\color{red}\underline{ERROR}: An error message\\[0.5ex]
              }
              where you can also put the label in \textbf{bold} if you wish.
              Write error messages to standard error.
        \item Work on the functions in a way such that each parameter is printed on a new line, with a hanged indentation with respect to the label.
              For example, 
              \begin{lstlisting}[style=MyBash]
                  PrintError 'An error message' 'which spans two lines' 'or three'
              \end{lstlisting}
              should give\\[0.3ex]
              {
                \ttfamily\color{red}
                \hspace*{5mm}\underline{ERROR}:~An error message\\
                \hspace*{5mm}\phantom{\underline{ERROR}:}~which spans two lines\\
                \hspace*{5mm}\phantom{\underline{ERROR}:}~or three
              }
        \item You should have noticed that there is a lot of code duplication.
              Write a generic \bash|Logger| function which take a label as first argument (either \texttt{INFO} or \texttt{ERROR}) and does what the other two functions where doing, depending on the label.
              Use the \bash|Logger| in the \bash|PrintInfo| and \bash|PrintError| functions.
              It is up to you if you want to use the \bash|Logger| everywhere or having a wrapper per level.
        \item Add a warning level to your logger, with the correspondent \bash|PrintWarning| wrapper.
        \item Additional optional level might be \texttt{FATAL}, \texttt{INTERNAL}, \texttt{DEBUG} and \texttt{TRACE}.
              A fatal error is an error which causes a script to terminate, while an error might not necessarily be cause of abort.
              Internal messages might be thought to be for developers.
              Debug and trace levels are thought for low level messages and they should be switched off in normal, production runs.
              Use an environment variable to set the logger level.
              Give a meaningful priority to your logger levels and make it such that only the levels more important than the chosen level are switched on.
    \end{enumerate}
\end{exercise}

