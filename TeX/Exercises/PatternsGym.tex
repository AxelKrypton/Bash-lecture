%-------------------------------%
%  Author: Alessandro Sciarra   %
%    Date: 3 Jul 2019           %
%-------------------------------%

\begin{exercise}[Inspirational]{Patterns gym}
    Create a new directory, move into it and run the command
    \begin{lstlisting}[style=MyBash]
        touch file{1..20}{.{dat,png,txt},\ backup.dat,_bkp.png}
    \end{lstlisting}
    checking afterwards what happened using \bash|ls|.

    Think about how to achieve the following tasks, just using what you learnt today.
    \begin{itemize}
        \item List only files with the \bash|.dat| extension.
        \item List only files with number 13 in the name.
        \item List only backup files and then all but backup files.
        \item List only files containing a space in the name.
        \item List all but files containing a space in the name.
        \item List files with a number that is multiple of 5 before the dot.
    \end{itemize}

    Now that you practised a bit with patterns, let us improve names of the files in this folder.
    \begin{enumerate}
        \item Rename the files containing a space replacing it by an underscore.
        \item Change the \bash|_bkp.png| suffix into \bash|_backup.png|.
        \item Add a leading 0 to numbers in files whose name contains a number smaller than 10.
    \end{enumerate}

    Note, that you might be tempted to iterate over files~--~and often it is necessary to do so~--~but the tasks above in this particular exercise can be achieved without using flow constructs, which will be discussed in detail tomorrow.
    For the renaming part, use the
    \begin{lstlisting}[style=MyBash]
        rename -s 'from' 'to' [file]...
    \end{lstlisting}
    command.
    You might need to install the \bash|rename| command via e.g.
    \begin{lstlisting}[style=MyBash, emph={[6]{brew}}]
        brew install rename
    \end{lstlisting}
    or
    \begin{lstlisting}[style=MyBash, emph={[6]{apt-get}}, alsoletter={-}]
        apt-get install rename
    \end{lstlisting}
    depending on your OS\footnote{In Fedora and RedHat-derived Linux distributions you need to use \bash|prename| and in other less common distributions the name might be even different.}.

    \vspace{5mm}
    \textbf{Bonus problem:} Can you think of an easy way to print all bit strings that can build up 1 byte (8 bits)?
    The output of your command should look like the following.
    \begin{lstlisting}[style=MyBash]
        00000000
        00000001
        00000010
        00000011
          ....
        11111100
        11111101
        11111110
        11111111
    \end{lstlisting}
    \vspace{5mm}
\end{exercise}