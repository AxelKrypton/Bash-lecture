%-------------------------------%
%  Author: Alessandro Sciarra   %
%    Date: 15 Oct 2020          %
%-------------------------------%

\begin{exercise}[Instructive]{Writing a simple autocompletion script}
    Consider again the basic example discussed today in the lecture.
    We saw that the implemented autocompletion mechanism is not ideal, since the same command line option is displayed pressing \texttt{<TAB-TAB>} although already given.
    How would you fix this aspect of the script, supposing that each option of the \bash|measure| script shall be given only once?
    You can create the autocompletion script in any folder (e.g. where you solve exercises) and source it in your terminal (after each modification).
\end{exercise}