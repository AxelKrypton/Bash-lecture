%-------------------------------%
%  Author: Alessandro Sciarra   %
%    Date: 23 Sep 2020          %
%-------------------------------%

%~~~~~~~~~~~~~~~~~~~~~~~~~~~~~~~~~~~~~~~~~~~~%
\begin{frame}{Exit code}
    \vspace{-3mm}
    Every command in Bash terminates with an exit code:
    \begin{description}
        \item[\texttt{\$?}]
            Shows the exit code of the last foreground process that terminated\\
            It is a 8-bit integer $\;$\texttt{\$?}${}\in{}$\texttt{\{0,\dots,255\}} \hfill\Remark{Indeed, only the least significant 8 bits count}
        \item[\texttt{0}] Denotes success
        \item[$\neq$\texttt{0}] Denotes failure, in general the meaning is up to the command
        \item[\texttt{1}] Miscellaneous errors
        \item[\texttt{2}] Misuse of shell builtins
        \item[\texttt{126}] Command invoked cannot execute
        \item[\texttt{127}] \PP{\texttt{"}command not found\texttt{"}} error
        \item[\texttt{128}] Invalid argument to exit
        \item[\texttt{128 + n}] Fatal error signal \texttt{"n"} \Remark{For example, \PB{\texttt{130}} means that the script terminated by Control-C}
    \end{description}
    \begin{varblock}{alerted}[0.45\textwidth]{Which exit code should I use?}
        \PQ{Avoid reserved one, be consistent!}
    \end{varblock}
\end{frame}
%~~~~~~~~~~~~~~~~~~~~~~~~~~~~~~~~~~~~~~~~~~~~%
\begin{frame}[fragile]{The \bash|if| keyword}
    \vspace{-3mm}
    \begin{itemize}
        \item Since it is a keyword, it requires a precise syntax (not a surprise)
        \item It executes a command (or a set of commands) and checks that command's exit code to see whether it was successful
        \item Different layouts possible, choose a readable one!
    \end{itemize}

    \begin{lstlisting}[style=MyBash, numbers=none]
        if@|\textvisiblespace|@COMMAND
        then
            # COMMAND's exit code was 0
        else
            # COMMAND's exit code was different from 0
        fi

        if@|\textvisiblespace|@COMMAND; then
            # COMMAND's exit code was 0
        elif@|\textvisiblespace|@ANOTHER_COMMAND; then
            # ANOTHER_COMMAND's exit code was 0
        else
            # ANOTHER_COMMAND's exit code was different from 0
        fi
    \end{lstlisting}
\end{frame}
%~~~~~~~~~~~~~~~~~~~~~~~~~~~~~~~~~~~~~~~~~~~~%
\begin{frame}[fragile]{The command in a conditional block}
    \vspace{-1mm}
    \begin{itemize}
        \setlength{\itemsep}{2mm}
        \item In principle it can be any command
        \item There are specifically designed commands to test things:\\[0.5em]
                \begin{tabular}{>{\ttfamily\color{PB}}rl}
                    \makecell[rt]{test \\ \tc{fg}{\sffamily or} [} &
                    \makecell[lt]{A normal command that reads its arguments and does \\
                                  some checks with them. The \bash{[} variant requires a \texttt{]} as last argument.}\\[2em]
                    \makecell[rt]{\tc{keywords-color}{[[}} &
                    \makecell[lt]{A special shell keyword that offers more versatility\\
                                  like \PP{pattern matching} and \PP{regex support}} \\
                \end{tabular}
        \item Multiple commands can be concatenated using control operators \texttt{\&\&} and \texttt{||}
    \end{itemize}
    \begin{varblock}{}[0.9\textwidth]{Which syntax should I prefer?}
        Whenever you are making a \textbf{Bash} script, you should always use \texttt{\tc{keywords-color}{[[}} rather than \bash{[}.\\
        If portability is an issue, e.g.\ you are writing a \textbf{sh}ell script, \\ you should use \bash{[}, because it is far more portable.
    \end{varblock}
\end{frame}
%~~~~~~~~~~~~~~~~~~~~~~~~~~~~~~~~~~~~~~~~~~~~%
\begin{frame}[fragile]{The \bash|test| command and its friend \bash{[}}
    \vspace{-4mm}
    \begin{center}
        \begin{minipage}{0.85\textwidth}
            \begin{description}[<only@1>][\texttt{-v VARIABLE}]
                \item[\texttt{-e FILE}] True if file exists
                \item[\texttt{-f FILE}] True if file is a regular file (not a directory or device file)
                \item[\texttt{-d FILE}] True if file is a directory
                \item[\texttt{-s FILE}] True if file exists and is not empty
                \item[\texttt{-z STRING}] True if the string is empty (it's length is zero)
                \item[\texttt{-n STRING}] True if the string is not empty (it's length is not zero)
                \item[\texttt{-v VARIABLE}] True if the shell variable is set (has been assigned a value)$^\star$
            \end{description}
        \end{minipage}
        \begin{minipage}{0.9\textwidth}
            \begin{description}[<only@2>][\texttt{STRING != STRING}]
                \item[\texttt{STRING  = STRING}] True if the first string is identical to the second
                \item[\texttt{STRING != STRING}] True if the first string is not identical to the second
                \item[\texttt{STRING \textbackslash< STRING}] True if the first string sorts before the second
                \item[\texttt{STRING \textbackslash> STRING}] True if the first string sorts after the second
                \item[\texttt{! EXPR}] Inverts the result of the expression (logical NOT)
            \end{description}
        \end{minipage}
        \begin{minipage}{0.9\textwidth}
            \begin{description}[<only@3>][\texttt{INT -eq INT}]
                \item[\texttt{INT -eq INT}] True if both integers are identical
                \item[\texttt{INT -ne INT}] True if the integers are not identical
                \item[\texttt{INT -lt INT}] True if the first integer is less than the second
                \item[\texttt{INT -gt INT}] True if the first integer is greater than the second
                \item[\texttt{INT -le INT}] True if the first integer is less than or equal to the second
                \item[\texttt{INT -ge INT}] True if the first integer is greater than or equal to the second
            \end{description}
        \end{minipage}
    \end{center}
    \begin{onlyenv}<1>
        \begin{lstlisting}[style=MyBash, style=oddnumbers, xleftmargin=10mm, xrightmargin=10mm, aboveskip=2mm]
            $ ls -d */
            |+TeX+|
            $ if [ -d 'TeX' ]; then echo 'YES'; else echo 'NO'; fi
            |+YES+|
            $ if [ -e 'TeX' ]; then echo 'YES'; else echo 'NO'; fi
            |+YES+|
            $ if [ -s 'TeX' ]; then echo 'YES'; else echo 'NO'; fi
            |+YES+|
            $ if [ -f 'TeX' ]; then echo 'YES'; else echo 'NO'; fi
            |+NO+|
        \end{lstlisting}
    \end{onlyenv}
    \begin{onlyenv}<2>
        \begin{lstlisting}[style=MyBash, style=oddnumbers, xleftmargin=-1mm, xrightmargin=-2mm, aboveskip=2mm]
            $ aVar="Kal El"
            $ bVar="Clark Kent"
            $ [ ${aVar} = ${aVar} ]
            bash: [: too many arguments
            $ if [ "${aVar}" = "${aVar}" ]; then echo 'YES'; else echo 'NO'; fi
            |+YES+|
            $ if [ "${aVar}" \> "${bVar}" ]; then echo 'YES'; else echo 'NO'; fi
            |+YES+|
            $ if [ 319 \< 7 ]; then echo 'YES'; else echo 'NO'; fi
            |+YES+| # 319 is < than 7 but it is not less than 7...
            $ unset aVar bVar
        \end{lstlisting}
    \end{onlyenv}
    \begin{onlyenv}<3>
        \begin{lstlisting}[style=MyBash, style=oddnumbers, xleftmargin=8mm, xrightmargin=8mm, aboveskip=2mm]
            $ if [ 319 -lt 7 ]; then echo 'YES'; else echo 'NO'; fi
            |+NO+|
            $ if [ 7 -ne 7 ]; then echo 'YES'; else echo 'NO'; fi
            |+NO+|
            $ if [ 7 -eq 7 ]; then echo 'YES'; else echo 'NO'; fi
            |+YES+|
            $ if [ 7 -gt 7 ]; then echo 'YES'; else echo 'NO'; fi
            |+NO+|
            $ if [ 7 -ge 7 ]; then echo 'YES'; else echo 'NO'; fi
            |+YES+|
        \end{lstlisting}
    \end{onlyenv}
    \begin{tikzpicture}[remember picture, overlay]
        \node[anchor=north east, font=\scriptsize, visible on=<1>] at ($(current page.north east)-(1mm,1mm)$) {$^\star$\PB{Since Bash v4.2}};
    \end{tikzpicture}
    \PrepareURLsymbol[PB]
    \FrameRemark{Many more tests are supported $\to$ for a complete list refer to the \URL*{https://www.gnu.org/software/bash/manual/}{Bash manual v5.2 (section 6.4).}}
\end{frame}
%~~~~~~~~~~~~~~~~~~~~~~~~~~~~~~~~~~~~~~~~~~~~%
\begin{frame}[fragile]{The \bash{[[} keyword}
    \vspace{-2mm}
    \begin{itemize}[<only@1>]
        \item It supports the tests supported by the \bash{test} and \bash{[} commands$^{\star}$
        \item String equality (\texttt{=} or \texttt{==}) and inequality (\texttt{!=}) comparison is changed to \alert{pattern matching} by default.
              \PQ{\textbf{Patterns must be on the right hand side of the (in)equality.}}
              Each quoted part of the pattern is matched literally.
        \item \alert{It does not allow word-splitting} of its arguments!
        \item However, be aware that simple strings still have to be quoted properly
        \item Few more additional tests supported:
              \begin{description}[\texttt{STRING =\textasciitilde\ REGEX}]
                \item[\texttt{STRING =\textasciitilde\ REGEX}] True if the string matches the regex pattern
                \item[\texttt{( EXPR )}] Parentheses can be used to change the evaluation precedence
                \item[\texttt{EXPR \&\& EXPR}] True if both expressions are true (logical AND)\\[-0.5em] \Remark[0pt]{it does not evaluate the second expression if the first already turns out to be false}
                \item[\texttt{EXPR || EXPR}] True if either expression is true (logical OR) \\[-0.5em] \Remark[0pt]{it does not evaluate the second expression if the first already turns out to be true}
              \end{description}
        \item The alphabetically sorted test operators (\texttt{<} and \texttt{>}) do not need to be escaped
    \end{itemize}
    \begin{onlyenv}<2>
        \begin{lstlisting}[style=MyBash, aboveskip=-1mm, style=oddnumbers, style=smaller, xleftmargin=1mm, xrightmargin=1mm]
            $ aVar="Day_1.tex"; bVar='Hello world'
            # Pattern matching VS string comparison
            $ if [[ ${aVar} = *.tex ]]; then echo 'YES'; else echo 'NO'; fi
            |+YES+|
            $ if [[ *.tex = ${aVar} ]]; then echo 'YES'; else echo 'NO'; fi
            |+NO+|
            $ if [[ ${aVar} = "*.tex" ]]; then echo 'YES'; else echo 'NO'; fi
            |+NO+|
            $ if [[ ${aVar} = Day_?.tex && ${bVar} = [hH]* ]]; then echo 'YES'; else echo 'NO'; fi
            |+YES+|
            # Simple strings still have to be quoted properly!
            $ if [[ ${bVar} = Hello world ]]; then echo 'YES'; else echo 'NO'; fi
            |+bash: syntax error in conditional expression
            bash: syntax error near `world'+|
            $ if [[ ${bVar} = 'Hello world' ]]; then echo 'YES'; else echo 'NO'; fi
            |+YES+| # No word splitting occurred!
            $ if [[ ${emptyVar} = '' ]]; then echo 'YES'; else echo 'NO'; fi
            |+YES+|
            $ if [ ${emptyVar} = '' ]; then echo 'YES'; else echo 'NO'; fi
            |+bash: [: =: unary operator expected+|
            |+NO+|
            # Quotes necessary to use test [ command to check for empty variables
            $ if [ "${emptyVar}" = '' ]; then echo 'YES'; else echo 'NO'; fi
            |+YES+|
            $ unset aVar bVar
        \end{lstlisting}
    \end{onlyenv}
    \FrameRemark{$^{\star}$ Except \PB{\texttt{~EXPR -a EXPR~}} and \PB{\texttt{~EXPR -o EXPR~}} which should in any case not be used!}
\end{frame}
%~~~~~~~~~~~~~~~~~~~~~~~~~~~~~~~~~~~~~~~~~~~~%
\begin{frame}[fragile]{Control operators}
    \begin{overlayarea}{\textwidth}{0.56\textheight}
        \begin{description}[<only@1>][\texttt{\&\&}]
            \setlength{\itemsep}{5mm}
            \item[\texttt{\&\&}]
                 Used to build AND lists: Run one command only if another exited successfully
                 \begin{lstlisting}[style=MyBash, numbers=none, aboveskip=2mm]
                     COMMAND_1 && COMMAND_2
                     # Equivalent to
                     if COMMAND_1; then COMMAND_2; fi
                 \end{lstlisting}
            \item[\texttt{||}]
                 Used to build OR lists: Run one command only if another exited unsuccessfully
                 \begin{lstlisting}[style=MyBash, numbers=none, aboveskip=2mm]
                     COMMAND_1 || COMMAND_2
                     # Equivalent to
                     if ! COMMAND_1; then COMMAND_2; fi
                 \end{lstlisting}
        \end{description}
        \begin{onlyenv}<2-3>
            \begin{lstlisting}[style=MyBash, numbers=none, belowskip=-5mm]
                # The use of control operators is handy in simple cases
                [[ ${PATH} ]] && echo 'PATH variable set and non empty!'
                # Mostly they are used in script
                rm file || echo 'Could not delete file!'
                mkdir TeX && cd TeX
                source AuxiliaryOps.bash || exit 1
                # Avoid complicated one-liners which might easily be wrong!
                grep -q goodword "${file}" && ! grep -q badword "${file}" && rm "${file}" || echo "Couldn't delete: ${file}"
            \end{lstlisting}
            \centerline{\small\PQ{What happens if the first \bash{grep} fails (sets the exit status to 1)?}}
            \begin{uncoverenv}<3>
                \begin{lstlisting}[style=MyBash, numbers=none, aboveskip=3mm]
                    grep -q goodword "${file}" && ! grep -q badword "${file}" && { rm "${file}" || echo "Couldn't delete: ${file}"; }
                \end{lstlisting}
            \end{uncoverenv}
        \end{onlyenv}
    \end{overlayarea}
    \begin{varblock}{alerted}[0.8\textwidth]{Do not get overzealous with conditional operators}
        They can make your script hard to understand and sometimes you need fancy grouping to get your logic right!
    \end{varblock}
\end{frame}
%~~~~~~~~~~~~~~~~~~~~~~~~~~~~~~~~~~~~~~~~~~~~%
\begin{frame}[fragile]{Regular expressions in Bash}
    \begin{itemize}[<only@1>]
        \item Regular expressions are similar to Glob Patterns, but not for filename matching
        \item Since v3.0, Bash supports the \texttt{=\textasciitilde} operator to the \bash{[[} keyword
        \item This operator matches the string that comes before it against the regular expression pattern that follows it and it returns
              \begin{description}
                  \item[0] when the string matches the pattern
                  \item[1] If the string does not match the pattern
                  \item[2] In case the pattern's syntax is invalid
              \end{description}
        \item Bash uses the Extended Regular Expression dialect
              \begin{itemize}
                  \item[$\circ$] \URL[PS]{http://pubs.opengroup.org/onlinepubs/9699919799/basedefs/V1_chap09.html}{POSIX standard}
                  \item[$\circ$] \URL[PQ]{https://www.regular-expressions.info/posix.html}{The Premier website about Regular Expressions}
              \end{itemize}
        \item Regular Expression patterns that use capturing groups (parentheses) will have their captured strings assigned to the \bash|BASH_REMATCH| variable (array) for later retrieval
    \end{itemize}
    \begin{onlyenv}<2>
        \begin{lstlisting}[style=MyBash, style=oddnumbers, style=smaller]
            $ echo "${LANG}"
            |+en_GB.utf8+|
            $ langRegex='(..)_(..)'
            $ if [[ ${LANG} =~ ${langRegex} ]]
            > then
            >     echo "Your country code (ISO 3166-1-alpha-2) is ${BASH_REMATCH[2]}."
            >     echo "Your language code (ISO 639-1) is ${BASH_REMATCH[1]}."
            > else
            >     echo "Your locale was not recognised"
            > fi
            |+Your country code (ISO 3166-1-alpha-2) is GB.+|
            |+Your language code (ISO 639-1) is en.+|
        \end{lstlisting}
        \begin{varblock*}{}[0.9\textwidth]{General remarks}
            \begin{itemize}
                \item The best way to always be compatible is to put your regex in a variable and expand that variable in \bash{[[} without quotes, as we showed above.
                \item The pattern matching achieved by the \texttt{=} and \texttt{!=} operators can be achieved using regex: \alert{Find your way!}
            \end{itemize}
        \end{varblock*}
    \end{onlyenv}
    \FrameRemark<2>{The syntax \bash{"\$\{BASH\_REMATCH[1]\}"}, e.g.\ how to use an array variable, will be discussed in detail at a later point.}
\end{frame}











