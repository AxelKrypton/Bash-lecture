%-------------------------------%
%  Author: Alessandro Sciarra   %
%    Date: 24 Jun 2019          %
%-------------------------------%

%~~~~~~~~~~~~~~~~~~~~~~~~~~~~~~~~~~~~~~~~~~~~%
\begin{frame}[fragile]{The first building block}
    Two flavours:
    \begin{itemize}
        \item Parameters/Variables\\
              {\scriptsize $\to$ Created and updated by the user}
        \item Special parameters\tikzmark{origin}\\
              {\scriptsize $\to$ Read-only and pre-set by Bash}
    \end{itemize}
    \begin{tikzpicture}[remember picture, overlay, line join=round]
        \pgfmathsetmacro{\cubex}{1}
        \pgfmathsetmacro{\cubey}{1}
        \pgfmathsetmacro{\cubez}{1}
        \coordinate (O) at ([xshift=30mm]origin);
        \draw[PS, fill=PS!40] (O) ++(\cubex,\cubey,0) -- ++(0,0,-\cubez) -- ++(0,-\cubey,0) -- ++(0,0,\cubez) -- cycle;
        \draw[PS, fill=PS!10] (O) ++(\cubex,\cubey,0) -- ++(-\cubex,0,0) -- ++(0,0,-\cubez) -- ++(\cubex,0,0) -- cycle;
        \draw[PS] (O) ++(0,\cubey,-\cubez) -- ++(0,-\cubey,0);
        \draw[PS, fill=PS!30] (O)                     -- ++(\cubex,0,0) -- ++(0,\cubey,0) -- ++(-\cubex,0,0) -- cycle;
        \node at ($(O)+(\cubex/2,\cubey/2,0)$) {Name};
        \path[from] ($(O)+(\cubex/2,\cubey,-\cubez/2)$) edge[out=90, in=180] node[pos=1, anchor=west] (content) {\underline{Content}:} ++(1.5,0.5,0);
        \node[font=\scriptsize, text=PS, below = 1mm of content.210, anchor=north west, inner sep=0] {
            \begin{tabular}{>{$\star\,$}l}
                strings \\
                integers \\
                indexed arrays \\
                associative arrays \\
            \end{tabular}
        };
    \end{tikzpicture}
    \begin{overlayarea}{\textwidth}{0.3\textheight}
        \vspace{-6mm}
        \begin{varblock*}{}[0.6\textwidth]{The variable name (also referred to as an \textbf{identifier})}<only@1>
            A word consisting only of
            \begin{itemize}
                \item \PB{letters}, \PB{digits} and \PB{underscores}
                \item and \PB{beginning with a letter} or \PB{an underscore}
            \end{itemize}
        \end{varblock*}
        \begin{varblock*}{}[0.98\textwidth]{\textbf{Assignment}}<only@2>
            \begin{lstlisting}[style=MyBash, numbers=none, xrightmargin=26mm, xleftmargin=26mm]
                $ variableName=variableContent
            \end{lstlisting}
            \begin{itemize}
                \item If not existing, the \alert{global} variable \bash|variableName| is created, and the content \bash|variableContent| is put into it
                \item If existing, the content of \bash|variableName| is set to \bash|variableContent|
                \item If \bash|variableName| exists and it is read-only, an error occurs
            \end{itemize}
        \end{varblock*}
        \begin{varblock}{}[0.96\textwidth]{Accessing the content: the \textbf{parameter expansion}}<only@3>
            Use the \texttt{\$} special character to tell Bash that you want to use the content of a variable
            \begin{lstlisting}[style=MyBash, numbers=none, belowskip=-5mm]
                $ prefix='Day_'
                $ day='Monday'
                $ echo "${prefix}0.pdf are the slides for ${day}"
                |+Day_0.pdf are the slides for Monday+|
            \end{lstlisting}
            \alert{Always} using curly braces \alert{\texttt{\$\{...\}}} can be considered \alert{good programming practice}!
        \end{varblock}
    \end{overlayarea}
\end{frame}
%~~~~~~~~~~~~~~~~~~~~~~~~~~~~~~~~~~~~~~~~~~~~%
\begin{frame}[fragile]{Crucial to know (I)}{\ldots{}and not to forget!}
    \begin{lstlisting}[style=MyBash, numbers=none]
        #This is WRONG
        $ variableName@|\textvisiblespace|@=@|\textvisiblespace|@variableContent   # spaces around = sign!
        |+bash: variableName: command not found+|
    \end{lstlisting}
    \bigskip
    \begin{enumerate}
        \item Bash will not know that you are attempting to assign something
        \item The parser will see \bash|variableName| with no \bash{=} and treat it as a command name
        \item \bash{=} and \bash|variableContent| are then passed to it as arguments
    \end{enumerate}
    \bigskip
    \begin{varblock}{}[0.72\textwidth]{}
        \Large\PB{If you think about it for a moment, it makes sense!}
    \end{varblock}
    
\end{frame}
%~~~~~~~~~~~~~~~~~~~~~~~~~~~~~~~~~~~~~~~~~~~~%
\begin{frame}[fragile]{Crucial to know (II)}{\ldots{}and not to forget!}
    \begin{overlayarea}{\textwidth}{0.5\textheight}
        \vspace{-9mm}
        \begin{varblock}{alerted}[0.98\textwidth]{\textbf{Attention!}}
            \PQ{After parameter expansion, Bash may still perform additional manipulations on the result!}
        \end{varblock}
        \begin{onlyenv}<1-2>
            \begin{lstlisting}[style=MyBash, belowskip=-4mm]
                $ today=Monday
                $ echo "Today is ${today}"
                |+Today is Monday+|
                # Bash takes the content of the variable today
                # and replaces ${today} by Monday. Equivalent to:
                $ echo "Today is Monday"
                |+Today is Monday+|
            \end{lstlisting}
            \centerline{Everything seems to work as expected\ldots\ but:}
        \end{onlyenv}
        \begin{onlyenv}<3-4>
            Why did it not work?
        \end{onlyenv}
        \begin{onlyenv}<4>
            \begin{enumerate}
                \item Bash replaced \texttt{\$}\bash|{song}| by its content
                \item Word splitting occurred before the command was executed!
                \item \bash|rm| was run with 2 arguments (there is white space between them and it is not quoted!)
            \end{enumerate}
            \begin{lstlisting}[style=MyBash, numbers=none]
                $ rm My song.mp3
            \end{lstlisting}
        \end{onlyenv}
        \begin{onlyenv}<5->
            \begin{lstlisting}[style=MyBash]
                # Please, do not try to put quotes in variables!
                $ song="\"My song.mp3\""
                $ rm ${song}
                |+rm: "My: No such file or directory+|
                |+rm: song.mp3": No such file or directory+|
                # Here the quotes contained in the variable song
                # are literal characters and they are not interpreted
                # as quotes when the rm command is run!!
            \end{lstlisting}
        \end{onlyenv}
        \begin{onlyenv}<6->
            \begin{lstlisting}[style=MyBash, firstnumber=9]
                # CORRECT WAY TO DO IT:
                $ rm "${song}"
            \end{lstlisting}
        \end{onlyenv}
    \end{overlayarea}
    \begin{onlyenv}<2-4>
        \begin{lstlisting}[style=MyBash]
            #This is probably not what you would like to do
            $ song="My song.mp3"
            $ rm ${song}
            |+rm: My: No such file or directory+|
            |+rm: song.mp3: No such file or directory+|
        \end{lstlisting}
    \end{onlyenv}
    \medskip
    \begin{varblock}{alerted}[0.75\textwidth]{How do we fix this?}<only@5->
        \uncover<6>{\PQ{Remember to put double quotes around every parameter expansion!}}
    \end{varblock}
    
\end{frame}




