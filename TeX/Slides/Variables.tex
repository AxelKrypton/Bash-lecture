%-------------------------------%
%  Author: Alessandro Sciarra   %
%    Date: 24 Jun 2019          %
%-------------------------------%

%~~~~~~~~~~~~~~~~~~~~~~~~~~~~~~~~~~~~~~~~~~~~%
\begin{frame}[fragile]{The first building block}
    Parameters come in two flavours:
    \begin{itemize}
        \item Variables (i.e. parameters with a name)\\[0.1em]
              {\scriptsize $\to$ Created and updated by the user \\[-0.5em] $\to$ Available in the environment}
        \item {\color{Gray!50}Special parameters (later)\tikzmark{origin}\\
              {\scriptsize $\to$ Read-only and pre-set by Bash}}
    \end{itemize}
    \begin{tikzpicture}[remember picture, overlay, line join=round]
        \pgfmathsetmacro{\cubex}{1}
        \pgfmathsetmacro{\cubey}{1}
        \pgfmathsetmacro{\cubez}{1}
        \coordinate (O) at ([xshift=30mm]origin);
        \draw[PS, fill=PS!40] (O) ++(\cubex,\cubey,0) -- ++(0,0,-\cubez) -- ++(0,-\cubey,0) -- ++(0,0,\cubez) -- cycle;
        \draw[PS, fill=PS!10] (O) ++(\cubex,\cubey,0) -- ++(-\cubex,0,0) -- ++(0,0,-\cubez) -- ++(\cubex,0,0) -- cycle;
        \draw[PS] (O) ++(0,\cubey,-\cubez) -- ++(0,-\cubey,0);
        \draw[PS, fill=PS!30] (O)                     -- ++(\cubex,0,0) -- ++(0,\cubey,0) -- ++(-\cubex,0,0) -- cycle;
        \node at ($(O)+(\cubex/2,\cubey/2,0)$) {Name};
        \path[from] ($(O)+(\cubex/2,\cubey,-\cubez/2)$) edge[out=90, in=180] node[pos=1, anchor=west] (content) {\underline{Content}:} ++(1.5,0.5,0);
        \node[font=\scriptsize, text=PS, below = 1mm of content.210, anchor=north west, inner sep=0] {
            \begin{tabular}{>{$\star\,$}l}
                strings \\
                integers \\
                indexed arrays \\
                associative arrays \\
                \tc{PS!50}{references} \\
            \end{tabular}
        };
    \end{tikzpicture}
    \begin{overlayarea}{\textwidth}{0.3\textheight}
        \vspace{-6mm}
        \begin{varblock*}{}[0.6\textwidth]{The variable name (also referred to as an \textbf{identifier})}<only@1>
            A word consisting only of
            \begin{itemize}
                \item \PB{letters}, \PB{digits} and \PB{underscores}
                \item and \PB{beginning with a letter} or \PB{an underscore}
            \end{itemize}
        \end{varblock*}
        \begin{varblock*}{}[0.98\textwidth]{\textbf{Assignment}}<only@2>
            \begin{lstlisting}[style=MyBash, numbers=none, xrightmargin=26mm, xleftmargin=26mm]
                $ variableName=variableContent
            \end{lstlisting}
            \begin{itemize}
                \item If not existing, the \alert{global} variable \bash|variableName| is created, and the content \bash|variableContent| is put into it
                \item If existing, the content of \bash|variableName| is set to \bash|variableContent|
                \item If \bash|variableName| exists and it is read-only, an error occurs
            \end{itemize}
        \end{varblock*}
        \begin{varblock}{}[0.96\textwidth]{Accessing the content: the \textbf{parameter expansion}}<only@3>
            Use the \texttt{\$} special character to tell Bash that you want to use the content of a variable
            \begin{lstlisting}[style=MyBash, numbers=none, belowskip=-5mm]
                $ prefix='Day_'
                $ day='Monday'
                $ echo "${prefix}0.pdf are the slides for ${day}"
                |+Day_0.pdf are the slides for Monday+|
            \end{lstlisting}
            \alert{Always} using curly braces \alert{\texttt{\$\{...\}}} can be considered \alert{good programming practice}!
        \end{varblock}
        \begin{varblock}{alerted}[0.9\textwidth]{Expanding undeclared variable}<only@4>
            Variables in Bash do not have to be declared, as they do in languages like C!\\
            If you try to read an undeclared variable, the result is the empty string.\\
            \begin{lstlisting}[style=MyBash, numbers=none, belowskip=-5mm, xleftmargin=20mm, xrightmargin=20mm]
                $ invisibleVariable='Hello'
                $ echo _${invisibelVariable}_
                |+__+| # Is this magic?! Well, no...
            \end{lstlisting}
            \alert{By default, you get no warnings or errors!}
        \end{varblock}
        \FrameRemark[4]{There is a way to activate a different behaviour of Bash when performing parameter expansion on unset variables: $\;$\URL[PB]{https://www.gnu.org/software/bash/manual/html_node/The-Set-Builtin.html}{\texttt{set -u}}}
    \end{overlayarea}
\end{frame}
%~~~~~~~~~~~~~~~~~~~~~~~~~~~~~~~~~~~~~~~~~~~~%
\begin{frame}[fragile]{Variables Types}
    \begin{onlyenv}<1>
        \vspace{-5mm}
        \begin{varblock}{example}[0.75\textwidth]{Bash is not a typed language}
            It does have a few different types of variables, though! \\
            It is more about activate particular rules when acting on variables.
        \end{varblock}
        \vspace{-5mm}
        \begin{columns}
            \begin{column}{\dimexpr\paperwidth-10mm}
                \begin{description}[\textbf{Associative arrays:}]
                    \setlength{\itemsep}{1mm}
                    \item[\textbf{Array:}]
                        \bash|declare -a variableName|\\
                        The variable is an array of strings
                    \item[\textbf{Associative array:}]
                        \bash|declare -A variableName|\\
                        The variable is an associative array of strings (v4.0 or higher)
                    \item[\textbf{Integer:}]
                        \bash|declare -i variableName|\\
                        The variable holds an integer\\
                        Assigning values to this variable automatically triggers Arithmetic Evaluation
                    \item[\textbf{Read only:}]
                        \bash|declare -r variableName|\\
                        The variable can no longer be modified or unset
                    \item[\textbf{Export:}]
                        \bash|declare -x variableName|\\
                        The variable is marked for export, i.e. it will be inherited by any child process
                \end{description}
            \end{column}
        \end{columns}
    \end{onlyenv}
    \begin{onlyenv}<2>
        \begin{lstlisting}[style=MyBash, style=oddnumbers, belowskip=-5mm]
            $ aVar=5; aVar+=2; echo "$aVar"; unset aVar
            52
            $ aVar=5; let aVar+=2; echo "$aVar"; unset aVar
            7
            $ declare -i aVar=5; aVar+=2; echo "$aVar"; unset aVar
            7
            $ aVar=5+2; echo "$aVar"; unset aVar
            5+2
            $ declare -i aVar=5+2; echo "$aVar"; unset aVar
            7
            $ declare -i aVar=5; aVar+=aVar; echo "$aVar"; unset aVar
            10
            $ declare -i aVar=5; aVar+='foo'; echo "$aVar"; unset aVar
            5
            # "foo" refers to the variable foo in arithmetic evaluation
        \end{lstlisting}
        \begin{varblock}{alerted}[0.9\textwidth]{The use of integer variables is exceedingly rare!}
            Most experienced shell programmers prefer to use explicit arithmetic commands (with \texttt{((\ldots))} or \bash|let|) when they want to perform arithmetic!
        \end{varblock}
    \end{onlyenv}
\end{frame}
%~~~~~~~~~~~~~~~~~~~~~~~~~~~~~~~~~~~~~~~~~~~~%
\begin{frame}[fragile]{Crucial to know (I)}{\ldots{}and not to forget!}
    \begin{lstlisting}[style=MyBash, numbers=none]
        #This is WRONG
        $ variableName@|\textvisiblespace|@=@|\textvisiblespace|@variableContent   # spaces around = sign!
        |+bash: variableName: command not found+|
    \end{lstlisting}
    \bigskip
    \begin{enumerate}
        \item Bash will not know that you are attempting to assign something
        \item The parser will see \bash|variableName| with no \bash{=} and treat it as a command name
        \item \bash{=} and \bash|variableContent| are then passed to it as arguments
    \end{enumerate}
    \bigskip
    \begin{varblock}{}[0.72\textwidth]{}
        \Large\PB{If you think about it for a moment, it makes sense!}
    \end{varblock}
    
\end{frame}
%~~~~~~~~~~~~~~~~~~~~~~~~~~~~~~~~~~~~~~~~~~~~%
\begin{frame}[fragile]{Crucial to know (II)}{\ldots{}and not to forget!}
    \begin{overlayarea}{\textwidth}{0.5\textheight}
        \vspace{-9mm}
        \begin{varblock}{alerted}[0.98\textwidth]{\textbf{Attention!}}
            \PQ{After parameter expansion, Bash may still perform additional manipulations on the result!}
        \end{varblock}
        \begin{onlyenv}<1-2>
            \begin{lstlisting}[style=MyBash, belowskip=-4mm]
                $ today=Monday
                $ echo "Today is ${today}"
                |+Today is Monday+|
                # Bash takes the content of the variable today
                # and replaces ${today} by Monday. Equivalent to:
                $ echo Today is Monday
                |+Today is Monday+|
            \end{lstlisting}
            \centerline{Everything seems to work as expected\ldots\ but:}
        \end{onlyenv}
        \begin{onlyenv}<3-4>
            Why did it not work?
        \end{onlyenv}
        \begin{onlyenv}<4>
            \begin{enumerate}
                \item Bash replaced \texttt{\$}\bash|{song}| by its content
                \item Word splitting occurred before the command was executed!
                \item \bash|rm| was run with 2 arguments (there is white space between them and it is not quoted!)
            \end{enumerate}
            \begin{lstlisting}[style=MyBash, numbers=none]
                $ rm My song.mp3
            \end{lstlisting}
        \end{onlyenv}
        \begin{onlyenv}<5->
            \begin{lstlisting}[style=MyBash]
                # Please, do not try to put quotes in variables!
                $ song="\"My song.mp3\""
                $ rm ${song}
                |+rm: "My: No such file or directory+|
                |+rm: song.mp3": No such file or directory+|
                # Here the quotes contained in the variable song
                # are literal characters and they are not interpreted
                # as quotes when the rm command is run!!
            \end{lstlisting}
        \end{onlyenv}
        \begin{onlyenv}<6->
            \begin{lstlisting}[style=MyBash, firstnumber=9]
                # CORRECT WAY TO DO IT:
                $ rm "${song}"
            \end{lstlisting}
        \end{onlyenv}
    \end{overlayarea}
    \begin{onlyenv}<2-4>
        \begin{lstlisting}[style=MyBash]
            #This is probably not what you would like to do
            $ song="My song.mp3"
            $ rm ${song}
            |+rm: My: No such file or directory+|
            |+rm: song.mp3: No such file or directory+|
        \end{lstlisting}
    \end{onlyenv}
    \medskip
    \begin{varblock}{alerted}[0.75\textwidth]{How do we fix this?}<only@5->
        \uncover<6>{\PQ{Remember to put double quotes around every parameter expansion!}}
    \end{varblock}
\end{frame}
%~~~~~~~~~~~~~~~~~~~~~~~~~~~~~~~~~~~~~~~~~~~~%
\begin{frame}[fragile]{Available Variables}{\URL[PB]{https://www.gnu.org/software/bash/manual/}{Bash manual v5.0 pages 73-84}}
    \vspace{-5mm}
    \begin{center}
        \begin{lstlisting}[style=MyBash, numbers=none, basicstyle={\ttfamily\tiny\color{basic-color}}]
            # Bourne Shell Variables (For some, Bash sets a default)
            CDPATH  HOME  IFS  MAIL  MAILPATH  OPTARG  OPTIND  PATH  PS1  PS2
            # Bash Variables (i.e. variables that are set or used by Bash)
            BASH                    COMP_POINT        HISTFILESIZE     OSTYPE
            BASHOPTS                COMP_TYPE         HISTIGNORE       PIPESTATUS
            BASHPID                 COMP_KEY          HISTSIZE         POSIXLY_CORRECT
            BASH_ALIASES            COMP_WORDBREAKS   HISTTIMEFORMAT   PPID
            BASH_ARGC               COMP_WORDS        HOSTFILE         PROMPT_COMMAND
            BASH_ARGV               COMPREPLY         HOSTNAME         PROMPT_DIRTRIM
            BASH_ARGV0              COPROC            HOSTTYPE         PS0
            BASH_CMDS               DIRSTACK          IGNOREEOF        PS3
            BASH_COMMAND            EMACS             INPUTRC          PS4
            BASH_COMPAT             ENV               INSIDE_EMACS     PWD
            BASH_ENV                EPOCHREALTIME     LANG             RANDOM
            BASH_EXECUTION_STRING   EPOCHSECONDS      LC_ALL           READLINE_LINE
            BASH_LINENO             EUID              LC_COLLATE       READLINE_POINT
            BASH_LOADABLES_PATH     EXECIGNORE        LC_CTYPE         REPLY
            BASH_REMATCH            FCEDIT            LC_MESSAGES      SECONDS
            BASH_SOURCE             FIGNORE           LC_NUMERIC       SHELL
            BASH_SUBSHELL           FUNCNAME          LC_TIME          SHELLOPTS
            BASH_VERSINFO           FUNCNEST          LINENO           SHLVL
            BASH_VERSION            GLOBIGNORE        LINES            TIMEFORMAT
            BASH_XTRACEFD           GROUPS            MACH_TYPE        TMOUT
            CHILD_MAX               histchars         MAILCHECK        TMPDIR
            COLUMNS                 HISTCMD           MAPFILE          UID
            COMP_CWORD              HISTCONTROL       OLDPWD           # And few more
            COMP_LINE               HISTFILE          OPTERR
        \end{lstlisting}
    \end{center}
    \FrameRemark{Use \,\,\bash|compgen -v|\,\, to get a complete list of all variables available in your session.}
\end{frame}
%~~~~~~~~~~~~~~~~~~~~~~~~~~~~~~~~~~~~~~~~~~~~%
\begin{frame}[fragile]{The Internal Field Separator}{A very important Bash special variable {\tiny~\{More on it later when we discuss word splitting in detail\}}}
    \vspace{-3mm}
    \begin{itemize}
        \item By default, \bash|IFS| is unset and acts as if it were set to \bash|<space><tab><newline>|
        \begin{lstlisting}[style=MyBash, numbers=none, xrightmargin=15mm, belowskip=-6mm, aboveskip=2mm]
            #The following two lines have the same effect
            $ IFS=$' \t\n'
            $ unset IFS
        \end{lstlisting}
        \item It is used to determine what characters to use as word splitting delimiters
        \item Therefore, you can use any of these forms of white space to delimit words
        \item The \bash|IFS| variable comes into play also in special parameter \texttt{\$*} expansion!
    \end{itemize}
    \vspace{-5mm}
    \begin{overlayarea}{\textwidth}{0.35\textheight}
        \begin{varblock}{alerted}[0.8\textwidth]{\Large \textbf{Attention}}<only@1>
            \PQ{\large It is crucial to understand the role of the \bash|IFS| variable!}
        \end{varblock}
        \begin{varblock}{quote}[0.9\textwidth]{\URL[PB]{http://mywiki.wooledge.org/Arguments}{When and why not to use a custom IFS}}[Greg's wiki]<only@2>
            It is important that you understand the danger of changing the way the shell behaves.
            If you modify \textnormal{\bash|IFS|}, word splitting will happen in a non-default manner henceforth.
            Some will recommend you save \textnormal{\bash|IFS|} and reset it to the default later on.
            Others will recommend to unset \textnormal{\bash|IFS|} after you're done with your custom word splitting.\\
            Personally, I prefer to recommend you \textbf{to NOT modify} \textnormal{\bash|IFS|} \textbf{on the script level}.
            \alert{\textbf{Ever!}}\\[-0.7em] ~
        \end{varblock}
    \end{overlayarea}
%     }
\end{frame}

