%-------------------------------%
%  Author: Alessandro Sciarra   %
%    Date: 15 Oct 2020          %
%-------------------------------%

%~~~~~~~~~~~~~~~~~~~~~~~~~~~~~~~~~~~~~~~~~~~~%
\begin{frame}{Error handling in the shell}
    \vspace{-2mm}
    \begin{varblock}{quote}[0.85\textwidth]{Naturally an extremely difficult task}[Greg's Wiki]
        The goal of automatic error detection is a noble one, but it requires the ability to tell when an error actually occurred.
        \PS{In modern high-level languages, most tasks are performed by using the language's builtin commands or features.}
        The language knows whether (for example) you tried to divide by zero, or open a file that you can't open, and so on.
        It can take action based on this knowledge.\\[2mm]
        But in the shell, most of the tasks you actually care about are done by external programs.
        \alert{The shell can't tell whether an external program encountered something that it considers an error} -- and even if it could, it wouldn't know whether the error is an important one, worthy of aborting the entire program, or whether it should carry on.
    \end{varblock}
    \begin{itemize}
        \item If you come from a language like C, you might be used to it.
        \item If you come from a language like Python or C++, do not be frustrated.
    \end{itemize}
\end{frame}
%~~~~~~~~~~~~~~~~~~~~~~~~~~~~~~~~~~~~~~~~~~~~%
\begin{frame}[fragile]{What does Bash do when an error occurs?}{I.e. when a command has a non-zero exit code!}
    \vspace{-6mm}
    \begin{overlayarea}{\textwidth}{0.7\textheight}
        \begin{varblock}{}[0.56\textwidth]{The default behaviour}
            An error is printed and \alert{execution continues}.
        \end{varblock}
        \begin{onlyenv}<1>
            \begin{lstlisting}[style=myBash, numbers=none, aboveskip=5mm]
                #!/bin/bash

                # A test.bash script
                echo "Before error"
                notExistingCommand
                echo "After error"   # <-- This line is exectued!
            \end{lstlisting}
            \begin{lstlisting}[style=myBash, numbers=none, aboveskip=2mm]
                $ ./test.bash
                |+Before error
                ./test.bash: line 5: notExistingCommand: command not found
                After error+|
            \end{lstlisting}
        \end{onlyenv}
        \begin{varblock}{alert}[0.8\textwidth]{Do not forget it!}<2->
            Things that we consider an error are not necessarily an error for bash!
        \end{varblock}
        \begin{onlyenv}<2>
            \begin{lstlisting}[style=myBash, numbers=none, aboveskip=5mm]
                $ invisibleVariable="I am magic"
                $ echo "_${invisibelVariable}_"
                |+__+|
            \end{lstlisting}
        \end{onlyenv}
        \begin{varblock}{example}[0.85\textwidth]{We clearly want more than this!}<3->
            \begin{enumerate}
                \item We often need to react on error and avoid execution to continue.
                \item It might be crucial to prevent undefined variables to be used!
            \end{enumerate}
        \end{varblock}
        \begin{onlyenv}<3>
            \begin{lstlisting}[style=myBash, numbers=none, aboveskip=3mm]
                # About 1: Somewhere in a script
                cd NotExistingFolder
                rm -r *   # AAAAAAARGH!
            \end{lstlisting}
        \end{onlyenv}
        \begin{onlyenv}<4>
            \begin{lstlisting}[style=myBash, numbers=none, aboveskip=3mm]
                #!/bin/bash
                # About 2: And if you call this script without arguments!?
                pathPrefix="$1"
                rm -r "${pathPrefix}/bin"   # AAAAAAARGH!
            \end{lstlisting}
        \end{onlyenv}
    \end{overlayarea}
\end{frame}
%~~~~~~~~~~~~~~~~~~~~~~~~~~~~~~~~~~~~~~~~~~~~%
\begin{frame}[fragile]{Exit codes are the only available ingredient}{\ldots{}the rest is basically up to you!}
    \vspace{-3mm}
    {
        \large
        \begin{varblock}{example}[0.78\textwidth]{Purist approach}
            Check and handle errors properly testing exit codes \textbf{when needed}.
        \end{varblock}
    }
    \begin{itemize}
        \item \PB{\texttt{\$?}} holds the exit code of the last command executed before any given line.
        \item Any executed command will terminate with an exit code that can be inspected.
        \item You can expect external command exit codes to be well documented.
    \end{itemize}
    \begin{onlyenv}<1>
        \begin{lstlisting}[style=myBash, numbers=none, aboveskip=4mm]
            # From 'man grep'
            ...
            |+EXIT STATUS
               Normally the exit status is 0 if a line is selected,
               1 if no lines were selected, and 2 if an error occurred.
               However, if the -q or --quiet or --silent is used and a
               line is selected, the exit status is 0 even if an
               error occurred.+|
            ...
        \end{lstlisting}
    \end{onlyenv}
    \begin{onlyenv}<2>
        \begin{lstlisting}[style=myBash, numbers=none, aboveskip=4mm]
            # In a script at some point
            grep 'Hello' MyBook.txt
            case $? in
                0 ) echo 'Found' ;;
                1 ) echo 'Not found' ;;
                2 ) echo 'Error occurred in grep!'; exit 1 ;;
                * ) echo 'Matrix changed something...'; exit 2 ;;
            esac
        \end{lstlisting}
    \end{onlyenv}
    \begin{onlyenv}<3>
        \begin{lstlisting}[style=myBash, numbers=none, aboveskip=4mm]
            # Coming back to the previous example
            cd NotExistingFolder || exit 1  # Exit on failure!
            rm -r *                         # Now better!
        \end{lstlisting}
        \begin{lstlisting}[style=myBash, numbers=none, aboveskip=2mm, belowskip=-6mm]
            # Alternatively not exiting
            if cd NotExistingFolder; then
                rm -r *
            fi
        \end{lstlisting}
        \centerline{\tiny Of course, this is just an example, you would rather write \;\bash|rm NotExistingFolder/*|\;, right!?}
    \end{onlyenv}
\end{frame}
%~~~~~~~~~~~~~~~~~~~~~~~~~~~~~~~~~~~~~~~~~~~~%
\begin{frame}{Do I have to add \;\PB{\texttt{|| exit 1}}\; everywhere in my scripts?}
    \vspace{-3mm}
    \begin{varblock}{example}[0.75\textwidth]{Purist approach}<2->
        Check and handle errors properly testing exit codes \textbf{when needed}.
    \end{varblock}
    \begin{varblock}{quote}[0.9\textwidth]{}[Greg's Wiki]<3>
        The developers of the original Bourne shell decided that they would create a feature that would allow the shell to check the exit status of every command that it runs, and abort if one of them returns non-zero.
        Thus, \PB{\texttt{set -e}} was born.
        But many commands return non-zero even when there wasn't an error.
        [\ldots]
        So the shell implementers made a bunch of special rules, like ``commands that are part of an if test are immune'', and ``commands in a pipeline, other than the last one, are immune''.
    \end{varblock}
    \begin{varblock}{alert}[0.9\textwidth]{}<3>
        \alert{The \,\PQ{\texttt{set -o errexit}}\, is probably the most controversial feature of bash!}
    \end{varblock}
    \begin{itemize}[<3>]
        \item Many articles are against using it, e.g. \URL[PQ]{http://mywiki.wooledge.org/BashFAQ/105}{Greg's Wiki}.
        \item People encourage its use, e.g. \URL[PS]{https://www.davidpashley.com/articles/writing-robust-shell-scripts/}{David Pashley}.
    \end{itemize}
    \FrameRemark[3]{Both references are definitely worth reading, especially if you decide at the end to use \;\texttt{set -e}\; in your scripts.}
\end{frame}
%~~~~~~~~~~~~~~~~~~~~~~~~~~~~~~~~~~~~~~~~~~~~%
\begin{frame}{Enabling the \;\texttt{-e}\; shell option: Let's read the manual}
    \vspace{-3mm}
    \begin{itemize}[<1>]
        \item Exit immediately if a pipeline, which may consist of a single simple command, a list, or a compound command returns a non-zero status.
        \item The shell does not exit if the command that fails
              \begin{itemize}
                  \item is part of the command list immediately following a \bash|while| or \bash|until| keyword,
                  \item is part of the test in an \bash|if| statement,
                  \item is part of any command executed in a \PB{\texttt{\&\&}} or \PB{\texttt{||}} list except the command following the final \PB{\texttt{\&\&}} or \PB{\texttt{||}},
                  \item is any command in a pipeline but the last,
              \end{itemize}
              or if the command’s return status is being inverted with \PB{\texttt{!}}.
        \item If a compound command other than a subshell returns a non-zero status because a command failed while \texttt{-e} was being ignored, the shell does not exit.
        \item A trap on \texttt{ERR}, if set, is executed before the shell exits.
        \item This option applies to the shell environment and each subshell environment separately, and may cause subshells to exit before executing all the commands in the subshell.
    \end{itemize}
    \begin{tikzpicture}[remember picture, overlay]
        \node[visible on=<3>] (fig) at ($(current page.center)-(0,8mm)$) {\includegraphics[width=0.7\textwidth]{OMG}};
        \node[above = 3mm of fig, visible on=<2->, font=\large, text=PB] {Your reaction might be like:};
    \end{tikzpicture}
    \FrameRemark[1]{And few more rules are there about when a compound command or shell function executes (or sets \texttt{-e}) in a context where \texttt{-e} is being ignored...}
\end{frame}
%~~~~~~~~~~~~~~~~~~~~~~~~~~~~~~~~~~~~~~~~~~~~%
\begin{frame}[fragile]{But what does it mean, in practice?}
    \vspace{-4mm}
    \begin{overlayarea}{\textwidth}{0.7\textheight}
        \begin{itemize}
            \setlength{\itemsep}{0pt}
            \item \alert<2>{You need to \textbf{deeply} understand this shell option}
            \item \alert<3-4>{Your way of programming has to adapt to all corner cases}
            \item \alert<5>{Checking and storing error codes via \PB{\texttt{\$?}} is not possible as before}
        \end{itemize}
        \vspace{2mm}
        \begin{onlyenv}<2>
            \begin{lstlisting}[style=myBash]
                #!/usr/bin/env bash
                set -e
                i=0
                ((i++))          # ((...)) is a compound command!
                echo "i is $i"   # This is NOT printed
            \end{lstlisting}
            \centerline{\tiny The postfix increment operator returns the old value of the variable.}
            \begin{lstlisting}[style=myBash, aboveskip=2mm]
                #!/usr/bin/env bash
                set -e
                i=1
                ((i++))          # ((...)) is a compound command!
                echo "i is $i"   # This is here printed
            \end{lstlisting}
        \end{onlyenv}
        \begin{onlyenv}<3>
            \begin{lstlisting}[style=myBash]
                #!/usr/bin/env bash
                set -e
                i=0
                ((i++)) || true
                echo "i is $i"   # This is ALWASY printed

                i=$((i++))       # Simple command without command subst.
                echo "i is $i"   # This is ALWASY printed

                i=$(false)       # Simple command with failing $(...)
                echo "i is $i"   # This is NOT printed
            \end{lstlisting}
            \footnotesize
            \begin{varblock}{quote}[0.9\textwidth]{Simple command expansion}[Bash manual]
                \scriptsize[\ldots]
                If one of the expansions contained a command substitution, the exit status of the command is the exit status of the last command substitution performed.
                If there were no command substitutions, the command exits with a status of zero.
            \end{varblock}
        \end{onlyenv}
        \begin{onlyenv}<4>
            \begin{lstlisting}[style=myBash]
                #!/usr/bin/env bash
                set -e

                function Failure() { false; }
                function FailureMsg() {
                    false
                    echo "Failed!" >&2
                }

                ( false )          # Subshell -> new environment, -e unset

                echo 'Begin'
                aVar=$(FailureMsg) # No failure because of echo in function
                echo 'Middle'
                aVar=$(Failure)    # Function fails
                echo "End"         # This is NOT printed
            \end{lstlisting}
        \end{onlyenv}
        \begin{onlyenv}<5>
            \begin{lstlisting}[style=myBash]
                #!/usr/bin/env bash
                set -e
                function Simulation(){ return 1; }

                if Simulation; then
                    echo 'Success!'
                else
                    echo 'Failure!'
                fi

                Simulation               # Here the script exits!

                if [[ $? -eq 0 ]]; then  # This if-clause is skipped!
                    echo 'Success!'
                else
                    echo 'Failure!'
                fi
            \end{lstlisting}
        \end{onlyenv}
        \begin{onlyenv}<6>
            \large
            \begin{varblock}{example}[0.5\textwidth]{Spend some time on it}
                This afternoon you will deal again in detail with some corner cases!
            \end{varblock}
            \begin{varblock}{}[0.65\textwidth]{And if I do not want to use it?}
                \large \PB{Do I have other alternatives beyond the purist approach and enabling \texttt{errexit}?}
            \end{varblock}
            \vspace*{-1mm}
            \begin{varblock}{alert}[0.25\textwidth]{}
                \PQ{Not really\ldots}
            \end{varblock}
        \end{onlyenv}
    \end{overlayarea}
\end{frame}
%~~~~~~~~~~~~~~~~~~~~~~~~~~~~~~~~~~~~~~~~~~~~%
\begin{frame}[fragile]{Consider to enable \;\PB{\texttt{inherit\_errexit}}\; shell option}{It was introduced in Bash v4.4}
    \vspace{-3mm}
    \begin{varblock}{quote}[0.95\textwidth]{}[Bash manual]
        If set, command substitution inherits the value of the errexit option, instead of unsetting it in the subshell environment.
        This option is enabled when POSIX mode is enabled.
    \end{varblock}
    \begin{lstlisting}[style=myBash]
        #!/usr/bin/env bash
        set -e
        shopt -s inherit_errexit

        function Failure() { false; }
        function FailureMsg() {
            false
            echo "Failed!" >&2
        }

        echo 'Begin'
        aVar=$(FailureMsg) # Now the script exits here before echo
        echo 'Middle'
        aVar=$(Failure)    # Function fails
        echo "End"         # This is NOT printed
    \end{lstlisting}
\end{frame}
%~~~~~~~~~~~~~~~~~~~~~~~~~~~~~~~~~~~~~~~~~~~~%
\begin{frame}[fragile]{Trap on ERR and the \PB{\texttt{errtrace}} shell option}
    \vspace{-2mm}
    \begin{onlyenv}<1>
        \begin{varblock}{alert}[0.85\textwidth]{The ERR signal specification}
            The trap on \texttt{ERR} is executed exactly when the bash would exit with the \texttt{errexit} option enabled.
            Said differently, the same conditions obeyed by the \texttt{errexit} option apply to traps on \texttt{ERR}.
        \end{varblock}
        \begin{varblock}{quote}[0.9\textwidth]{The \;\texttt{errtrace}\; shell option}[Bash manual]
            If set, any trap on \texttt{ERR} is inherited by shell functions, command substitutions, and commands executed in a subshell environment.
            The \texttt{ERR} trap is normally not inherited in such cases.
        \end{varblock}
        \begin{varblock}{quote}[0.8\textwidth]{}[St\'ephane Chazelas]
            \PB{\texttt{set -E}} in combination with \texttt{ERR} traps is a bit like an improved version of \PB{\texttt{set -e}} in that it allows you to define your own error handling.\\[-0.5em] ~
        \end{varblock}
    \end{onlyenv}
    \begin{onlyenv}<2>
        \begin{lstlisting}[style=myBash, aboveskip=2mm]
            #!/usr/bin/env bash
            set -E
            trap '[ "$?" -ne 77 ] || exit 77' ERR
            (
              echo 'Here'
              (
                echo 'There'
                ( exit 12 )      # not 77, exit only this subshell
                echo 'Somewhere'
                exit 77          # exit all subshells
              )
              echo 'Not here'
            )
            echo 'Not here either'
        \end{lstlisting}
        \begin{lstlisting}[style=myBash, aboveskip=2mm, numbers=none]
            $ ./previous-script.bash
            |+Here
            There
            Somewhere+|
        \end{lstlisting}
    \end{onlyenv}
    \FrameRemark{Use the \;\PB{\texttt{functrace}}\; shell option to inherit any trap on DEBUG and RETURN in shell functions, command substitutions, and subshells.}
\end{frame}
%~~~~~~~~~~~~~~~~~~~~~~~~~~~~~~~~~~~~~~~~~~~~%
\begin{frame}[fragile]{Enabling the \;\texttt{-u}\; shell option}
    \vspace{-3mm}
    \begin{varblock}{quote}[0.9\textwidth]{\texttt{set -o nounset}}[Bash manual]
        Treat unset variables and parameters other than the special parameters \PB{\texttt{@}} or \PB{\texttt{*}} as an error when performing parameter expansion.
        An error message will be written to the standard error, and a non-interactive shell will exit.
    \end{varblock}
    \begin{lstlisting}[style=myBash, numbers=none, belowskip=-4mm]
        $ echo "_${unsetVariable}_"
        |+__+|
        $ set -u; echo "_${unsetVariable}_"   # A script would
        |+bash: unsetVariable: unbound variable+| # exit here!
        $ echo "_${unsetVariable-}_"; set +u
        |+__+|
    \end{lstlisting}
    \begin{varblock}{alert}[0.96\textwidth]{Few (minor) drawbacks exist}
        \begin{enumerate}
            \small
            \item Turning \PB{\texttt{nounset}} on in a correct existing script might still break it.
            \item If some code is not executed, \PB{\texttt{set -u}} will have no effect on it either.
            \item Versions of bash \alert{prior to v4.4} \URL[PP]{https://gist.github.com/dimo414/2fb052d230654cc0c25e9e41a9651ebe}{handle the \texttt{[@]}-expansion of empty arrays differently}.
        \end{enumerate}
    \end{varblock}
    \FrameRemark{On Greg's Wiki you can \URL[PB]{http://mywiki.wooledge.org/BashFAQ/112}{read more about \texttt{set -u}}.}
\end{frame}
%~~~~~~~~~~~~~~~~~~~~~~~~~~~~~~~~~~~~~~~~~~~~%
\begin{frame}{Is the glass half full or half empty?}
    \vspace{-1mm}
    \begin{varblock}{quote}[0.8\textwidth]{A reasonable person might say:}
        I don't want to rely on an unreliable feature.
        I understood the behaviour of the shell when an error occurs and I can do error handling myself.
    \end{varblock}
    \begin{varblock}{quote}[0.8\textwidth]{Another reasonable person might say:}<2->
        Yes, \texttt{errexit} is not a perfect feature of bash, but it has useful semantics, so I don't want to exclude it from the toolbox.
        Still I have to work a bit to get acquainted with all corner cases, but I think it is worth the effort.
        Especially because I do not want to be bitten by mistakes in doing error handling by hand.
        Hopefully, future versions of bash will improve this aspect.
    \end{varblock}
    \begin{varblock}{example}[0.94\textwidth]{Choose your way}<3>
        Probably consistency within a script is the most important aspect.
        My recommendation is to \PS{explore this area of bash in all ways before taking your decision}.
    \end{varblock}
    \FrameRemark{\PB{\texttt{set -o pipefail}} is also another interesting feature \URL[PB]{http://mywiki.wooledge.org/BashPitfalls\#pipefail}{with minor drawbacks}.}
\end{frame}
%~~~~~~~~~~~~~~~~~~~~~~~~~~~~~~~~~~~~~~~~~~~~%
