%-------------------------------%
%  Author: Alessandro Sciarra   %
%    Date: 6 Sep 2019           %
%-------------------------------%

%~~~~~~~~~~~~~~~~~~~~~~~~~~~~~~~~~~~~~~~~~~~~%
\begin{frame}[fragile]{Motivation}
    \vspace{-2mm}
    Consider the following script (\alert{not nice code}):
    \begin{lstlisting}[style=MyBash, aboveskip=3mm, belowskip=-5mm]
        #!/bin/bash
        
        function CreateAuxiliaryFiles(){
            # ...
        }
        function CleanAuxiliaryFiles(){
            # ...
        }
        CreateAuxiliaryFiles
        gnuplot "${temporaryFileToPlot}"
        if [[ "${savePlot}" = 'TRUE' ]]; then
            pdflatex "${outputFilename}.tex"
        fi
        CleanAuxiliaryFiles
        exit 0
    \end{lstlisting}
    \begin{varblock}{alert}[\textwidth]{}
        \Large \alert{What happens if the script is terminated by the user, e.g. via CTRL-C?}
    \end{varblock}
\end{frame}
%~~~~~~~~~~~~~~~~~~~~~~~~~~~~~~~~~~~~~~~~~~~~%
\begin{frame}{Signal handlers or traps}
    \begin{onlyenv}<1-2>
        \begin{itemize}
            \item As you know you can send signals to processes via the \bash|kill| builtin
            \item However, in shell scripts, \PP{\textbf{it is possible to associate a behaviour upon receiving a signal!}}
            \item If you think about it, it is a very powerful technique
            \item In order to perform an action when a signal is received, a trap has to be set up
        \end{itemize}
        \smallskip
        \begin{center}
            \begin{tikzpicture}
                \begin{scope}[every node/.style={thick, ellipse, minimum width=2cm}]
                    \node[draw=PB, fill=PP!20] (a) {Process A};
                    \node[draw=PQ, fill=PT!20, below right = of a] (b) {Process B};
                \end{scope}
                \begin{scope}[scope on=<2>]
                    \path[to] (a) edge[out=0, in=90] node[midway, right=2mm, font=\small, text=PS] {Signal} (b);
                    \path[to] (b) edge[out=270, in=180] node[pos=1, right=1mm] {Do the action associated to the signal} ++(1cm, -15mm);
                \end{scope}
            \end{tikzpicture}
        \end{center}
    \end{onlyenv}
    \begin{onlyenv}<3>
        \medskip
        \colorbox{background-color}{\bash|trap 'some code' signal(s)|}\\[0.3em]
        \begingroup\leftskip1em
            Using this form, a signal handler is set up for each signal in the list\\
            When one of these signals is received, the commands in the first argument will be executed\\[0.4em]
        \endgroup
        \colorbox{background-color}{\bash|trap '' signal(s)|}\\[0.3em]
        \begingroup\leftskip1em
            Using this form, each signal in the list will be ignored. \alert{Most scripts should not do this!}\\[0.4em]
        \endgroup
        \colorbox{background-color}{\bash|trap - signal(s)|}\\[0.3em]
        \begingroup\leftskip1em
            Using this form, each signal in the list will be restored to its default behaviour\\[0.4em]
        \endgroup
        \colorbox{background-color}{\bash|trap signal(s)|}\\[0.3em]
        \begingroup\leftskip1em
            Using this form, the one signal listed will be restored to its default behaviour\\[0.4em]
        \endgroup
        \colorbox{background-color}{\bash|trap|}\\[0.3em]
        \begingroup\leftskip1em
            With no arguments, print a list of signal handlers\\
        \endgroup
    \end{onlyenv}
\end{frame}
%~~~~~~~~~~~~~~~~~~~~~~~~~~~~~~~~~~~~~~~~~~~~%
\begin{frame}{Common used signals}
    \begin{description}[TERMX]
        \item[\textbf{HUP}] Hang Up. The controlling terminal has gone away.
        \item[\textbf{INT}] Interrupt. The user has pressed the interrupt key (usually \textbf{Ctrl-C} or \textbf{DEL}).
        \item[\textbf{QUIT}] Quit. The user has pressed the quit key (usually \textbf{Ctrl-\textbackslash}). Exit and dump core.
        \item[\textbf{KILL}] Kill. \alert{This signal cannot be caught or ignored.} Unconditionally fatal. \textbf{\textbf{No cleanup possible.}}
        \item[\textbf{TERM}] Terminate. This is the default signal sent by the \bash|kill| builtin.
        \item[\textbf{EXIT}] Not really a signal. In a Bash script (non-interactive), an EXIT trap is run on any exit, signalled or not. In other POSIX shells only when the shell process exits.
    \end{description}
    \begin{varblock}{}[0.9\textwidth]{Remember}
        If you are asking a program to terminate, you should always use SIGTERM.% (simply \bash|kill process_ID|).
        This will give the program a chance to catch the signal and clean up.
        If you use SIGKILL, the program cannot clean up, and \textbf{may leave files in a corrupted state}.
    \end{varblock}
\end{frame}
%~~~~~~~~~~~~~~~~~~~~~~~~~~~~~~~~~~~~~~~~~~~~%
\begin{frame}[fragile]{A simple example}
    \vspace{-4mm}
    \begin{varblock}{quote}[\textwidth]{Common use case}
        \normalsize\textnormal{Traps can be set up to intercept a fatal signal, perform cleanup, and then exit gracefully}
    \end{varblock}
    \begin{lstlisting}[style=MyBash, emph={[7]temporaryFile}, belowskip=-5mm]
        #!/bin/bash
        
        temporaryFile=$(mktemp) || exit
        trap 'rm -f "${temporaryFile}"' EXIT
    \end{lstlisting}
    Use a function whenever you need to achieve a more complex task
    \begin{lstlisting}[style=MyBash, emph={[7]temporaryFile}, aboveskip=2mm, belowskip=-5mm, xleftmargin=2mm, xrightmargin=3mm]
        #!/bin/bash
        
        function CleanAuxiliaryFiles(){ # If you trap INT (CTRL-C)
            # ...                       # do not forget to exit unless
        }                               # you have reasons not to do so
        trap 'CleanAuxiliaryFiles' EXIT
    \end{lstlisting}
    \small
    \begin{varblock}{}[\textwidth]{Only in Bash!}
        The special name EXIT is preferred for any signal handler that simply wants to clean up upon exiting.
        So to clean up, just trap EXIT and call a cleanup function from there. \textbf{Don't trap a bunch of signals.}
    \end{varblock}
\end{frame}
%~~~~~~~~~~~~~~~~~~~~~~~~~~~~~~~~~~~~~~~~~~~~%
\begin{frame}[fragile]{When is the signal exactly handled?}
    \vspace{-5mm}
    \begin{varblock}{alert}[0.9\textwidth]{A subtle but important feature}
        When Bash is executing an external command in the foreground, it does not handle any signals received until the foreground process terminates
    \end{varblock}
    \begin{overlayarea}{\textwidth}{0.33\textheight}
        \begin{onlyenv}<1>
            \begin{lstlisting}[style=MyBash]
                #!/bin/bash
                
                echo "My PID is $$ and my PGID is $(ps -h -p $$ -o pgid)"
                trap 'echo "Doing some cleaning before exiting!"' EXIT
                echo "Wait 1h"
                sleep 3600
            \end{lstlisting}
        \end{onlyenv}
        \begin{onlyenv}<2>
            \begin{lstlisting}[style=MyBash]
                #!/bin/bash
                
                echo "My PID is $$ and my PGID is $(ps -h -p $$ -o pgid)"
                trap 'echo "Doing some cleaning before exiting!"' EXIT
                echo "Wait 1h"; sleep 3600 & wait $!
                # Note that sleep 3600 will not be killed and will
                # continue to run when you send a INT signal!
            \end{lstlisting}
        \end{onlyenv}
        \begin{onlyenv}<3>
            \begin{lstlisting}[style=MyBash]
                #!/bin/bash
                
                echo "My PID is $$ and my PGID is $(ps -h -p $$ -o pgid)"
                unset -v pid
                trap 'echo "Cleaning!"; [[ $pid ]] && kill "${pid}"' EXIT
                echo "Wait 1h"
                sleep 3600 & pid=$!; wait
            \end{lstlisting}
        \end{onlyenv}
    \end{overlayarea}
    \begin{itemize}
        \item If you kill the script using $\;$\bash|kill -s INT|$\;$ from another terminal (not with CTRL-C), bash will wait for \bash|sleep| to exit before calling the trap\\
              $\;\to\;$\alert{\textbf{That's probably not what you expect!}}
        \item<2-> A work-around is to use a \PB{\textbf{builtin}} that will be interrupted, such as \bash|wait|
        \item<2-> \PP{Any bash internal command will be interrupted by a (non-ignored) incoming signal!}
        \item<3-> If you want the background job to be killed when the script is killed, add that to the trap!
    \end{itemize}
\end{frame}
%~~~~~~~~~~~~~~~~~~~~~~~~~~~~~~~~~~~~~~~~~~~~%
\begin{frame}[fragile]{What is indeed CTRL-C doing?}
    \vspace{-3mm}
    Let us consider again the previous example
    \begin{lstlisting}[style=MyBash, numbers=none, aboveskip=2mm]
        #!/bin/bash

        echo "My PID is $$ and my PGID is $(ps -h -p $$ -o pgid)"
        trap 'echo "Doing some cleaning before exiting!"' EXIT
        echo "Wait 1h"; sleep 3600
    \end{lstlisting}
    \begin{overlayarea}{\textwidth}{0.5\textheight}
        \begin{itemize}
            \item<1-> As said, sending to this script the INT signal from another terminal does not terminate it
            \item<1-> However, CTRL-C does terminate it\ldots\ but why?
            \item<2-> Because \PP{\textbf{processes are organised in groups}}:
                      \begin{itemize}
                          \item The leader process (i.e. the process that created the group)
                          \item Any other process started by the leader process
                      \end{itemize}
            \item<2-> \alert{\textbf{CTRL-C sends the INT signal to the entire process group}}
            \item<2-> Terminals keep track of the foreground process group: When receiving a CTRL-C, they send the SIGINT to the entire foreground group!
        \end{itemize}
        \begin{onlyenv}<3>
            \begin{lstlisting}[style=MyBash, numbers=none, xleftmargin=2mm, xrightmargin=2mm]
                kill -s INT -123 # will kill the process group with the ID 123
            \end{lstlisting}
        \end{onlyenv}
    \end{overlayarea}
    \FrameRemark{Note that you can't rely on the process group ID of a script to be the same as \texttt{\$\$}, as that depends greatly on how the script was started.}
\end{frame}
%~~~~~~~~~~~~~~~~~~~~~~~~~~~~~~~~~~~~~~~~~~~~%
\begin{frame}[fragile]{Trapping SIGINT and SIGQUIT: Be careful!}
    \vspace{-3mm}
    \begin{overlayarea}{\textwidth}{0.7\textheight}
        \begin{itemize}
            \item Bash is among a few shells that implement a \textbf{wait and cooperative exit} approach at handling SIGINT/SIGQUIT delivery
            \item When interpreting a script, upon receiving a SIGINT,
            \begin{enumerate}
                \item \textbf{it doesn't exit straight away}, instead
                \item \PP{it waits for the currently running command to return} and
                \item \alert{it only exits -- \textbf{by killing itself with SIGINT} -- if that command was also killed by that SIGINT}
            \end{enumerate}
            \item<only@1> The idea is that, if your script calls \bash|vi| for instance, and you press CTRL-C within \bash|vi| to cancel an action, that should not be considered as a request to abort the script
        \end{itemize}
        \begin{varblock}{alert}[0.9\textwidth]{What does it mean?}<only@1>
            Imagine you're writing a script and that script exits normally upon receiving SIGINT.
            That means that if that script is invoked from another bash script, \\\alert{\textbf{CTRL-C will no longer interrupt that other script!}}
        \end{varblock}
        \begin{onlyenv}<2>
            \begin{varblock}{}[0.95\textwidth]{}
                \small The \bash|ping| command returns with 0 when host is reachable (the ping has been answered) and non-zero otherwise when interrupted (which is the only way for ping to return in that case).
            \end{varblock}
            \begin{lstlisting}[style=MyBash, belowskip=-5mm]
                #!/bin/bash
                for index in {1..254}; do
                    ping -c 2 "192.168.1.${index}"
                done
            \end{lstlisting}
            \begin{varblock}{}[0.95\textwidth]{}
                \small CTRL-C will send a INT signal to the script, but since \bash|ping| just returns 0 or 1 and is not killed by SIGINT, only the \alert{\textbf{current}} \bash|ping| will terminate and the \bash|for| loop will continue!
            \end{varblock}
        \end{onlyenv}
        \begin{onlyenv}<3>
            \begin{varblock}{}[1.03\textwidth]{}
                Commands that don't have a SIGINT handler (like \bash|sleep|) or \textbf{do the right thing of killing themselves with SIGINT upon receiving SIGINT} (like \bash|bash| itself does) don't exhibit the problem!
            \end{varblock}
            \begin{lstlisting}[style=MyBash]
                #!/bin/bash
                index=1
                while [[ "${index}" -le 100 ]]; do
                  printf "%d " "$((index++))"
                  sleep 1
                done; echo  # This script terminates correctly via CTRL-C
            \end{lstlisting}
        \end{onlyenv}
        \begin{onlyenv}<4>
            \medskip
            \begin{varblock}{alert}[\textwidth]{Take-home lesson}
                If you choose to set up a handler for SIGINT (rather than using the EXIT trap), you should be aware that \alert{\textbf{a process that exits in response to SIGINT should kill itself with SIGINT rather than simply exiting}}, to avoid causing problems for its caller.
                The same goes for SIGQUIT.
            \end{varblock}
            \begin{lstlisting}[style=MyBash, numbers=none]
                trap 'DoYourStuffHere; trap - INT; kill -s INT "$$"' INT
            \end{lstlisting}
        \end{onlyenv}
    \end{overlayarea}
\end{frame}











