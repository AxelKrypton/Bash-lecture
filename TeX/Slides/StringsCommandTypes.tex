%-------------------------------%
%  Author: Alessandro Sciarra   %
%    Date: 23 Sep 2020          %
%-------------------------------%

%~~~~~~~~~~~~~~~~~~~~~~~~~~~~~~~~~~~~~~~~~~~~%
\begin{frame}[noframenumbering, plain]{After all it is simple...}
    \centering
    \includegraphics[width=0.9\textwidth]{ToyStory}
\end{frame}
%~~~~~~~~~~~~~~~~~~~~~~~~~~~~~~~~~~~~~~~~~~~~%
\begin{frame}{Strings, strings everywhere}
    \vspace{-3mm}
    The term string refers to a sequence of characters which is treated as a single unit:
    \begin{itemize}
        \item The command's name is a string
        \item Each argument of a command is a string
        \item Variable names are strings
        \item The contents of variables are strings as well
        \item A filename is a string
        \item Most files contain strings
    \end{itemize}
    \begin{varblock}{alerted}[0.7\textwidth]{Strings do not have any intrinsic meaning}
        Their meaning is defined by how and where they are used. 
    \end{varblock}
    \begin{varblock}{}[0.85\textwidth]{We have \textbf{all} the responsibility}
        We need to be sure everything that needs to be separated is separated properly, and everything that needs to stay together stays together properly!
    \end{varblock}
    \FrameRemark{I will loosely use the term \textbf{string} throughout the lecture, mostly referring to a portion of text contained in a variable}
\end{frame}
%~~~~~~~~~~~~~~~~~~~~~~~~~~~~~~~~~~~~~~~~~~~~%
\begin{frame}<1>[fragile, label=CommandsTypes]{Types of commands}
    \only<1>{\framesubtitle{There are basically 5 different classes of commands}}
    \only<2->{\framesubtitle{}}
    \begin{onlyenv}<1>
        \begin{center}
            \begin{tikzpicture}
                \node[minimum size=5cm, circle, draw, thin] (O) at (0,0) {};
                \foreach \l [count=\li, evaluate=\li as \angle using int(\li*72)] in {Aliases, Functions, Builtins, Keywords, Executables}{
                    \node[cloud, draw, text=PB, fill=PP!20, aspect=2] at (O.\angle) {\l};
                }
            \end{tikzpicture}
        \end{center}
    \end{onlyenv}
    \vspace{-6mm}
    \begin{onlyenv}<2->
        \setlength{\columnsep}{-5mm}
        \begin{multicols}{5}
            \begin{enumerate}
                \item \tc<2>{PP}{Aliases}
                \item \tc<3>{PP}{Functions}
                \item \tc<4-6>{PP}{Builtins}
                \item \tc<7>{PP}{Keywords}
                \item \tc<8>{PP}{Executables}
            \end{enumerate}
        \end{multicols}
    \end{onlyenv}
    \vspace{-3mm}
    \begin{overlayarea}{\textwidth}{0.7\textheight}
        \begin{onlyenv}<2>
            \begin{itemize}
                \item An alias is a rudimentary way of shortening a command
                \item \PP{It is a word that is mapped to a certain string}
                \item They are only used in \textbf{interactive} shells and \textbf{not} in scripts
                \item They are limited in power; the replacement only happens in the first word
                \item An alias should effectively not do more than change the default options of a command
                \item For more complex tasks and more flexibility, use a function
            \end{itemize}
            \begin{lstlisting}[style=MyBash]
                $ help alias
                |+alias: alias [-p] [name[=value] ... ]+|
                |+    Define or display aliases.       +|

                # [More information]

                $ alias ls='ls --color=auto'
                $ ls  # The command executed is, instead:  ls --color=auto
            \end{lstlisting}
            \FrameRemark{Bash stores the values of aliases in an array called \bash|BASH_ALIASES|}
        \end{onlyenv}
        \begin{onlyenv}<3>
            {\Large
            \begin{varblock}{alerted}[0.7\textwidth]{Functions are tricky!}
                They will be covered in depth later in the course
            \end{varblock}}
            \bigskip
            For the moment, a few hints:
            \begin{itemize}
                \item Functions in Bash are more powerful than aliases and the are often the way to go
                \item Unlike aliases, they can be used in scripts, i.e. in non-interactive mode
                \item A function contains shell commands, and acts very much like a small script
                \item They can even take arguments and create local variables
                \item When a function is called, the commands in it are executed
            \end{itemize}
        \end{onlyenv}
        \begin{onlyenv}<4-6>
            \begin{itemize}
                \item Builtins are basic commands that Bash has built into it
                \item They can be thought as functions that are already provided
                \item We will learn about (or at least mention) \tc<5>{PP}{most of them}
                \item Keywords which are provided as builtins are nevertheless highlighted as \tc<6>{keywords-color}{keywords}
            \end{itemize}
            \begin{onlyenv}<4>
                \begin{lstlisting}[style=MyBash, numbers=none, keywordstyle=\color{builtins-color},]
                   :         complete   export    let        shift     umask  
                   .         compgen    false     local      shopt     unalias
                   [         continue   fc        logout     source    unset  
                   alias     declare    fg        popd       suspend   until  
                   bg        dirs       getopts   printf     test      wait   
                   bind      disown     hash      pushd      times     while  
                   break     echo       help      pwd        trap      
                   builtin   enable     history   read       true      
                   case      eval       if        readonly   type      
                   cd        exec       jobs      return     typeset   
                   command   exit       kill      set        ulimit    
                \end{lstlisting}
            \end{onlyenv}
            \begin{onlyenv}<5>
                \begin{lstlisting}[style=MyBash, numbers=none, deleteemph={[5]:,.,[,alias, bg, break, case, cd, continue, declare, disown, echo, eval, exec, exit, export, false, fg, help, if, jobs, kill, let, local,printf, pwd, read, readonly, return, set, shift, shopt, source, test, trap, true, unalias, unset, until, wait, while}, emph={[7]:,.,[,alias, bg, break, case, cd, continue, declare, disown, echo, eval, exec, exit, export,false, fg, help, if, jobs, kill, let, local,printf, pwd, read, readonly, return, set, shift, shopt, source, test, trap, true, unalias, unset, until, wait, while}, emphstyle={[7]\color{PP}}]
                   :         complete   export    let        shift     umask  
                   .         compgen    false     local      shopt     unalias
                   [         continue   fc        logout     source    unset  
                   alias     declare    fg        popd       suspend   until  
                   bg        dirs       getopts   printf     test      wait   
                   bind      disown     hash      pushd      times     while  
                   break     echo       help      pwd        trap      
                   builtin   enable     history   read       true      
                   case      eval       if        readonly   type      
                   cd        exec       jobs      return     typeset   
                   command   exit       kill      set        ulimit    
                \end{lstlisting}
            \end{onlyenv}
            \begin{onlyenv}<6>
                \begin{lstlisting}[style=MyBash, numbers=none]
                   :         complete   export    let        shift     umask  
                   .         compgen    false     local      shopt     unalias
                   [         continue   fc        logout     source    unset  
                   alias     declare    fg        popd       suspend   until  
                   bg        dirs       getopts   printf     test      wait   
                   bind      disown     hash      pushd      times     while  
                   break     echo       help      pwd        trap      
                   builtin   enable     history   read       true      
                   case      eval       if        readonly   type      
                   cd        exec       jobs      return     typeset   
                   command   exit       kill      set        ulimit    
                \end{lstlisting}
            \end{onlyenv}
        \end{onlyenv}
        \begin{onlyenv}<7>
            \begin{varblock}{alerted}[0.75\textwidth]{}
                Keywords are like builtins, but \alert{special parsing rules apply to them}
            \end{varblock}
            \begin{itemize}
                \item Flow control constructs is achieved thanks to keywords
                \item We will explore all of them in detail \Remark[2mm]{except \bash|coproc|}
            \end{itemize}
            \begin{lstlisting}[style=MyBash, numbers=none]
                if     elif   esac     while   done      time   !    coproc
                then   fi     for      until   in        {      [[
                else   case   select   do      function  }      ]]
            \end{lstlisting}
            \bigskip
            For example:
            \medskip
            \begin{lstlisting}[style=MyBash]
                $ [ a < b ]   # Oops, < means here input redirection!
                |+-bash: b: No such file or directory+|
                $ [[ a < b ]] # The meaning of < changes within [[ ]]
            \end{lstlisting}
        \end{onlyenv}
        \begin{onlyenv}<8>
            \begin{itemize}
                \item Executables are the last type of command that Bash can execute
                \item They are also known as external commands or applications
                \item They are typically invoked by their name, on the constraint that Bash is able to find them
                \item When a command is specified without a file path and it is \alert{not an alias, a function, a builtin or a keyword}, Bash searches through the directories contained in the variable \bash|PATH|
                \item The search is done from left to right and the first executable found is run
            \end{itemize}
            \begin{lstlisting}[style=MyBash]
                $ ls /home/sciarra/.local/bin
                |+...   g++   ...+| # Version 8.3.0
                $ ls /usr/bin
                |+...   g++   ...+| # Version 5.4.0
                $ echo ${PATH}
                |+/home/sciarra/.local/bin:/usr/bin:/bin+|
                $ g++ -dumpversion
                |+8.3.0+|
            \end{lstlisting}
        \end{onlyenv}
    \end{overlayarea}
\end{frame}
%~~~~~~~~~~~~~~~~~~~~~~~~~~~~~~~~~~~~~~~~~~~~%
\againframe<2>{CommandsTypes}
%~~~~~~~~~~~~~~~~~~~~~~~~~~~~~~~~~~~~~~~~~~~~%
\againframe<3>{CommandsTypes}
%~~~~~~~~~~~~~~~~~~~~~~~~~~~~~~~~~~~~~~~~~~~~%
\againframe<4-6>{CommandsTypes}
%~~~~~~~~~~~~~~~~~~~~~~~~~~~~~~~~~~~~~~~~~~~~%
\againframe<7>{CommandsTypes}
%~~~~~~~~~~~~~~~~~~~~~~~~~~~~~~~~~~~~~~~~~~~~%
\againframe<8>{CommandsTypes}
%~~~~~~~~~~~~~~~~~~~~~~~~~~~~~~~~~~~~~~~~~~~~%
\begin{frame}[fragile]{Bash scripts}
    \vspace{-6mm}
    \begin{itemize}
        \item A script is a sequence of Bash commands in a file
        \item The commands are read and processed \PP{in order}
        \item A new command is processed after the previous has \PP{ended} \Remark{unless differently required}
        \item The first line of a script should be reserved for an \textbf{interpret directive} also called \alert{hashbang} or \alert{shebang}.
              This is used when the kernel executes a non-binary file.
              Use one of the two following alternatives: \\
              \begin{center}
                  \begin{tikzpicture}
                      \node[fill=background-color, inner sep=1mm, font=\scriptsize] (shebangA) {\bash|#!/bin/bash|};
                      \node[fill=background-color, inner sep=1mm, font=\scriptsize, right = 3cm of shebangA] (shebangB) {\bash|#!/usr/bin/env bash|};
                      \draw[to, shorter={2mm}{2mm}] (shebangA) -- node[midway, above=-1pt, font=\tiny] {or, preferably,} (shebangB);
                  \end{tikzpicture}
                  \par{\ssmall The right directive has the benefit of looking for whatever the default version of the program is in your current \textbf{env}ironment (i.e. in \bash|PATH|).}
              \end{center}
        \item<only@1> Please, \alert{do not use} \,\raisebox{0.1em}{\fbox{\ssmall\bash|#!/bin/sh|}}\, as shebang, even if you might see it around on the web!
        \item<only@1> Avoid giving your scripts a \textbf{.sh} extension. Do not use one or use \alert{\textbf{.bash}}!
        \item<only@2> Execute the script in either of the following ways:
    \end{itemize}
    \begin{varblock}{alerted}[0.75\textwidth]{\textbf{Sh is NOT Bash!}}<only@1>
        \textbf{Bash} itself is a \textbf{sh-compatible} shell however, the opposite is not true!
    \end{varblock}
    \begin{onlyenv}<2>
        \begin{lstlisting}[style=MyBash]
            $ bash myscript   #The shebang is treated as a comment!
        \end{lstlisting}
        \centerline{or}
        \begin{lstlisting}[style=MyBash]
            $ chmod +x myscript  # Make the file executable
            $ ./myscript         # Execute it, the shebang is used!
        \end{lstlisting}
    \end{onlyenv}
\end{frame}
%~~~~~~~~~~~~~~~~~~~~~~~~~~~~~~~~~~~~~~~~~~~~%


