%-------------------------------%
%  Author: Alessandro Sciarra   %
%    Date: 17 Jun 2019          %
%-------------------------------%

%~~~~~~~~~~~~~~~~~~~~~~~~~~~~~~~~~~~~~~~~~~~~%
\begin{frame}[noframenumbering, plain]{After all it is simple...}
    \centering
    \includegraphics[width=0.9\textwidth]{ToyStory}
\end{frame}
%~~~~~~~~~~~~~~~~~~~~~~~~~~~~~~~~~~~~~~~~~~~~%
\begin{frame}{Strings, strings everywhere}
    \vspace{-3mm}
    The term string refers to a sequence of characters which is treated as a single unit:
    \begin{itemize}
        \item The command's name is a string
        \item Each argument of a command is a string
        \item Variable names are strings
        \item The contents of variables are strings as well
        \item A filename is a string
        \item Most files contain strings
    \end{itemize}
    \begin{varblock}{alerted}[0.7\textwidth]{Strings do not have any intrinsic meaning}
        Their meaning is defined by how and where they are used. 
    \end{varblock}
    \begin{varblock}{}[0.85\textwidth]{We have \textbf{all} the responsibility}
        We need to be sure everything that needs to be separated is separated properly, and everything that needs to stay together stays together properly!
    \end{varblock}
    \begin{tikzpicture}[remember picture, overlay]
        \node[anchor=south, font=\tiny] at (current page.south) {I will loosely use the term \textbf{string} throughout the lecture, mostly referring to a portion of text contained in a variable};
    \end{tikzpicture}
\end{frame}
%~~~~~~~~~~~~~~~~~~~~~~~~~~~~~~~~~~~~~~~~~~~~%
\begin{frame}[fragile]{Types of commands}{There are basically 5 different classes of commands}
    \vspace{-3mm}
    \setlength{\columnsep}{-5mm}
    \begin{multicols}{5}
        \begin{enumerate}
            \item \tc<2>{PP}{Aliases}
            \item \tc<3>{PP}{Functions}
            \item \tc<4>{PP}{Builtins}
            \item \tc<5>{PP}{Keywords}
            \item \tc<6>{PP}{Executables}
        \end{enumerate}
    \end{multicols}
    \begin{overlayarea}{\textwidth}{0.7\textheight}
        \begin{onlyenv}<2>
            \begin{lstlisting}[style=MyBash]
                for, while, do done man apropos "hellp" { [[ ]] } !hi ! 
                # if else ! [[
            \end{lstlisting}
        \end{onlyenv}
        \begin{onlyenv}<3>
            
        \end{onlyenv}
        \begin{onlyenv}<4>
            
        \end{onlyenv}
        \begin{onlyenv}<5>
            
        \end{onlyenv}
        \begin{onlyenv}<6>
            
        \end{onlyenv}
    \end{overlayarea}
\end{frame}
%~~~~~~~~~~~~~~~~~~~~~~~~~~~~~~~~~~~~~~~~~~~~%
