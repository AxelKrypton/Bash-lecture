%-------------------------------%
%  Author: Alessandro Sciarra   %
%    Date: 4 Jul 2019           %
%-------------------------------%

%~~~~~~~~~~~~~~~~~~~~~~~~~~~~~~~~~~~~~~~~~~~~%
\begin{frame}{Conditial loops}{Often we need to repeat things: Copy and paste is not the solution!}
    \vspace{-3mm}
    \begin{description}[\texttt{while}xxx]
        \item[\texttt{while}] Repeat as long as a command is executed successfully (exit code is 0)
        \item[\texttt{until}] Repeat as long as a command is executed unsuccessfully (exit code is not 0)
        \item[\texttt{for}] It comes in two versions
                            \begin{itemize}
                                \item to iterate over a list
                                \item to iterate over an integer index as in \texttt{C}
                            \end{itemize}
    \end{description}
    \begin{varblock}{quote}[0.9\textwidth]{Which one should be used?}
        There will nearly always be multiple approaches to solving a problem.
        The test of your skill soon won't be about solving a problem as much as about how best to solve it.
    \end{varblock}
    \begin{varblock}{}[0.6\textwidth]{While instead of until}
        The \,\bash|until|\, loop is barely ever used, if only because it is pretty much exactly the same as \,\bash|while !|
    \end{varblock}
\end{frame}
%~~~~~~~~~~~~~~~~~~~~~~~~~~~~~~~~~~~~~~~~~~~~%
\begin{frame}[fragile]{The \,\bash|while|\, and \,\bash|until|\, keywords}
    \begin{onlyenv}<1>
        \begin{lstlisting}[style=MyBash, numbers=none]
            while COMMAND; do
                # Body of the loop: entered if COMMAND's exit code = 0
            done
            
            until COMMAND; do
                # Body of the loop: entered if COMMAND's exit code @|\color{comment-color}$\neq$|@0
            done
        \end{lstlisting}
        \bigskip
        \begin{itemize}
            \item Testing command like \bash{[} or preferably \bash{[[} are often used
            \item Infinite loops can be achieved using the builtins \,\bash|true|, \bash|false|\, and \,\bash|:|
            \item Use the \,\bash|continue|\, builtin to skip ahead to the next iteration of a loop without executing the rest of the body
            \item use the \,\bash|break|\, builtin to jump out of the loop and continue with the script after it
            \item Both \,\bash|continue|\, and \,\bash|break|\, accept an optional integer to act on nested loops
        \end{itemize}
    \end{onlyenv}
    \begin{onlyenv}<2>
        \begin{lstlisting}[style=MyBash]
            $ while true; do   # while :; do
            >   echo "Infinite loop"
            >   sleep 3
            > done
            |+Infinite loop+|
            |+Infinite loop+|
            |+^C+|    # Press CTRL-C
        \end{lstlisting}
        \medskip
        \begin{lstlisting}[style=MyBash]
            # An example of countdown...
            $ deadline=$(date -d "8 seconds" +'%s'); \
            > now=$(date +'%s'); \
            > while (( deadline - now > 0 )); do
            >   echo "$((deadline - now)) seconds to BOOM!"
            >   sleep 3
            >   now=$(date +'%s')
            > done; echo 'BOOOOOM!!'
            |+8 seconds to BOOM!
            5 seconds to BOOM!
            2 seconds to BOOM!
            BOOOOOM!!+|
        \end{lstlisting}
    \end{onlyenv}
\end{frame}
%~~~~~~~~~~~~~~~~~~~~~~~~~~~~~~~~~~~~~~~~~~~~%
\begin{frame}[fragile]{The \,\bash|for|\, keywords}
    \begin{onlyenv}<1>
        \begin{lstlisting}[style=MyBash, numbers=none]
            for VARIABLE in WORDS; do
                # Body of the loop: VARIABLE set to WORD
            done
            
            for (( EXPR1; EXPR2; EXPR3 )) # Expressions can be empty
                # Body of the loop
            done
        \end{lstlisting}
        \bigskip
        \begin{itemize}
            \item In the second form,
                  \begin{enumerate}
                      \item it starts by evaluating the first arithmetic expression;
                      \item it repeats as long as the second arithmetic expression is successful;
                      \item at the end of each loop evaluates the third arithmetic expression.
                  \end{enumerate}
            \item Bash takes the characters between \bash|in| and the end of the line and 
                  \begin{itemize}
                      \item[$\circ$] it splits them up into words;
                      \item[$\circ$] this splitting is done on spaces and tabs, just like argument splitting;
                      \item[$\circ$] if there are any unquoted substitutions, \alert{they will be word-split as well} (using \bash|IFS|).
                  \end{itemize}
        \end{itemize}
    \end{onlyenv}
    \begin{onlyenv}<2>
        \begin{lstlisting}[style=MyBash, xrightmargin=4mm]
            $ for (( ; 1; )); do echo "Infinite loop"; sleep 1; done
            |+Infinite loop+|
            |+Infinite loop+|
            |+^C+|    # Press CTRL-C
        \end{lstlisting}
        \medskip
        \begin{lstlisting}[style=MyBash, xrightmargin=4mm]
            $ for index in {0,1}{0,1}; do
            >   echo "${index} in base 2 is $(( 2#${index})) in base 10"
            > done; unset index
            |+00 in base 2 is 0 in base 10
            01 in base 2 is 1 in base 10
            10 in base 2 is 2 in base 10
            11 in base 2 is 3 in base 10+|
        \end{lstlisting}
        \medskip
        \begin{lstlisting}[style=MyBash, xrightmargin=4mm]
            # BAD code!
            
            $ for file in $(ls *.mp3); do   # AAAARGH!
            >   rm "$file"
            > done; unset file
        \end{lstlisting}
    \end{onlyenv}
    \begin{onlyenv}<3>
        \begin{lstlisting}[style=MyBash, xrightmargin=4mm]
            $ ls
            Happy birthday.mp3
            $ for file in $(ls *.mp3); do   # AAAARGH!
            >   rm "$file"
            > done; unset file
            |+rm: cannot remove `Happy': No such file or directory+|
            |+rm: cannot remove `birthday.mp3': No such file or directory+|
        \end{lstlisting}
        \medskip
        You want to quote it, you say?
        \medskip
        \begin{lstlisting}[style=MyBash, xrightmargin=4mm]
            $ ls
            Happy birthday.mp3   Hello.mp3
            $ for file in "$(ls *.mp3)"; do   # AAAARGH!
            >   rm "$file"
            > done; unset file
            |+rm: cannot remove `Happy birthday.mp3   Hello.mp3': No such file or directory+|
        \end{lstlisting}
        \medskip
        So, what do we do?
    \end{onlyenv}
    \begin{onlyenv}<4>
        \begin{varblock}{alerted}[0.9\textwidth]{\textbf{Use globs!}}
            Bash \textbf{does} know that it is dealing with filenames, and it \textbf{does} know what the filenames are, and as such it can split them up nicely!
        \end{varblock}
        \medskip
        \begin{lstlisting}[style=MyBash, xrightmargin=4mm]
            $ ls
            Happy birthday.mp3   Hello.mp3
            $ for file in *.mp3; do   # GOOD code
            >   rm "$file"
            > done; unset file
        \end{lstlisting}
        \medskip
        \begin{varblock}{}[0.9\textwidth]{Do not be tempted}
            You might argue that, if there are no spaces in filenames, then you would have no troubles.
            But would you throw in the air a sharp knife trying to catch it afterwards, just because if you do not touch the blade it would be safe?
        \end{varblock}
    \end{onlyenv}
\end{frame}







