%-------------------------------%
%  Author: Alessandro Sciarra   %
%    Date: 21 Oct 2020          %
%-------------------------------%

%~~~~~~~~~~~~~~~~~~~~~~~~~~~~~~~~~~~~~~~~~~~~%
\begin{frame}{Average students retention rates}{\ssmall\textbf{Source:} The learning Pyramid, researched and created by the National training Laboratories in Betel, Maine.}
    \vspace{-3mm}
    \begin{center}
        \resizebox{!}{0.8\textheight}{
            \begin{tikzpicture}
                \path coordinate (A) at (0,0)
                      coordinate (B) at (0,1.4)
                      coordinate (C) at (0,2.8)
                      coordinate (D) at (0,4.2)
                      coordinate (E) at (0,5.6)
                      coordinate (F) at (0,7.0)
                      coordinate (G) at (0,8.4)
                      coordinate (O) at ($(A)+(150:5.5)-(0.5,0)$);
                \draw (O) -- ++(30:2); %Just draw axis behind pyramid
                %Draw Pyramid -> new long edge is previous layer long edge minus half vertical distance (e.g. distance between A and B)
                \pyramidLayer[Teach others]{A}{1.2}{5.5}{red}{90\%};
                \pyramidLayer[Practice doing]{B}{1.2}{4.8}{PB}{75\%};
                \pyramidLayer[Discussion]{C}{1.2}{4.1}{orange!30!yellow}{50\%};
                \pyramidLayer[Demonstration]{D}{1.2}{3.4}{PS}{30\%};
                \pyramidLayer[Audiovisual]{E}{1.2}{2.7}{PP}{20\%};
                \pyramidLayer[Reading]{F}{1.2}{2.0}{PT}{10\%};
                \pyramidLayer[Lecture]{G}{1.2}{1.3}{magenta}{5\%};
                \draw[from] (O) ++(0,-0.5) -- ++(0,8) node[right, anchor=north west] {Retention rate};
                \draw[to]   (O) -- ++(330:6.2) node[right] {Passive to active learning};
            \end{tikzpicture}
        }
    \end{center}
\end{frame}
%~~~~~~~~~~~~~~~~~~~~~~~~~~~~~~~~~~~~~~~~~~~~%
\begin{frame}{Do not be afraid}
    \vspace{-4mm}
    \begin{itemize}
        \item \alert{You received a huge amount of information}
        \item Nobody can expect to have everything well settled in their head \textbf{right now}
        \item \PS{\textbf{You have the material of the course, work on it}}
        \item Links in the slides are important to make a further step forward
    \end{itemize}
    \vspace{-2mm}
    \begin{varblock}{}[0.6\textwidth]{It will be probably enough for your next future}<2>
        The material discussed in these 5 days is enough to be able to say that you know how bash works!
    \end{varblock}
    \vspace{5mm}
    \begin{columns}[c]
        \begin{column}{0.3\textwidth}
            \begin{tikzpicture}
                \node[inner sep=0mm] (fig) {\includegraphics[width=35mm, clip, trim=8mm 5mm 8mm 0]{KnowBash.jpg}};
                \begin{scope}[scope on=<2>]
                    \draw[red, line width=2mm] (fig.south west) -- (fig.north east);
                    \path[overlay, from] (fig.north east) ++(2pt,2pt) edge[out=45, in=180] node[pos=1, right=2pt, font=\scriptsize, text=PS] {Well, not yet, I would say!} ++(6mm,4pt);
                \end{scope}
            \end{tikzpicture}
        \end{column}
        \begin{column}{0.5\textwidth}
            \begin{varblock}{alert}[\textwidth]{}<2>
                \alert{Unfortunately it is not that straightforward!}
            \end{varblock}
            \begin{varblock}{quote}[\textwidth]{}<2>
                \normalfont
                You need practice to digest most of aspects and our job offers plenty of possibility for that!
            \end{varblock}
        \end{column}
    \end{columns}
\end{frame}
%~~~~~~~~~~~~~~~~~~~~~~~~~~~~~~~~~~~~~~~~~~~~%
\begin{frame}{So, what should I do now?}
    \vspace{-2mm}
    \begin{enumerate}
        \setlength{\itemsep}{1mm}
        \item<1-> \PB{\textbf{It is matter of doing small steps!}}
                  \begin{itemize}
                    \item Whenever reasonable, give it a chance
                    \item Take every daily-life opportunity to recall aspects of Bash
                    \item These five days should be a very good and complete starting point for your work!
                  \end{itemize}
        \item<2-> \PS{\textbf{I hope you enjoyed the course}}
                  \begin{itemize}
                      \item I would love to get feedback from each of you
                      \item Feel free to send me an email \,\PS{\href{mailto:sciarra@itp.uni-frankfurt.de}{{\small\Letter}}}
                            \begin{itemize}
                                \item Did I meet your expectation? If no, why?
                                \item Was the lecture \textbf{prohibitively} dense? \Remark{Dense it had to be\ldots}
                                \item What would you change--add--remove?
                            \end{itemize}
                  \end{itemize}
        \item<3-> Last but not least, \PP{\textbf{haven't you already planned your trip to Iceland?}} $\;$\raisebox{-1pt}{\includegraphics[height=12pt]{Wink}}
    \end{enumerate}
    \begin{tikzpicture}[remember picture, overlay]
        \node[anchor=south, cloud, cloud puffs=19, aspect=4, draw=PB, fill=PP!20, text=PB, font=\large\bfseries, visible on=<3>] at ($(current page.south)+(0mm,8mm)$) {Thank you for attending!};
    \end{tikzpicture}
\end{frame}