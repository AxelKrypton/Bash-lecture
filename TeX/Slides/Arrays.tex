%-------------------------------%
%  Author: Alessandro Sciarra   %
%    Date: 23 Sep 2020          %
%-------------------------------%

%~~~~~~~~~~~~~~~~~~~~~~~~~~~~~~~~~~~~~~~~~~~~%
\begin{frame}[fragile]{Motivation}
    \vspace{-3mm}
    \begin{itemize}
        \item \textbf{Strings} are without any doubt the most used parameter type
        \item They are also the most \textbf{misused} parameter type, though!
        \item Strings hold \alert{\textbf{just one}} element!
        \item Capturing a list in a string is very often \PP{plain wrong}, even if it might work\ldots
        \item A parameter contains just one string of characters, no matter the meaning of them
        \item If you put multiple filenames in a string, maybe to iterate over them at a later point, which delimiter would you use?
        \item Is there a delimiter that is not accepted to be part of a filename? Mmmh\ldots
    \end{itemize}
    \begin{lstlisting}[style=MyBash, numbers=none]
        # This does NOT work in the general case
        $ files=$(ls ~/*.jpg); cp ${files} /backups/    # BAD code!
    \end{lstlisting}
    \begin{varblock}{}[0.75\textwidth]{Array}
        An array is a numbered list of strings: It maps integers to strings. 
    \end{varblock}
    \FrameRemark{For filenames in Linux, there are usually almost no restrictions, apart from \texttt{/} and \texttt{\textbackslash{}0}, you are allowed to use anything.}
\end{frame}
%~~~~~~~~~~~~~~~~~~~~~~~~~~~~~~~~~~~~~~~~~~~~%
\begin{frame}[fragile]{Creating an array}
    \vspace{-1mm}\centering
    \PB{\texttt{array=(\ldots)}}\, The most used syntax: space separated list in parenthesis
    \smallskip
    \begin{lstlisting}[style=MyBash, numbers=none]
        array=(one two "three and four") # Equivalent to:
        array=([0]=one [1]=two [2]="three and four")
        # Sparse array perfectly legal!
        array=([0]=First [11]=Middle [23]=Last)
    \end{lstlisting}
    \medskip
    \PB{\texttt{array[n]=}}\, Setting single elements works for undeclared variable, too
    \smallskip
    \begin{lstlisting}[style=MyBash, numbers=none]
        array[0]=First
        array[11]=Middle
        array[23]=Last
    \end{lstlisting}
    \medskip
    \PB{\texttt{declare -a array}}\, Rarely used but possible
    \smallskip
    \begin{lstlisting}[style=MyBash, numbers=none]
        declare -a array  # Because of declare,
        array=One         # equivalent to array[0]=One, AVOID!
        array[1]=Two 
    \end{lstlisting}
    \bigskip
    \PB{\texttt{array+=(\ldots)}}\, Concatenate r.h.s. array to l.h.s. array\par
    \FrameRemark{The command \bash|read -a array| allows to store what has been read in an array.}
\end{frame}
%~~~~~~~~~~~~~~~~~~~~~~~~~~~~~~~~~~~~~~~~~~~~%
\begin{frame}[fragile]{Storing filenames into an array}
    \begin{varblock}{alerted}[0.95\textwidth]{Take home message}
        If you want to fill an array with filenames, then you'll probably want to use \alert{\textbf{globs}} in there!
    \end{varblock}
    \medskip
    \begin{onlyenv}<1>
        \begin{lstlisting}[style=MyBash, numbers=none, xleftmargin=30mm, xrightmargin=30mm]
            $ files=(~/"My Photos"/*.jpg)
        \end{lstlisting}
        \medskip
        \begin{itemize}
            \item Here we quoted the \bash{"My Photos"} part because it contains a space
            \item If we hadn't quoted it, Bash would have split it up into\\ \centerline{\small\bash{files=('~/My' 'Photos/'*.jpg)}} which is obviously not what we want!
            \item We quoted \textbf{only} the part that contained the space: We cannot quote the \PB{\texttt{\textasciitilde}} or the \PB{\texttt{*}}
            \item If we do, they'll become literal and Bash won't treat them as special characters anymore!
        \end{itemize}
    \end{onlyenv}
    \begin{onlyenv}<2>
        \begin{lstlisting}[style=MyBash, numbers=none]
            # Please, really, use globs instead of commands!
            
            $ files=$(ls)      # BAD, BAD, BAD!
            $ files=($(ls))    # STILL BAD!
            
            # So, how should you do?
            
            $ files=(*)        # GOOD!
        \end{lstlisting}
        \medskip
        \begin{varblock}{example}[0.95\textwidth]{Globs know about files}
            Using glob patterns is possible to create an array with filenames in different entries!
        \end{varblock}
    \end{onlyenv}
\end{frame}
%~~~~~~~~~~~~~~~~~~~~~~~~~~~~~~~~~~~~~~~~~~~~%
\begin{frame}[fragile]{Accessing array's content (I)}{Definition, single elements and length}
    \vspace{-2mm}\centering
    \PB{\texttt{declare -p array}}\, It prints the definition of the variable {\tiny\{~not so useful in scripts~\}}
    \smallskip
    \begin{lstlisting}[style=MyBash, numbers=none]
        $ declare -p array
        |+declare -a array='([0]="one" [1]="two and four")'+|
    \end{lstlisting}
    \medskip
    \PB{\texttt{\$\{array[n]\}}}\, It retrieves the n-th element {\tiny\{~no error if entry missing in array~\}}
    \smallskip
    \begin{lstlisting}[style=MyBash, numbers=none]
        $ echo "${array[1]}"; echo "_${array[3156]}_"
        |+two and four+|
        |+__+|
        $ echo "${array[0+1]}" # [...] is an arithmetic context
        |+two and four+|
    \end{lstlisting}
    \medskip
    \PB{\texttt{\$\{\#array[@]\}}}\, It retrieves the length of the array {\tiny\{~same as \PB{\texttt{\$\{\#array[*]\}}}~\}}
    \smallskip
    \begin{lstlisting}[style=MyBash, numbers=none]
        $ echo "${#array[@]}"; array[7]=Hello; echo "${#array[@]}";
        |+2+|  # Quoting the expansion does not change anything!
        |+3+|
        $ echo "${#array[7]}"; echo "${#array[-1]}";
        |+5+|
        |+5+|  # What does this mean?
    \end{lstlisting}
    \FrameRemark{Curly braces in parameter expansions accessing arrays are mandatory.}
\end{frame}
%~~~~~~~~~~~~~~~~~~~~~~~~~~~~~~~~~~~~~~~~~~~~%
\begin{frame}[fragile]{Accessing array's content (II)}{All elements at once}
    \vspace{-2mm}\centering
    \tc{strings-color}{\texttt{"\$\{array[@]\}"}}\, Bash replaces this syntax with each element properly quoted
    \smallskip
    \begin{lstlisting}[style=MyBash, numbers=none]
        $ printf '%s\n' "${array[@]}"
        |+one+|
        |+two and four+|
        |+Hello+|
    \end{lstlisting}
    \medskip
    \tc{strings-color}{\texttt{"\$\{array[*]\}"}}\, expands to a \bash|IFS|-first-character-separated list of the array entries
    \smallskip
    \begin{lstlisting}[style=MyBash, numbers=none]
        $ IFS=':'; printf '%s\n' "${array[*]}"; unset IFS
        |+one:two and four:Hello+|
    \end{lstlisting}
    \medskip
    Unquoted \PB{\texttt{\$\{array[@]\}}} and \PB{\texttt{\$\{array[*]\}}} let \alert{word splitting} kick in!
    \smallskip
    \begin{lstlisting}[style=MyBash, numbers=none]
        $ printf '%s\n' ${array[@]} # The same with ${array[*]}
        |+one+|
        |+two+|
        |+and+|
        |+four+|
        |+Hello+|
    \end{lstlisting}
\end{frame}
%~~~~~~~~~~~~~~~~~~~~~~~~~~~~~~~~~~~~~~~~~~~~%
\begin{frame}[fragile]{Accessing array's content (III)}{All indices}
    \vspace{-2mm}\centering
    \tc{strings-color}{\texttt{"\$\{!array[@]\}"}}\, Bash replaces this syntax with the indices of the entries that are set
    \smallskip
    \begin{lstlisting}[style=MyBash, numbers=none]
        $ printf '%d\n' "${!array[@]}"
        |+0+|
        |+1+|
        |+7+|
    \end{lstlisting}
    \medskip
    \tc{strings-color}{\texttt{"\$\{!array[*]\}"}}\, expands to a \bash|IFS|-first-character-separated list of the array indices
    \smallskip
    \begin{lstlisting}[style=MyBash, numbers=none]
        $ IFS=':'; printf '%s\n' "${!array[*]}"; unset IFS
        |+0:1:7+|
    \end{lstlisting}
    \medskip
    \begin{varblock}{}[0.9\textwidth]{In any case, prefer quoted versions!}
        Unquoted \PB{\texttt{\$\{!array[@]\}}} and \PB{\texttt{\$\{!array[*]\}}} let \alert{word splitting} kick in.
        This, for \textbf{normal} arrays, is harmless, since indices are just integers.
    \end{varblock}    
\end{frame}
%~~~~~~~~~~~~~~~~~~~~~~~~~~~~~~~~~~~~~~~~~~~~%
\begin{frame}[fragile]{Iterating over array elements}
    \vspace{-3mm}
    \begin{itemize}
        \item Depending on the needs you can iterate over the elements or the indices in the array
        \item When iterating over the \textbf{\PP{elements}}, 99\% of the time you will need \,\tc{strings-color}{\texttt{"\$\{array[@]\}"}}
        \item When iterating over the \textbf{\PP{indices}}, 99\% of the time you will need \,\tc{strings-color}{\texttt{"\$\{!array[@]\}"}}
        \item To implicitly iterate over arrays is possible using different commands (e.g.\ \bash{printf}, \bash{cp})
    \end{itemize}
    \begin{lstlisting}[style=MyBash]
        $ array=(one "two and four")
        $ for entry in "${array[@]}"; do
        >     echo "${entry}"
        > done
        |+one+|
        |+two and four+|
        $ for index in "${!array[@]}"; do
        >     echo "array[${index}]=${array[index]}"
        > done
        |+array[0]=one+|
        |+array[1]=two and four+|
        $ unset entry index
        # Coming back to motivation: GOOD code!
        $ files=(~/*.jpg); cp "${files[@]}" /backups/
    \end{lstlisting}
\end{frame}
%~~~~~~~~~~~~~~~~~~~~~~~~~~~~~~~~~~~~~~~~~~~~%
\begin{frame}[fragile]{Deleting arrays or array elements}
    \vspace{-3mm}
    \begin{onlyenv}<1>
        \begin{itemize}
            \item To delete an element of an array, pass it to the \bash{unset} builtin
        \end{itemize}
        \begin{lstlisting}[style=MyBash, belowskip=-5mm]
            $ array=({a..e}); unset 'array[2]' 'array[3]'
            $ for index in "${!array[@]}"; do
            >     echo "array[${index}]=${array[index]}"
            > done; unset 'index'
            |+array[0]=a
            array[1]=b
            array[4]=e+|
        \end{lstlisting}
        \begin{itemize}
            \item If you pass the name of an array to \bash|unset|, the whole array will be unset!
        \end{itemize}
        \begin{lstlisting}[style=MyBash]
            $ array=({a..c})
            $ for index in "${!array[@]}"; do
            >     echo "array[${index}]=${array[index]}"
            > done; unset 'index'
            |+array[0]=a
            array[1]=b
            array[2]=c+|
            $ unset array
            $ echo "_${array[@]}_ -> ${#array[@]} entries"
            |+__ -> 0 entries+|
        \end{lstlisting}
    \end{onlyenv}
    \begin{onlyenv}<2>
        \begin{varblock}{alerted}[0.8\textwidth]{}
            \large \PT{Do not forget to \textbf{quote your variable} to avoid file pattern matching!}
        \end{varblock}
        \begin{lstlisting}[style=MyBash, emph={[7]a, a0}]
            $ a=(file{1..3})
            $ ls
            |+a0   Day_1.tex   Day_1.pdf+|
            $ a0='Hello'
            $ printf '%s\n' "${a[@]}"
            |+file1
            file2
            file3+|
            $ echo "${a0}"
            |+Hello+|
            $ unset a[0] # <--- BAD, BAD, BAD CODE!
            $ printf '%s\n' "${a[@]}"
            |+file1+|   # Wait! I unset the first element, didn't I?
            |+file2
            file3+|
            $ echo "_${a0}_"
            |+__+|                 # Is it clear what happened?
        \end{lstlisting}
    \end{onlyenv}
    \FrameRemark[2]{In doubt, always quote the variable passed to \bash|unset|. Using either the \texttt{-v} or \texttt{-f} option is also encouraged!}
\end{frame}
%~~~~~~~~~~~~~~~~~~~~~~~~~~~~~~~~~~~~~~~~~~~~%
\begin{frame}[fragile]{Sparse arrays}
    \medskip
    \begin{columns}[c]
        \begin{column}{0.75\textwidth}
            \begin{varblock}{alerted}[0.5\textwidth]{}
                \Large \alert{Arrays might be sparse!}
            \end{varblock}
        \end{column}
        \begin{column}{0.25\textwidth}
        \end{column}
    \end{columns}
    \bigskip
    \begin{itemize}
        \item Don't assume that your indices are sequential
        \item If the index values matter, always iterate over the indices {\tiny\{~instead of making assumptions about them\ldots~\}}
        \item If you loop over the values instead, don't assume anything about indices
        \item In particular, don't assume that just because you're currently in the first iteration of your loop, that you must be on index 0\ldots
    \end{itemize}
    \begin{varblock}{example}[0.63\textwidth]{\large If you need a non-sparse array, just make it such!}
        \begin{lstlisting}[style=MyBash, numbers=none, aboveskip=2mm, belowskip=-5mm, xleftmargin=13mm, xrightmargin=13mm]
            $ array=("${array[@]}")
        \end{lstlisting}
    \end{varblock}
    \begin{tikzpicture}[remember picture, overlay]
        \node[anchor=north east] at ($(current page.north east)-(5mm,5mm)$) {\includegraphics[width=3cm, clip, trim=0 17mm 0 0]{Assumptions}};
    \end{tikzpicture}
\end{frame}
%~~~~~~~~~~~~~~~~~~~~~~~~~~~~~~~~~~~~~~~~~~~~%
\begin{frame}[fragile]{Associative arrays}{Since Bash v4.0 (2009)}
    \vspace{-5mm}
    \begin{overlayarea}{\textwidth}{0.7\textheight}
        \begin{itemize}
            \item An associative array is \PP{a map of strings to strings}
            \item It has to be \textbf{explicitly declared} as such via \,\bash|declare -A|
            \item Since keys are strings, they might contain spaces!\\
                  $\Rightarrow\,$ Iterating over keys requires quotes: \tc{strings-color}{\texttt{"\$\{!array[@]\}"}}
        \end{itemize}
        \begin{onlyenv}<1>
            \begin{lstlisting}[style=MyBash, xrightmargin=1mm]
                $ declare -A dict
                $ dict[Apartment]="Die Wohnung"
                $ declare -p dict
                |+declare -A dict='([Apartment]="Die Wohnung" )'+|
                $ dict[Water]="Das Wasser"
                $ for key in "${!dict[@]}"; do
                >     printf 'EN: %20s -> DE: %s\n' "${key}" "${dict[$key]}"
                > done
                |+EN:                Water -> DE: Das Wasser
                EN:            Apartment -> DE: Die Wohnung+|
                $ echo "${#dict[@]}"; unset dict
                |+2+|
                $ echo "${#dict[@]}"
                |+0+|
            \end{lstlisting}
        \end{onlyenv}
        \begin{onlyenv}<2>
            \begin{lstlisting}[style=MyBash, firstnumber=15, xrightmargin=1mm]
                $ declare -A dict
                $ dict[Die Wohnung]="Apartment"
                $ dict[Das Wasser]="Water"
                $ for key in ${!dict[@]}; do   # AAAAAARGH!
                >     printf 'DE: %20s -> EN: _%s_\n' "${key}" "${dict[$key]}"
                > done
                |+DE:                  Die -> EN: __
                DE:              Wohnung -> EN: __
                DE:                  Das -> EN: __
                DE:               Wasser -> EN: __+|
                $ echo "${#dict[@]}"; unset -v 'dict[Das Wasser]'
                |+2+|
                $ echo "${#dict[@]}"; unset -v 'dict'
                |+1+|
            \end{lstlisting}
        \end{onlyenv}
    \end{overlayarea}
\end{frame}
%~~~~~~~~~~~~~~~~~~~~~~~~~~~~~~~~~~~~~~~~~~~~%
\begin{frame}[fragile]{Associative arrays: Remarks}{Since Bash v4.0 (2009)}
    \vspace{-5mm}
    \begin{overlayarea}{\textwidth}{0.7\textheight}
        \begin{itemize}
            \item \alert{The order of both keys and elements} of an associative array is \alert{unpredictable}\\
                  $\Rightarrow\,$ They are \textbf{not} well suited to store lists that need to be processed in order
            \item If you use a parameter as key of an associative array, \alert{you must use the \texttt{\$} sign}\\
                  $\Rightarrow\,$ Indeed, it makes sense: The name of the parameter might simply be a valid key
        \end{itemize}
        \begin{onlyenv}<1>
            \begin{lstlisting}[style=MyBash]
                $ declare -A array
                $ index=1; for entry in one two three four five six; do
                >     array[${entry}]=$((index++))
                > done
                $ echo "${array[@]}"
                |+4 1 5 6 2 3+|
                $ echo "${!array[@]}"
                |+four one five six two three+|
                $ unset array entry index
            \end{lstlisting}
        \end{onlyenv}
        \begin{onlyenv}<2>
            \begin{lstlisting}[style=MyBash, firstnumber=10, emph={[7]associative, indexed}]
                $ indexed=("one" "two")  index=0  key="foo"
                $ declare -A associative=|+([foo]=bar [alpha]=omega)+|
                $ echo "${indexed[$index]}"
                |+one+|
                $ echo "${indexed[index]}"
                |+one+|
                $ echo "${indexed[index + 1]}"
                |+two+|
                $ echo "${associative[$key]}"
                bar
                $ echo "_${associative[key]}_"
                |+__+|
                $ echo "_${associative[key + 1]}_"
                |+__+|
                $ unset indexed associative index key
            \end{lstlisting}
        \end{onlyenv}
    \end{overlayarea}
\end{frame}
%~~~~~~~~~~~~~~~~~~~~~~~~~~~~~~~~~~~~~~~~~~~~%
\begin{frame}[fragile]{The power of parameter expansion on arrays}
    \vspace{-5mm}
    \begin{overlayarea}{\textwidth}{0.7\textheight}
         \begin{itemize}
             \item Arrays are very flexible, because they are well integrated with the other shell expansions
             \item Any parameter expansion can be carried out on individual array elements
             \item Parameter expansions can equally well apply to an entire array!
             \item To use parameter-expansion manipulations on all array entries, use \PB{\texttt{[@]}} and \PB{\texttt{[*]}}
             \item It is critical that these special expansions are properly quoted
         \end{itemize}
        \begin{onlyenv}<1>
            \begin{lstlisting}[style=MyBash, xrightmargin=3mm]
                $ array=(alpha beta gamma)
                $ echo "${array[-1]:2:2}"   # Substring of last entry
                |+mm+|
                $ echo "${array[@]#a}"      # Chop 'a' from the beginning
                |+lpha beta gamma+|
                $ echo "${array[@]%a}"      # Chop 'a' from the end
                |+alph bet gamm+|
                $ echo "${array[@]//a/_}"   # Substitute all 'a' by '_'
                |+_lph_ bet_ g_mm_+|
                $ echo "${array[@]/#a/_}"   # Substitute 'a' by '_' at start
                |+_lpha beta gamma+|
                $ echo "${array[@]/%a/_}"   # Substitute 'a' by '_' at end
                |+alph_ bet_ gamm_+|
            \end{lstlisting}
        \end{onlyenv}
        \begin{onlyenv}<2>
            \begin{lstlisting}[style=MyBash, xrightmargin=3mm, firstnumber=14]
                $ echo "${array[@]/#/prefix_}"  # Add prefix to all entries
                |+prefix_alpha prefix_beta prefix_gamma+|
                $ echo "${array[@]/%/_suffix}"  # Add suffix to all entries
                |+alpha_suffix beta_suffix gamma_suffix+|
                $ echo "${array[@]^^}"          # Capitalise all characters
                |+ALPHA BETA GAMMA+|
                $ echo "${array[3]?Third entry is unset}"
                |+bash: array[3]: Third entry is unset+|
                $ echo "${array[3]=delta}"
                |+delta+|
                $ echo "${array[@]^}"
                |+Alpha Beta Gamma Delta+|
                $ unset array
            \end{lstlisting}
        \end{onlyenv}
    \end{overlayarea}
\end{frame}





