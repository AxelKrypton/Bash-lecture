%-------------------------------%
%  Author: Alessandro Sciarra   %
%    Date: 13 Sep 2019          %
%-------------------------------%

%~~~~~~~~~~~~~~~~~~~~~~~~~~~~~~~~~~~~~~~~~~~~%
\begin{frame}[fragile]{Require a minimum Bash version}
    \vspace{-3mm}
    \begin{varblock}{example}[0.6\textwidth]{}
        \PS{Using modern Bash features is important}, but be sure you can then use them where your scripts are needed!
    \end{varblock}
    \vspace*{-1mm}
    \begin{varblock*}{quote}[0.5\textwidth]{}
        \normalfont
        Bash \PQ{v2.0} $\;\longrightarrow\;$ Released in \PQ{1996}\\
        Bash \PT{v3.0} $\;\longrightarrow\;$ Released in \PT{2004}\\
        Bash \PP{v4.0} $\;\longrightarrow\;$ Released in \PP{2009}\\
        Bash \PP{v4.1} $\;\longrightarrow\;$ Released in \PP{2009}\\
        Bash \PP{v4.2} $\;\longrightarrow\;$ Released in \PP{2011}\\
        Bash \PP{v4.3} $\;\longrightarrow\;$ Released in \PP{2014}\\
        Bash \PS{v4.4} $\;\longrightarrow\;$ Released in \PS{2016}\\
        Bash \PS{v5.0} $\;\longrightarrow\;$ Released in \PS{2019}\\
        Bash \PS{v5.1} $\;\longrightarrow\;$ In progress \Remark{$\alpha$-, $\beta$-versions already out}
    \end{varblock*}
    \begin{lstlisting}[style=MyBash, numbers=none]
        $ echo "${BASH_VERSINFO[@]}"
        |+5 0 17 1 release x86_64-pc-linux-gnu+|
        $ tr ' ' '.' <<< "${BASH_VERSINFO[@]:0:3}"
        |+5.0.17+|
        # You can use the 'hash' builtin to check command existence 
    \end{lstlisting}
    \begin{tikzpicture}[remember picture, overlay, scope on=<2>]
        \path[from] (current page.center) ++(1.3,-0.88) edge[out=0, in=270]
            node[pos=1, above, PB, draw, rounded corners=3pt, text width=26mm, align=center, font=\small] {Best for empty arrays and \texttt{-u} shell option} ++(3.3,1);
    \end{tikzpicture}
    \FrameRemark{Keep in mind that locally install a recent Bash version is pretty straightforward $\;\to\;$ \URL[PB]{https://ftp.gnu.org/gnu/bash/}{Download Bash}.}
\end{frame}
%~~~~~~~~~~~~~~~~~~~~~~~~~~~~~~~~~~~~~~~~~~~~%
\begin{frame}{Be sensible!}
    \vspace{-2mm}
    \begin{varblock}{example}[0.95\textwidth]{}
        \large \PS{Using Bash v3.0 or higher means you can avoid ancient scripting techniques!}
    \end{varblock}
    \medskip
    \parbox[c][0.7\textheight][t]{\textwidth}{
        \begin{onlyenv}<1>
            \begin{enumerate}
                \item When writing Bash scripts, \alert{do not use the \bash{[} command.}\\
                      \PS{Bash has a far better keyword:} \bash{[[}
                \item It's also time \alert{you forget about \texttt{`...`}}.
                      It isn't consistent with the syntax of expansion and is terribly hard to nest without a dose of painkillers.
                      \PS{Use \texttt{\$(...)} instead.}
                \item And for heaven's sake, \textbf{use more quotes!}\\
                      \PS{Protect your strings and parameter expansions from word splitting.}\\
                      Word splitting will eat your babies if you don't quote things properly.
                \item Learn to \PS{use parameter expansions} instead of \bash|sed|, \bash|awk| or \bash|cut| to manipulate simple strings in Bash.
                \item \PS{Use built-in arithmetic} instead of \bash|expr| to do simple calculations, especially when just incrementing a variable!
            \end{enumerate}
        \end{onlyenv}
        \begin{onlyenv}<2>
            \begin{enumerate}
                \setcounter{enumi}{5}
                \item Always use the correct shebang and script extension.
                      If you're writing a Bash script, you need to put \bash|\#!/usr/bin/env bash|$\;$ at the top of your script.
                \item Stay away from scripting examples you see on the Web!
                      If you want to get inspired, understand them \textbf{completely}.
                      Any script you'll find on the Web is likely to be broken in some way.
                      \alert{Do not copy/paste from them!}
                \item Almost as important as the result of your code is the \textbf{readability of your code}.
                      Chances are that you aren't just going to write a script once and then forget about it.
            \end{enumerate}
            \begin{varblock}{quote}[0.85\textwidth]{}[Greg's Wiki]
                Trust me when I say, no piece of code is ever 100\% finished, with the exception of some very short and useless chunks of nothingness.
            \end{varblock}
        \end{onlyenv}
    }
\end{frame}
%~~~~~~~~~~~~~~~~~~~~~~~~~~~~~~~~~~~~~~~~~~~~%
\begin{frame}{Being aware is the first step}
    \begin{itemize}
        \item Writing good Bash scripts is after all tough
        \item Bash shell allows you to do quite a lot of things, offering you considerable flexibility
        \item However, it does very little to discourage misuse and other ill-advised behaviour
        \item Many awful and dangerous scripts end up in production or even in Linux distributions!!
        \item The result of these, and also your very own scripts written in a time of neglect, can often be \textbf{disastrous}
    \end{itemize}
    \begin{varblock*}{example}[0.45\textwidth]{Important reading}
        \URL[PT]{http://mywiki.wooledge.org/BashPitfalls}{Bash Pitfalls}\Remark[1em]{59 discussed items}\\
        \URL[PB]{https://wiki.bash-hackers.org/scripting/obsolete}{Obsolete and deprecated syntax}\\
        \URL[PP]{http://mywiki.wooledge.org/ShellHallOfShame}{Shell Hall of Shame}
    \end{varblock*}
\end{frame}
%~~~~~~~~~~~~~~~~~~~~~~~~~~~~~~~~~~~~~~~~~~~~%
\begin{frame}[fragile]{Never do anything along these lines in your scripts!}
    \begin{enumerate}
        \setcounter{enumi}{-1}
        \item<only@0> % Here I cheat and I do a horrible work-around to avoid a vertical shift I have no clue why it occurs!
        \item<only@1>
            \alert{\large\textbf{Don't ever parse the output of the \bash|ls| command!}}
            \begin{lstlisting}[style=MyBash, numbers=none, aboveskip=2mm, xrightmargin=38mm]
                ls -l | awk '{ print $8 }'  # BAD CODE!
            \end{lstlisting}
            \begin{itemize}
                \item \bash|ls| will mangle the filenames of files if they contain characters unsupported by your locale.
                      As a result, \alert{parsing filenames out of ls is \textbf{never} guaranteed to actually give you a file that you will be able to find}.
                      \bash|ls| might have replaced certain characters in the filename by question marks.
                \item Secondly, \bash|ls| separates lines of data by newlines.
                      This way, every bit of information on a file is on a new line.
                      \alert{\textbf{Unfortunately} filenames themselves can \textbf{also} contain newlines!}
                      This means that if you have a filename that contains a newline in the current directory, it will completely break your parsing and as a result, break your script!
                \item Last but not least,\\ $\to\;$\alert{\textbf{the output format} of \bash|ls -l| \textbf{is not guaranteed consistent across platforms}}.
            \end{itemize}
            \PS{Use e.g.\ \textbf{globbing} or \bash|stat| or \bash|find|.}\\
            {$\;$\small\URL[PB]{http://mywiki.wooledge.org/ParsingLs}{Instructive and very encouraged reading}}
        \item<only@2>
            \alert{\large\textbf{Don't ever test or filter filenames with the \bash|grep| command!}}
            \begin{lstlisting}[style=MyBash, numbers=none, aboveskip=2mm, xrightmargin=15mm]
                if echo "${file}" | fgrep '.txt'; then  # BAD CODE!

                ls *.txt | grep 'story'            # ALSO BAD CODE!
            \end{lstlisting}
            \begin{itemize}
                \item Unless your \bash|grep| pattern is really smart it will probably not be trustworthy
                \item In the first example above, the test would match both \texttt{story.txt} \textbf{and} \texttt{story.txt.exe}!
                \item If you make \bash|grep| patterns that are smart enough, they'll probably be so ugly, massive and unreadable that you shouldn't use them anyway!
            \end{itemize}
            \PS{The alternative is called \textbf{globbing}.}\\
            Bash has a feature called \textbf{Pathname Expansion}.
            This will help you enumerate all files that match a certain pattern.
            Also, you can use globs to test whether a filename matches a certain glob pattern.
        \item<only@3>
            \alert{\large\textbf{Avoid unnecessary use of the \bash|cat| command!}}
            \begin{lstlisting}[style=MyBash, numbers=none, aboveskip=2mm, xrightmargin=35mm]
                cat filename | grep 'pattern'  # BAD CODE!
            \end{lstlisting}
            \begin{itemize}
                \item Don't use \bash|cat| to feed a single file's content to a filter.
                      \bash|cat| is a utility used to \textbf{concatenate} the contents of several files together.
            \end{itemize}
            To feed the contents of a file to a process you will probably be able to\\ $\to\;$\PS{pass the filename as an argument to the program!}\\
            If the manual of the program does not specify any way to do this,\\ $\to\;$\PS{you should use redirection!}
        \item<only@3>
            \alert{\large\textbf{Don't use a \bash|for| loop to read the lines of a file!}}
            \begin{lstlisting}[style=MyBash, numbers=none, aboveskip=2mm, xrightmargin=35mm]
                for line in $(<file); do  # BAD CODE!
            \end{lstlisting}
            \begin{itemize}
                \item command substitutions remove any trailing newlines!
            \end{itemize}
            \PS{Use a \bash|while read| loop instead.}
        \item<only@4>
            \alert{\large\textbf{For the love of god and all that is holy, do not use \bash|seq| to count!}}
            \begin{lstlisting}[style=MyBash, numbers=none, aboveskip=2mm, xrightmargin=35mm]
                for index in $(seq 1 10); do  # BAD CODE!
            \end{lstlisting}
            \begin{itemize}
                \item Bash is able enough to do the counting for you.
                      You do not need to spawn an external application (especially a single-platform one) to do some counting and then pass that application's output to Bash for word splitting.
            \end{itemize}
            \PS{Use one of the following approaches.}
            \begin{itemize}
                \item In general, C-style for loops are the best method for implementing a counter
                      \begin{lstlisting}[style=MyBash, numbers=none, aboveskip=2mm, belowskip=-5mm, xrightmargin=38mm]
                          for ((i=1; i<=10; i++)); do  # BEST WAY
                      \end{lstlisting}
                \item If you actually wanted a stream of numbers separated by newlines as test input, consider
                      \begin{lstlisting}[style=MyBash, numbers=none, aboveskip=2mm, belowskip=-5mm, xrightmargin=65mm]
                          printf '%d\n' {1..10}
                      \end{lstlisting}
                      You can also loop over the result of a sequence expansion if the range is constant, but the shell needs to expand the full list into memory before processing the loop body.
                      This method can be wasteful and is less versatile than other arithmetic loops.
            \end{itemize}
        \item<only@5>
            \alert{\large\textbf{\bash|expr| is a relic of ancient Rome. Do not wield it!}}
            \begin{lstlisting}[style=MyBash, numbers=none, aboveskip=2mm, xrightmargin=48mm]
                i=`expr $i + 1`  # VERY BAD CODE!
            \end{lstlisting}
            \begin{itemize}
                \item You're basically
                      \begin{itemize}
                          \item spawning a new process,
                          \item calling another C program to do some arithmetic for you and
                          \item returning the result as a string to Bash.
                      \end{itemize}
                      Bash can do all this itself and so much faster, more reliably and in all, better!
            \end{itemize}
            \medskip
            \PS{In Bash you should use either $\;$\bash|let i++|$\;$ or $\;$\bash|((i++))|.}
    \end{enumerate}
    \begin{varblock}{}[0.75\textwidth]{Do not worry!}<only@5>
        We all did at least one of the previous in the past.\\
        You learn by your mistakes as soon you know they are mistakes!
    \end{varblock}
\end{frame}
%~~~~~~~~~~~~~~~~~~~~~~~~~~~~~~~~~~~~~~~~~~~~%
\begin{frame}{Clean code}
    \vspace{-3mm}
    \begin{itemize}
        \setlength{\itemsep}{1mm}
        \item Bash is a scripting language\\ $\to$ this does not mean at all that it has not to fulfil good coding practices
        \item If you ignore what clean code is, have a look \URL[PB]{https://itp.uni-frankfurt.de/\~strongmatter/seminar_slides_GU/2019.06.03.sciarra.pdf}{to this presentation}! \Remark{It is worth the read in any case!}
        \item Use meaningful names and strive for readability.
        \item If you do tricky stuff in your script, comment it!
        \item If you need to work on a larger script in Bash,
              \begin{itemize}
                  \item split your source code across files and
                  \item test your code, for real! \Remark{More on this in a moment}
              \end{itemize}
        \item A good/clean code is
              \begin{itemize}
                  \item Readable
                  \item Maintainable
                  \item Easy to extend
                  \item Easy to use
                  \item Hard to break
                  \item Testable/Tested
                  \item \ldots
              \end{itemize}
    \end{itemize}
    \begin{tikzpicture}[remember picture, overlay]
        \node[anchor=south east] at ($(current page.south east)+(-6mm,6mm)$) {\includegraphics[width=4cm]{Readability.jpg}};
    \end{tikzpicture}
\end{frame}















