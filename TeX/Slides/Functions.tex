%-------------------------------%
%  Author: Alessandro Sciarra   %
%    Date: 23 Aug 2019          %
%-------------------------------%

%~~~~~~~~~~~~~~~~~~~~~~~~~~~~~~~~~~~~~~~~~~~~%
\begin{frame}{Functions, finally! Overview}
    \vspace{-4mm}
    \begin{itemize}
        \item Functions are the last basic Bash feature we'll learn
              \begin{itemize}
                  \item They give you an incredible opportunity to structure and increase readability of your script
                  \item They can be used to split large script across multiple files in an elegant way
                  \item \alert{Learn them well, for real!}
              \end{itemize}
        \item \PB{Functions are a tricky world!}
        \item They have several features that we might call \textbf{issues} if compared to other languages
        \item[\PT{$\bullet$}] Indeed, functions in Bash are not as powerful as we might expect
              \begin{itemize}
                  \item[\PT{$\to$}] Return value
                  \item[\PT{$\to$}] Reusability
                  \item[\PT{$\to$}] Scope
                  \item[\PT{$\to$}] I/O + \ldots
              \end{itemize}
              \begin{varblock}{quote}[0.9\textwidth]{}[Greg's Wiki]
                  Don't bite the newbie for not understanding all this. Shell functions are totally f\texttt{***}ed. \\[-1.3ex] ~
              \end{varblock}
    \end{itemize}
    \vspace{-4mm}
    \begin{varblock}{example}[1.03\textwidth]{\textbf{Do not be scared!}}
        Learn, understand and use functions for what they are, not for what you would like them to be!
    \end{varblock}
\end{frame}
%~~~~~~~~~~~~~~~~~~~~~~~~~~~~~~~~~~~~~~~~~~~~%
\begin{frame}[fragile]{Basic syntax}
    \vspace{-2mm}
    \begin{lstlisting}[style=MyBash, numbers=none]
        # POSIX compliant syntax
        NAME () |+COMPOUND-COMMAND [ REDIRECTIONS ]+|
        # Totally equivalent syntax (in Bash), but not POSIX
        function NAME |+[()] COMPOUND-COMMAND [ REDIRECTIONS ]+|
    \end{lstlisting}
    \vspace{1mm}
    \begin{description}[X]
        \setlength{\itemsep}{3mm}
        \item[\textbf{NAME:}] ~\\
            A \textbf{sane} function name should be an alphanumeric string, maybe containing underscore and not starting with a number.
            However, insane names are accepted in Bash and, in principle, names like \PB{\texttt{:}} or \PB{\texttt{[\}\{}} are allowed.
            But please, avoid them! Really!! $\;${\tiny\URL[PB]{https://stackoverflow.com/a/44041384}{Exploring allowed names}}
        \item[\textbf{COMPOUND-COMMAND:}] ~\\
            The body of a function can be any compound command.
            \begin{itemize}
                \item \bash|{ list; }| \alert{$\;\longleftarrow\;$ Use this if you do not have a reason to use a different one}
                \item \bash|(list)| or \bash|((expression))| or \bash|[[ expression ]]|
            \end{itemize}
            %This is typically \bash|{ list; }|, but three other forms of compound commands are technically allowed: \bash|(list)|, \bash|((expression))|, and \bash|[[ expression ]]|.
        \item[\textbf{REDIRECTIONS:}] ~\\
            They take place when the function is called and they refer to the whole compound-command.
    \end{description}
\end{frame}
%~~~~~~~~~~~~~~~~~~~~~~~~~~~~~~~~~~~~~~~~~~~~%
\begin{frame}[fragile]{Functions: The most basic example}
    \vspace{-5mm}
    \begin{varblock}{example}[0.95\textwidth]{Just accomplish a \textbf{detached} task}
        Whenever a block of code can be executed as standalone, without needing either input information nor variables, it is straightforward to include it in a dedicated function
    \end{varblock}
    \begin{lstlisting}[style=MyBash, xrightmargin=1mm, xleftmargin=1mm]
        #!/bin/bash
        
        function CreateListOfFiles()
        {
            printf "#%19s %20s %15s %20s %20s %6s     %s\n" \
                   "user" "group" "permissions"               \
                   "size(KB)" "permissions" "type" "path"
            find "${PWD}" |+-printf+| "%20u %20g %15m %20k %20M %6y     %p\n"
        }
    \end{lstlisting}
    \begin{varblock}{}[0.9\textwidth]{Note}
        This will do absolutely nothing when run. This is because it has only been stored in memory, much like a variable, but it has not yet been called.
    \end{varblock}
    \begin{lstlisting}[style=MyBash, firstnumber=10, xrightmargin=1mm, xleftmargin=1mm]
        CreateListOfFiles
    \end{lstlisting}
\end{frame}
%~~~~~~~~~~~~~~~~~~~~~~~~~~~~~~~~~~~~~~~~~~~~%
\begin{frame}[fragile]{Variables in functions and their scopes}
    \vspace{-4mm}
    \begin{overlayarea}{\textwidth}{0.7\textheight}
        \begin{itemize}
            \item Variable in Bash are by definition \alert{\textbf{global}}!
            \item Variables declared using the \bash|local| builtin have a lifetime limited to the function scope
            \item Bash uses \textbf{dynamic scoping} to control a variable's visibility within functions
                  \begin{itemize}
                      \item In a function all variables visible in the caller are visible and might be changed
                      \item Declaring a local variable with the same name of an already existing variable shadows the variable from the caller, whose value cannot be retrieved from within the function.
                  \end{itemize}
            \item<only@3-> Unless you have a reason not to do so, declare all variables in functions as \bash|local|
            \item<only@3-> Do not assign a value to a local variable at declaration, because you might obfuscate/loose an exit code!
            \item<only@4-> \PP{If a variable at the current local scope is unset, it will remain so until it is reset in that scope or until the function returns.
                           Once the function returns, any instance of the variable at a previous scope will become visible.}\\
                           \textbf{However,} \PB{if the unset acts on a variable at a previous scope, any instance of a variable with that name that had been shadowed will become visible!}
        \end{itemize}
        \begin{onlyenv}<1>
            \begin{lstlisting}[style=MyBash]
                #!/bin/bash
                
                LevelOne() {
                    LevelTwo
                    echo "In ${FUNCNAME}, aVar = ${aVar}"
                }
                LevelTwo() {
                    local aVar; aVar="${FUNCNAME} local"
                    LevelThree
                }
                LevelThree() { echo "In ${FUNCNAME}, aVar = ${aVar}"; }
                
                aVar='global'; LevelOne
            \end{lstlisting}
        \end{onlyenv}
        \begin{onlyenv}<2>
            \begin{lstlisting}[style=MyBash]
                $ ./script.bash
                |+In LevelThree, aVar = LevelTwo local
                In LevelOne, aVar = global+|
            \end{lstlisting}
        \end{onlyenv}
        \begin{onlyenv}<3>
            \begin{lstlisting}[style=MyBash, style=oddnumbers, xleftmargin=1mm, xrightmargin=1mm]
                $ function Test { aVar="$(exit 1)"; echo $?; }; Test
                1  # Fine, but we pollute caller with 'aVar' variable
                $ function Test { local aVar="$(exit 1)"; echo $?; }; Test
                0  # <- local's exit code!
                $ function Test { local aVar; aVar="$(exit 1)"; echo $?; }; Test
                1  # GOOD CODE
            \end{lstlisting}
        \end{onlyenv}
    \end{overlayarea}
\end{frame}
%~~~~~~~~~~~~~~~~~~~~~~~~~~~~~~~~~~~~~~~~~~~~%
\begin{frame}{Passing arguments to functions}
    \vspace{-3mm}
    \begin{varblock}{example}[0.8\textwidth]{So far so good}
        Delegating tasks to functions, even if the task requires variables, is quite straightforward, provided to keep scope rules in mind
    \end{varblock}
    \begin{varblock}{quote}[0.95\textwidth]{The Pandora's box}[Bash manual]<2->
        When a function is executed, the arguments to the function become the positional parameters during its execution.
        The special parameter \texttt{\#} that expands to the number of positional parameters is updated to reflect the change.
        Special parameter 0 is unchanged.\\[-1.5ex] ~
    \end{varblock}
    \vspace{-2mm}
    \begin{itemize}[<3->]
        \item Arguments to functions are meant to provide \textbf{input} to a function
        \item Bash strictly uses \PP{call-by-value} semantics
        \item You can't pass arguments ``by reference''\\[-0.5ex] {\tiny\{~at least not until Bash 4.3 (and even there the \bash|declare -n| mechanism has serious security flaws)~\}}
        \item \alert{Passing the name of a variable to a function}, which should use it, requires \bash|eval| acrobatics and it \alert{should be as far as possible avoided!}
    \end{itemize}
\end{frame}
%~~~~~~~~~~~~~~~~~~~~~~~~~~~~~~~~~~~~~~~~~~~~%
\begin{frame}[fragile]{Passing arguments to functions: Example}
    \vspace{-2mm}
    \begin{onlyenv}<1>
        \begin{lstlisting}[style=MyBash]
            #!/bin/bash
            
            function SecondsToTimeStringWithDays()
            {
                local inputTime days hours minutes seconds
                inputTime=$1
                ((    days=(inputTime)/86400 ))
                ((   hours=(inputTime - days*86400)/3600 ))
                (( minutes=(inputTime - days*86400 - hours*3600)/60 ))
                (( seconds=inputTime%60 ))
                printf "%d-%02d:%02d:%02d\n" \
                       "${days}" "${hours}" "${minutes}" "${seconds}"
            }
            
            for example in 31536000 86400 3600; do
                SecondsToTimeStringWithDays ${example}
            done
        \end{lstlisting}
        \begin{lstlisting}[style=MyBash, aboveskip=2mm]
            $ ./scriptAbove
            |+365-00:00:00+|
            |+1-00:00:00+|     # What is bad in the function above?
            |+0-01:00:00+|
        \end{lstlisting}
    \end{onlyenv}
    \begin{onlyenv}<2>
        \begin{lstlisting}[style=MyBash]
            #!/bin/bash
            
            function SecondsToTimeStringWithDays() # Better code!
            {
                if [[ ! ${1} =~ ^[1-9][0-9]*$ ]]; then
                   echo "ERROR: Function ${FUNCNAME} wrongly called!"
                   return 1
                fi
                local inputTime days hours minutes seconds
                inputTime=$1
                ((    days=(inputTime)/86400 ))
                ((   hours=(inputTime - days*86400)/3600 ))
                (( minutes=(inputTime - days*86400 - hours*3600)/60 ))
                (( seconds=inputTime%60 ))
                printf "%d-%02d:%02d:%02d\n" \
                       "${days}" "${hours}" "${minutes}" "${seconds}"
            }
            
            for example in 31536000 86400 3600 ''; do
                SecondsToTimeStringWithDays ${example}
            done
        \end{lstlisting}
    \end{onlyenv}
\end{frame}
%~~~~~~~~~~~~~~~~~~~~~~~~~~~~~~~~~~~~~~~~~~~~%
\begin{frame}[fragile]{The \bash|return| builtin and functions' exit code}
    \vspace{-3mm}
    \begin{varblock}{quote}[0.6\textwidth]{Functions' exit code}
        \textnormal{When executed, the exit status of a function is the exit status of the \textbf{last command executed} in the body.}
    \end{varblock}
    \begin{lstlisting}[style=MyBash, numbers=none]
        $ help return
        |+return: return [n]+|           # ATTENTION: 0 <= N <=255
        |+Return from a shell function.

        Causes a function or sourced script to exit with the
        return value specified by N. If N is omitted, the
        return status is that of the last command executed
        within the function or script.

        Exit Status:
        Returns N, or failure if the shell is not executing
        a function or script.+|
    \end{lstlisting}
    \medskip
    \begin{varblock}{alert}[0.7\textwidth]{}
        \large \alert{Do not use \bash|return| to retrieve the function result!}
    \end{varblock}
\end{frame}
%~~~~~~~~~~~~~~~~~~~~~~~~~~~~~~~~~~~~~~~~~~~~%
\begin{frame}[fragile]{How do I return something from a function?}
    \vspace{-18.2mm}
    \begin{overlayarea}{\textwidth}{0.7\textheight}
        \begin{columns}[c]
            \begin{column}{0.55\textwidth}
            \end{column}
            \begin{column}{0.35\textwidth}
                \begin{varblock}{alert}[\textwidth]{}<2->
                    \Large It's simple: \alert{You don't!}
                \end{varblock}
            \end{column}
        \end{columns}
        \smallskip
        \setbeamertemplate{itemize/enumerate subbody begin}{\scriptsize}
        \begin{enumerate}
            \item<only@3-> Capturing standard output
                           \begin{itemize}[<only@{3,7}>]
                               \item[\PT{\ding{55}}] the function is executed in a subshell, i.e. variable assignments not seen in the caller
                               \item[\PT{\ding{55}}] \textbf{everything} printed by the function is captured (debug output!? $\to$ use stderr)
                               \item[\PS{\ding{51}}] good solution if performance is not critical
                           \end{itemize}
            \item<only@4-> Global variables
                           \begin{itemize}[<only@{4,7}>]
                               \item[\PS{\ding{51}}] the function is \textbf{not} executed in a subshell $\to$ potentially \PS{much faster}
                               \item[\PT{\ding{55}}] if the function is executed in a subshell, the method fails!
                               \item[\PT{\ding{55}}] the function cannot be used in a pipeline (as in any implicit subshell)
                           \end{itemize}
            \item<only@5-> Writing to a file
                           \begin{itemize}[<only@{5,7}>]
                               \item[\PT{\ding{55}}] you need to manage a temporary file, which is always inconvenient
                               \item[\PT{\ding{55}}] there must be a writable directory somewhere and sufficient space to hold the data therein
                               \item[\PS{\ding{51}}] it will work regardless of whether your function is executed in a subshell
                           \end{itemize}
            \item<only@6-> Dynamically scoped variables
                           \begin{itemize}
                               \item[\PT{\ding{55}}] if the function is executed in a subshell, the method fails!
                               \item[\PT{\ding{55}}] this technique doesn't work with recursive functions
                               \item[\PS{\ding{51}}] global variable namespace isn't polluted by the function return variable
                           \end{itemize}
        \end{enumerate}
        \begin{onlyenv}<3>
            \begin{lstlisting}[style=MyBash, numbers=none]
                ExampleFunction() {
                    echo "running foo()..."  >&2
                    echo "this is my data"
                }
                aVar=$(ExampleFunction)
                echo "ExampleFunction returned '${aVar}'"
            \end{lstlisting}
        \end{onlyenv}
        \begin{onlyenv}<4>
            \begin{lstlisting}[style=MyBash, numbers=none]
                ExampleFunction() {
                    globalVar="this is my data"
                }
                ExampleFunction
                echo "ExampleFunction returned '${globalVar}'"
            \end{lstlisting}
        \end{onlyenv}
        \begin{onlyenv}<5>
            \begin{lstlisting}[style=MyBash, numbers=none]
                ExampleFunction() {
                    echo "this is my data" > "$1"
                }
                # This is NOT solid code to handle temp files! @|\URL[comment-color]{http://mywiki.wooledge.org/BashFAQ/062}{Solid way}[background-color]|@
                tmpfile=$(mktemp)   # GNU/Linux
                ExampleFunction "${tmpfile}"
                echo "ExampleFunction returned '$(cat < "@|\tc{strings-color}{\$\{tmpfile\}}|@" )'"
                rm "${tmpfile}"
            \end{lstlisting}
            \FrameRemark[5]{If this were a real program, there would have been error checking, and a trap.}
        \end{onlyenv}
        \begin{onlyenv}<6>
            \begin{lstlisting}[style=MyBash, numbers=none, emph={[7]notGlobalVar},]
                ExampleFunction_implementation() {
                   notGlobalVar="this is my data"
                }
                ExampleFunction() {
                    local notGlobalVar; ExampleFunction_implementation
                }
                # Here at the global scope, 'notGlobalVar' is not visible
                ExampleFunction
                echo "ExampleFunction returned '${notGlobalVar}'"
            \end{lstlisting}
        \end{onlyenv}
        \vspace{-3mm}
        \begin{varblock}{}[0.7\textwidth]{Choose with care}<only@7>
            Use the approach you prefer depending on the situation!
        \end{varblock}
    \end{overlayarea}
\end{frame}








