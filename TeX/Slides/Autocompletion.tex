%-------------------------------%
%  Author: Alessandro Sciarra   %
%    Date: 25 Sep 2020          %
%-------------------------------%

%~~~~~~~~~~~~~~~~~~~~~~~~~~~~~~~~~~~~~~~~~~~~%
\begin{frame}[fragile]{Programmable completion}{A complex world plenty of opportunities: \enspace\URL[PB]{https://www.gnu.org/software/bash/manual/}{Bash manual v5.2 sections 8.6 and 8.7}}
    \vspace{-3mm}
    \begin{itemize}
        \item Whenever you press \texttt{TAB} while typing in your terminal,\\ \PS{you trigger a series of operations that the shell executes underneath}.
        \item This mechanism is actually programmable and it is usually available system-wide.
        \item It is nothing but\ldots\ a bash script to be sourced in your environment!
        \item It is usually automatically sourced somewhere, at university in the shell configuration file.
    \end{itemize}
    \begin{lstlisting}[style=myBash, numbers=none, aboveskip=2mm]
        # In a default ~/.bashrc file at some point:
        if [ -f /etc/bash_completion ]; then
            . /etc/bash_completion
        fi
    \end{lstlisting}
    \begin{lstlisting}[style=myBash, numbers=none, aboveskip=2mm, belowskip=-5mm]
        $ cat /etc/bash_completion
        |+. /usr/share/bash-completion/bash_completion+|
        $ wc -l /usr/share/bash-completion/bash_completion
        |+2171+|
    \end{lstlisting}
    \begin{itemize}
        \item The mechanism is triggered via the \bash|complete| builtin.
    \end{itemize}
\end{frame}
%~~~~~~~~~~~~~~~~~~~~~~~~~~~~~~~~~~~~~~~~~~~~%
\begin{frame}{The program completion ingredients}
    \vspace{-6mm}
    \begin{columns}
        \begin{column}{\dimexpr\paperwidth-18mm}
            \begin{onlyenv}<1>
                \begin{description}[\texttt{XXCOMP\_WORDS}]
                    \setlength{\itemsep}{2mm}
                    \item[\PB{\texttt{compgen}}]
                        Generate possible completion matches for word according to the options and write the matches to the standard output.
                    \item[\PB{\texttt{complete}}]
                        Specify how arguments should be completed.
                    \item[\PB{\texttt{compopt}}]
                        Modify completion options for each name according to the options, or for the currently-executing completion if no names are supplied.\\[6mm]
                    \item[\PB{\texttt{COMP\_WORDS}}]
                        An array variable consisting of the individual words in the current command line.
                    \item[\PB{\texttt{COMP\_CWORD}}]
                        An index into \PB{\texttt{\$\{COMP\_WORDS\}}} of the word containing the current cursor position.
                    \item[\PB{\texttt{COMPREPLY}}]
                        An array variable from which Bash reads the possible completions generated. Each array element contains one possible completion.
                \end{description}
            \end{onlyenv}
        \end{column}
    \end{columns}
    \PrepareURLsymbol[PB]
    \FrameRemark{There are further relevant variables --- \bash|COMP_WORDBREAKS|, \bash|COMP_LINE|, \bash|COMP_POINT|, etc. --- refer to the \URL*{https://www.gnu.org/software/bash/manual/}{Bash manual v5.2 sections 5.2}.}
\end{frame}
%~~~~~~~~~~~~~~~~~~~~~~~~~~~~~~~~~~~~~~~~~~~~%
\begin{frame}[fragile]{The standard pattern}
    \vspace{-3mm}
    \begin{enumerate}
        \item Make the script for which you want to implement autocompletion executable.
        \item Create an autocompletion script \textbf{to be sourced} in your environment.
        \item Associate a function to the command (your script's name) you want to be provided with autocompletion.
              \begin{lstlisting}[style=myBash, numbers=none, aboveskip=3mm, belowskip=-5mm, xrightmargin=15mm]
                  complete -F _script_completion  script
              \end{lstlisting}
              After having sourced the autocompletion script, this function will be automatically invoked by the shell when pressing \texttt{TAB} after the name of your script.
        \item Write the autocompletion function keeping in mind how it works.
    \end{enumerate}
    \begin{varblock}{alert}[0.9\textwidth]{A standard invocation}
        \begin{itemize}
            \item \PB{\texttt{\$1}} is the name of the command whose arguments are being completed;
            \item \PB{\texttt{\$2}} is the word being completed;
            \item \PB{\texttt{\$3}} is the word preceding the word being completed;
            \item The array variable \PB{\texttt{COMPREPLY}} contains possible completions.
        \end{itemize}
    \end{varblock}
\end{frame}
%~~~~~~~~~~~~~~~~~~~~~~~~~~~~~~~~~~~~~~~~~~~~%
\begin{frame}[fragile]{A basic example as starting point}
    \vspace{-1mm}
    \begin{onlyenv}<1-2,9->
        Suppose to have a \,\PS{\texttt{measure}}\, script that accepts the following options on the command line:
        \begin{itemize}
            \item \PB{\texttt{-{}-inputfile}} followed by a filename;
            \item a series of other options, e.g. \PB{\texttt{-{}-verbose}} and \PB{\texttt{-{}-conservative}}.
        \end{itemize}
        How do I implement autocompletion for it?
        \begin{onlyenv}<2>
            \begin{lstlisting}[style=myBash, numbers=none, aboveskip=4mm]
                $ emacs -nw ~/.measure-completion.bash
            \end{lstlisting}
        \end{onlyenv}
    \end{onlyenv}
    \begin{onlyenv}<3>
        \begin{lstlisting}[style=myBash, numbers=none, style=smaller]
            # You do not need a shebang here! This script will be sourced.
        \end{lstlisting}
    \end{onlyenv}
    \begin{onlyenv}<4>
        \begin{lstlisting}[style=myBash, numbers=none, style=smaller]
            # You do not need a shebang here! This script will be sourced.

            if [[ "${BASH_SOURCE[0]}" = "${0}" ]]; then
                printf "\n ERROR: File \"${BASH_SOURCE[0]}\" cannot be executed!\n\n"
                exit 1
            fi
        \end{lstlisting}
    \end{onlyenv}
    \begin{onlyenv}<5-6>
        \begin{lstlisting}[style=myBash, numbers=none, style=smaller]
            # You do not need a shebang here! This script will be sourced.

            if [[ "${BASH_SOURCE[0]}" = "${0}" ]]; then
                printf "\n ERROR: File \"${BASH_SOURCE[0]}\" cannot be executed!\n\n"
                exit 1
            fi

            function _measure_completion() @|\tikzmark{bad}|@
            {
                case "$3" in
                    --inputfile )
                        COMPREPLY=( $(compgen -f -- "$2") )
                        ;;
                    * )
                        COMPREPLY=(
                            $(compgen -W "--inputfile --verbose --conservative" -- "$2")
                        )
                        ;;
                esac
            }

            complete -F _measure_completion   measure
        \end{lstlisting}
    \end{onlyenv}
    \begin{onlyenv}<7-8>
            \begin{lstlisting}[style=myBash, numbers=none, style=smaller]
            # You do not need a shebang here! This script will be sourced.

            if [[ "${BASH_SOURCE[0]}" = "${0}" ]]; then
                printf "\n ERROR: File \"${BASH_SOURCE[0]}\" cannot be executed!\n\n"
                exit 1
            fi

            function _measure_completion() @|\tikzmark{stillbad}|@
            {
                case "$3" in
                    --inputfile )
                        COMPREPLY=( $(compgen -f -- "$2") )
                        ;;
                    * )
                        COMPREPLY=(
                            $(compgen -W "--inputfile --verbose --conservative" -- "$2")
                        )
                        ;;
                esac
            }

            #complete -F _measure_completion   measure
            complete -o filenames -F _measure_completion   measure
            #            @|\tc{red}{\then}|@ handles directories as such and not as files
        \end{lstlisting}
    \end{onlyenv}
    \begin{tikzpicture}[remember picture, overlay]
        \begin{scope}[every node/.style={anchor=north west, text=ALERT, font=\small}]
            \node[visible on=<6>] at ($(bad)+(5mm,3mm)$) {
                    What is bad in this implementation?
            };
            \node[visible on=<7>] (whystillbad) at ($(stillbad)+(5mm,3mm)$) {
                What is still bad in this implementation?
            };
        \end{scope}
        \node[visible on=<8>, font=\small, text=PB] at (whystillbad) {
            Valid options are proposed also if already given!
        };
    \end{tikzpicture}
    \begin{onlyenv}<9>
        \begin{lstlisting}[style=myBash, numbers=none, aboveskip=4mm]
            $ emacs -nw ~/.measure-completion.bash
            $ . ~/.measure-completion.bash
            $ ls
            file01.dat   file02.dat   file03.dat   @|\tc{RoyalBlue}{backup}|@
            $ ./measure <TAB>
            $ ./measure --<TAB-TAB>
            --conservative  --inputfile     --verbose
            $ ./measure --inputfile <TAB-TAB>
            backup/   file01.dat   file02.dat   file03.dat
            # There is room for improvement --> See today's Exercise 5
            $ ./measure --inputfile file02.dat <TAB-TAB>
            --conservative  --inputfile     --verbose
        \end{lstlisting}
    \end{onlyenv}
    \FrameRemark{Usually, you want to implement autocompletion for a script that you use very often. It is common to make it executable and remove the \,\texttt{.bash}\, extension.}
\end{frame}
%~~~~~~~~~~~~~~~~~~~~~~~~~~~~~~~~~~~~~~~~~~~~%
