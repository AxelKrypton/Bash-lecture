%-------------------------------%
%  Author: Alessandro Sciarra   %
%    Date: 25 Sep 2020          %
%-------------------------------%

%~~~~~~~~~~~~~~~~~~~~~~~~~~~~~~~~~~~~~~~~~~~~%
\begin{frame}[fragile]{Programmable completion}{A complex world plenty of opportunities: \enspace\URL[PB]{https://www.gnu.org/software/bash/manual/}{Bash manual v5.0 sections 8.6 and 8.7}}
    \vspace{-3mm}
    \begin{itemize}
        \item Whenever you press \texttt{TAB} while typing in your terminal,\\ \PS{you trigger a series of operations that the shell executes underneath}.
        \item This mechanism is actually programmable and it is usually available system-wide.
        \item It is nothing but\ldots\ a bash script to be sourced in your environment!
        \item It is usually automatically sourced somewhere, at university in the shell configuration file.
    \end{itemize}
    \begin{lstlisting}[style=myBash, numbers=none, aboveskip=2mm]
        # In a default ~/.bashrc file at some point:
        if [ -f /etc/bash_completion ]; then
            . /etc/bash_completion
        fi
    \end{lstlisting}
    \begin{lstlisting}[style=myBash, numbers=none, aboveskip=2mm, belowskip=-5mm]
        $ cat /etc/bash_completion
        |+. /usr/share/bash-completion/bash_completion+|
        $ wc -l /usr/share/bash-completion/bash_completion
        |+2171+|
    \end{lstlisting}
    \begin{itemize}
        \item The mechanism is triggered via the \bash|complete| builtin.
    \end{itemize}
\end{frame}
%~~~~~~~~~~~~~~~~~~~~~~~~~~~~~~~~~~~~~~~~~~~~%
\begin{frame}{The program completion ingredients}
    \vspace{-6mm}
    \begin{columns}
        \begin{column}{\dimexpr\paperwidth-18mm}
            \begin{onlyenv}<1>
                \begin{description}[\texttt{XXCOMP\_WORDS}]
                    \setlength{\itemsep}{2mm}
                    \item[\PB{\texttt{compgen}}]
                        Generate possible completion matches for word according to the options and write the matches to the standard output.
                    \item[\PB{\texttt{complete}}]
                        Specify how arguments should be completed.
                    \item[\PB{\texttt{compopt}}]
                        Modify completion options for each name according to the options, or for the currently-executing completion if no names are supplied.\\[6mm]
                    \item[\PB{\texttt{COMP\_WORDS}}]
                        An array variable consisting of the individual words in the current command line.
                    \item[\PB{\texttt{COMP\_CWORD}}]
                        An index into \PB{\texttt{\$\{COMP\_WORDS\}}} of the word containing the current cursor position.
                    \item[\PB{\texttt{COMPREPLY}}]
                        An array variable from which Bash reads the possible completions generated. Each array element contains one possible completion.
                \end{description}
            \end{onlyenv}
        \end{column}
    \end{columns}
    \FrameRemark{There are further relevant variables --- \bash|COMP_WORDBREAKS|, \bash|COMP_LINE|, \bash|COMP_POINT|, etc. --- refer to the \URL[PB]{https://www.gnu.org/software/bash/manual/}{Bash manual v5.0 sections 5.2}.}
\end{frame}
%~~~~~~~~~~~~~~~~~~~~~~~~~~~~~~~~~~~~~~~~~~~~%
\begin{frame}[fragile]{The standard pattern}
    \vspace{-3mm}
    \begin{enumerate}
        \item Make the script for which you want to implement autocompletion executable
        \item Create an autocompletion script \textbf{to be sourced} in your environment
        \item Associate a function to the command you want (your script's name) using.
              \begin{lstlisting}[style=myBash, numbers=none, aboveskip=3mm, belowskip=-5mm, xrightmargin=15mm]
                  complete -F _script_completion  script
              \end{lstlisting}
              After having sourced the autocompletion script, this function will be automatically invoked by the shell when pressing \texttt{TAB} after the name of your script.
        \item Write the autocompletion function keeping in mind how it works.
    \end{enumerate}
    \begin{varblock}{alert}[0.9\textwidth]{A standard invocation}
        \begin{itemize}
            \item \PB{\texttt{\$1}} is the name of the command whose arguments are being completed;
            \item \PB{\texttt{\$2}} is the word being completed;
            \item \PB{\texttt{\$3}} is the word preceding the word being completed;
            \item The array variable \PB{\texttt{COMPREPLY}} contains possible completions.
        \end{itemize}
    \end{varblock}
\end{frame}
%~~~~~~~~~~~~~~~~~~~~~~~~~~~~~~~~~~~~~~~~~~~~%
