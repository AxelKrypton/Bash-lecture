%-------------------------------%
%  Author: Alessandro Sciarra   %
%    Date: 25 Sep 2020          %
%-------------------------------%

%~~~~~~~~~~~~~~~~~~~~~~~~~~~~~~~~~~~~~~~~~~~~%
\begin{frame}{Compound commands}
    \begin{itemize}
        \item \only<2>{\pgfsetfillopacity{0.5}}\bash{if} statements\tikzmark{doneStart}
        \item \bash{for} loops
        \item \bash{while}, \bash{until} loops
        \item \bash{[[} keyword
        \item \bash{case}, \bash{select} constructs\tikzmark{doneEnd}\pgfsetfillopacity{1}\medskip
        \item \tc<2>{PP}{Subshells}\tikzmark{nowStart}
        \item \tc<2>{PP}{Command grouping}
        \item \tc<2>{PP}{Arithmetic evaluation}\tikzmark{nowEnd}\\[-0.5ex] \Remark{Slightly different from the \PQ{Arithmetic expansion} we already discussed}\medskip
        \item \tc<2>{PB}{[Functions]\tikzmark{fcts}} \Remark{Discussed in detail in a separate section}
    \end{itemize}
    \begin{tikzpicture}[remember picture, overlay]
        \begin{scope}[scope on=<2>]
            \draw[very thick, decorate, decoration={brace,amplitude=6pt}] (doneStart -| doneEnd) ++(8mm,1mm) -- ($(doneEnd)+(8mm,-1mm)$)
                      node[midway, right=3mm, text width=40mm, align=center] {Already discussed in detail};
            \draw[very thick, decorate, decoration={brace,amplitude=6pt}, PP] (nowStart -| nowEnd) ++(8mm,1mm) -- ($(nowEnd)+(8mm,-1mm)$)
                      node[midway, right=3mm, text width=40mm, align=center, text=PP] {What we are going to discuss};
            \path[from, PB] ($(fcts)-(6.8mm,3mm)$) edge[out=270, in=180] node[pos=1, right, font=\footnotesize, text=PB] {Strictly speaking not a compound command, but they work in a similar way} ++(2cm,-1cm);
        \end{scope}
    \end{tikzpicture}
\end{frame}
%~~~~~~~~~~~~~~~~~~~~~~~~~~~~~~~~~~~~~~~~~~~~%
\begin{frame}{Subshells}
    \vspace{-5mm}
    \begin{overlayarea}{\textwidth}{0.7\textheight}
        \begin{varblock}{}[0.8\textwidth]{Definition}
            A subshell is \textbf{a child process} but it is one \textbf{that inherits more than a normal external command} does.
            It can see all the variables of your script, not just the ones that have been exported to the environment.
        \end{varblock}
        \begin{varblock}{quote}[0.88\textwidth]{Unix process}<only@1>
            Every process on a Unix system has its own parcel of memory, for holding its variables, its file descriptors, its copy of the Environment inherited from its parent process, and so on.
            The changes to the variables (and other private information) in one process do not affect any other processes currently running on the system.
        \end{varblock}
        \begin{varblock}{alerted}[0.6\textwidth]{Use them consciously}<only@1>
            Forking a subshell leads to a speed penalty which often is irrelevant but which you should keep in mind!
        \end{varblock}
        \begin{itemize}[<2>]
            \item It is explicitly forced using the parenthesis \PB{\texttt{(\ldots)}}
            \item Changes e.g. to variables done in a subshell are not remembered when exiting the subshell: A subshell can be thought as a temporary shell!
            \item There are many instances in which a shell creates a subshell without parentheses being placed by the programmer
                  \begin{itemize}
                      \item \PP{In pipelines} $\;\longleftarrow\;$ \alert{\textbf{Every command in a pipeline is run in its own subshell!}}
                      \item \PP{In command substitution}
                      \item Executing other programs or scripts
                      \item In any external command that executes new shells (e.g. \bash{awk}, \bash{sed}, \bash{perl})
                      \item In process substitution
                      \item In backgrounded commands and coprocs
                  \end{itemize}
        \end{itemize}
    \end{overlayarea}
    \FrameRemark<2>{From Bash v4.2 the shell option \texttt{lastpipe} can be enabled so that the last command in a pipeline is not run in a subshell.}
\end{frame}
%~~~~~~~~~~~~~~~~~~~~~~~~~~~~~~~~~~~~~~~~~~~~%
\begin{frame}[fragile]{Subshells: Examples}
    \begin{lstlisting}[style=MyBash, aboveskip=0mm]
        # Changes in a subshell do not propagate back
        $ aVar="Hello"; pwd
        /home/sciarra/Documents
        $ ( aVar="Goodbye"; echo "${aVar}" ); echo "${aVar}"
        |+Goodbye+|
        |+Hello+|
        $ ( cd /tmp; pwd ); pwd
        |+/tmp
        /home/sciarra/Documents+|
        # It is often a feature to take advantage of!
        $ (cd /tmp || exit 1; tar ...)
        $ (source ~/AuxiliaryBashTools.bash; ...)
        # In a subshell the script variable are accessible
        $ ( echo "${aVar} from the subshell" )
        |+Hello from the subshell+|
        # Implicit subshells: be aware of them!
        $ echo "Goodbye" | read aVar; echo "${aVar}"
        |+Hello+|
        $ read aVar <<< "Goodbye"; echo "${aVar}"; unset aVar
        |+Goodbye+|
    \end{lstlisting}
    \PrepareURLsymbol[PB]
    \FrameRemark{\URL*{http://mywiki.wooledge.org/BashFAQ/024}{Advanced reading} about pipelines and subshells: Why can't I pipe data to \;\bash|read|?}
\end{frame}
%~~~~~~~~~~~~~~~~~~~~~~~~~~~~~~~~~~~~~~~~~~~~%
\begin{frame}[fragile]{Command grouping}{We already said something about it, let us go through once again}
    \vspace{-1mm}
    \begin{overlayarea}{\textwidth}{0.5\textheight}
        \begin{onlyenv}<1-4>
            \begin{itemize}
                \item Commands may be grouped together using curly braces \PB{\texttt{\{\ldots\}}}
                \item Command groups allow a collection of commands to be considered as a whole with regards to redirection and control flow
                \item All compound commands such as \bash|if| statements and \bash|while| loops do this as well, but command groups do \textbf{only} this
                \item Command groups are executed in the same shell as everything else, NOT in a new one!
            \end{itemize}
            \begin{lstlisting}[style=MyBash, numbers=none, aboveskip=2mm]
                $ { echo "$(date)"
                >   rsync -av |+.+| /backup; echo "$(date)";@|\tikzmark{sc}|@ } >backup.log 2>&1
            \end{lstlisting}
        \end{onlyenv}
        \begin{onlyenv}<5>
            \begin{lstlisting}[style=myBash, aboveskip=3mm]
                $ TIMEFORMAT='%3R seconds'
                $ time for((i=0; i<10000; i++)); do
                >   echo "Hello ${i}" >> FileInFor.dat
                > done
                |+11.451 seconds+|
                $ time for((i=0; i<10000; i++)); do
                >   echo "Hello ${i}"
                > done >> FileOutFor.dat
                |+0.054 seconds+|
                # The above for-loops do the same thing
                $ diff FileInFor.dat FileOutFor.dat; echo $?
                |+0+|
            \end{lstlisting}
        \end{onlyenv}
    \end{overlayarea}
    \vspace{-1mm}
    \begin{varblock}{quote}[0.9\textwidth]{}<only@4>
        The above example truncates and opens the file backup.log on stdout, then points stderr at where stdout is currently pointing (backup.log), then runs each command with those redirections applied.
        The file descriptors remain open until all commands within the command group complete before they are automatically closed.
        This means backup.log is only opened a single time, not opened and closed for each command.
    \end{varblock}
    \begin{varblock}{alert}[0.5\textwidth]{}<only@5>
        Take advantage of command grouping!
    \end{varblock}
    \begin{tikzpicture}[remember picture, overlay]
        \begin{scope}[scope on=<2-3>]
            \path[from, PT] ($(sc)-(0.9mm,1.5mm)$) edge[out=270, in=90] node[pos=1, below, font=\large, text=PT] (q) {Why is this semicolon absolutely \textbf{mandatory}?} ++(0cm,-1cm);
        \end{scope}
        \node[visible on=<3>, text width=5.6cm, align=center, below = 1mm of q, font=\scriptsize]
            {Because otherwise the closing curly bracket would be a argument of the final \mbox{command} in the group, \bash|echo| in this case};
    \end{tikzpicture}
\end{frame}
%~~~~~~~~~~~~~~~~~~~~~~~~~~~~~~~~~~~~~~~~~~~~%
\begin{frame}[fragile]{Arithmetic evaluation: The \bash|let| builtin}
    \vspace{-4mm}
    \begin{itemize}
        \item Sometimes we want to do arithmetic instead of string operations
        \item One way to do so is to use the \bash|let| builtin
              \begin{lstlisting}[style=MyBash, style=oddnumbers, aboveskip=2mm, belowskip=-6mm]
                  $ aVar=4+5; echo "${aVar}"
                  |+4+5+|
                  $ let aVar=4+5; echo "${aVar}"; unset aVar
                  |+9+|
              \end{lstlisting}
        \item However, it requires quotes to use arithmetic operators \Remark[1em]{\bash|help let|}
              \begin{lstlisting}[style=MyBash, style=oddnumbers, aboveskip=2mm, belowskip=-6mm, firstnumber=4]
                  $ let aVar=2<3
                  |+bash: 3: No such file or directory+|
                  $ let aVar='2<3'; echo "${aVar}"; unset aVar
                  |+1+|
              \end{lstlisting}
        \item \bash|let| accepts more than one expression
              \begin{lstlisting}[style=MyBash, style=oddnumbers, aboveskip=2mm, belowskip=-6mm, firstnumber=8]
                  $ let aVar='2<3' bVar=3*7; echo "${aVar} ${bVar}"
                  |+1 21+|
                  $ unset aVar bVar
              \end{lstlisting}
        \item If the last expression evaluates to 0, \bash|let| returns 1; \bash|let| returns 0 otherwise.
    \end{itemize}
\end{frame}
%~~~~~~~~~~~~~~~~~~~~~~~~~~~~~~~~~~~~~~~~~~~~%
\begin{frame}[fragile]{Arithmetic evaluation: The command grouping \PB{\texttt{((\ldots))}}}
    \vspace{-1mm}
    \begin{overlayarea}{\textwidth}{0.7\textheight}
        \begin{itemize}
            \item<only@1> \PB{\texttt{(( expression ))}} $\;$is  equivalent to$\;$ \bash|let "expression"|
            \item<only@1> No quote is needed in it, since only arithmetic operations are there performed
            \item<only@1> However, only one expression can be evaluated (not bad for the exit code)
                  \begin{lstlisting}[style=MyBash, style=oddnumbers, aboveskip=2mm, belowskip=-6mm]
                      $ (( aVar=7*3**2 )); echo "${aVar}"
                      |+63+|
                      $ (( aVar=1+${aVar}/20 )); echo "${aVar}"; unset aVar
                      |+4+|        # '${    }' not really needed@|$^\star$|@
                  \end{lstlisting}
            \item<only@1> Although not a compound command, the arithmetic substitution uses the same syntax
                  \begin{lstlisting}[style=MyBash, style=oddnumbers, aboveskip=2mm, belowskip=-6mm, firstnumber=4]
                      $ (( aVar=7*3**2 )); echo "${aVar}"
                      |+63+|
                      $ echo "aVar=$(( 7*3**2 ))"; unset aVar
                      |+63+|
                  \end{lstlisting}
            \item<only@1> Assignments in arithmetic substitution work but are confusing and should be avoided!
                  \begin{lstlisting}[style=MyBash, style=oddnumbers, aboveskip=2mm, belowskip=-6mm, firstnumber=8]
                      $ echo "_${bVar}_ $(( bVar=7*3**2 )) _${bVar}_"
                      |+__ 63 _63_+|
                  \end{lstlisting}
            \item<only@2> Arithmetic evaluation is very helpful in combination with conditionals
                  \begin{lstlisting}[style=MyBash, aboveskip=2mm, belowskip=-6mm, firstnumber=11]
                      $ aVar=$(( (5+2)*3 ))
                      $ if ((aVar == 21)); then
                      >   echo 'Blackjack!'
                      > fi
                      |+Blackjack!+|
                  \end{lstlisting}
            \item<only@2> Because the inside of \PB{\texttt{((\ldots))}} is C-like, a variable (or expression) that \PQ{evaluates to zero} will be considered \PQ{false} for the purposes of the arithmetic evaluation.
                  Then, because the evaluation is false, it will \PQ{exit with a status of 1}.
                  Likewise, if the expression inside \PB{\texttt{((\ldots))}} \PS{is non-zero}, it will be considered \PS{true}; and since the evaluation is true, it will \PS{exit with status 0}.
                  \begin{lstlisting}[style=MyBash, aboveskip=2mm, belowskip=-6mm, firstnumber=16]
                      $ flag=0  # no error
                      $ while read line; do
                      >   if [[ ${line} == *err* ]]; then flag=1; fi
                      > done < inputfile
                      $ if ((flag)); then echo 'Houston, we have a problem!'; fi
                  \end{lstlisting}
        \end{itemize}
    \end{overlayarea}
    \FrameRemark<1>{$^\star$Using the \texttt{\$\{\}} makes Bash use \bash|''| for uninitialised variables and might trigger errors (without \texttt{\$\{\}} syntax, \texttt{0} is used for uninitialised referenced variables).}
\end{frame}
%~~~~~~~~~~~~~~~~~~~~~~~~~~~~~~~~~~~~~~~~~~~~%

