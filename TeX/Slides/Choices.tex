%-------------------------------%
%  Author: Alessandro Sciarra   %
%    Date: 23 Sep 2020          %
%-------------------------------%

%~~~~~~~~~~~~~~~~~~~~~~~~~~~~~~~~~~~~~~~~~~~~%
\begin{frame}[fragile]{The \,\bash|case|\, keyword}
    \vspace{-3mm}
    Completely equivalent to a \bash|if-elif-else-fi| construct, but often handy:
    \medskip
    \begin{lstlisting}[style=MyBash, numbers=none, belowskip=-5mm]
        case WORD in
            |+[ [(] pattern [| pattern]...) command-list ;;]...+|
        esac
    \end{lstlisting}
    \vspace{2mm}
    \begin{overlayarea}{\textwidth}{\textheight}
        \begin{onlyenv}<1>
            \begin{itemize}
                \setlength{\itemsep}{0mm}
                \item The command-list corresponding to the first pattern that matches word is executed
                \item The match is performed according \PP{Pattern Matching} rules
                \item The \bash{|} is used to separate multiple patterns
                \item The \bash|)| operator terminates a pattern list
                \item A list of patterns and an associated command-list is known as a \textbf{clause}
            \end{itemize}
            \vspace{-3mm}
            \begin{center}
                \alert{\textbf{Before matching is attempted:}}\\[0.3em]
                \tikzmark{wordStart}tilde expansion\tikzmark{patternStart}\\
                parameter expansion\\
                command substitution\\
                arithmetic expansion\tikzmark{patternEnd}\\
                \tikzmark{wordEnd}quote removal
            \end{center}
        \end{onlyenv}
        \begin{tikzpicture}[remember picture, overlay, PP]
            \begin{scope}[scope on=<1>]
                \draw[very thick, decorate, decoration={brace,amplitude=6pt}] (patternStart -| patternEnd) ++(5mm,1mm) -- ($(patternEnd)+(5mm,-1mm)$)
                          node[midway, right=3mm, text width=18mm, align=center] {Pattern undergoes};
                \draw[very thick, decorate, decoration={brace,amplitude=6pt}] (wordEnd  -| wordStart) ++(-8mm,-1mm) -- ($(wordStart)+(-8mm,1mm)$)
                          node[midway, left=3mm, text width=18mm, align=center] {\texttt{WORD} undergoes};
            \end{scope}
        \end{tikzpicture}
        \begin{onlyenv}<2>
            \vspace{-3mm}
            \begin{lstlisting}[style=MyBash]
                case ${LANG} in
                    en* )
                        echo 'Hello!' ;;
                    fr* )
                        echo 'Salut!' ;;
                    de* )
                        echo 'Hallo!' ;;
                    it* )
                        echo 'Ciao!' ;;
                    es* )
                        echo 'Hola!' ;;
                    C | POSIX )
                        echo 'Hello World!' ;;
                    *)
                        echo 'I do not speak your language.' ;;
                esac
            \end{lstlisting}
        \end{onlyenv}
    \end{overlayarea}
    \FrameRemark{Using \bash|;&| or \bash|;;&| instead of \bash|;;| allows to execute the command list or to test the patterns of the next clause, respectively.}
\end{frame}
%~~~~~~~~~~~~~~~~~~~~~~~~~~~~~~~~~~~~~~~~~~~~%
\begin{frame}[fragile]{The \,\bash|select|\, keyword}
    \vspace{-3mm}
    A convenient statement for generating a menu of choices that the user can choose from:
    \medskip
    \begin{lstlisting}[style=MyBash, numbers=none, belowskip=-5mm]
        select NAME in WORDS; do
            # Body with commands -> break needed to terminate
        done
    \end{lstlisting}
    \vspace{1.5mm}
    \begin{overlayarea}{\textwidth}{\textheight}
        \begin{onlyenv}<1>
            \vspace{1.5mm}
            \begin{itemize}
                %\setlength{\itemsep}{0mm}
                \item The list of words following \;\bash|in|\; is expanded, generating a list of items
                \item The set of expanded words is printed on the \alert{standard error}, each preceded by a number
                \item If the \;\bash|in WORDS|\; is omitted, the positional parameters are printed, as if \;\bash{in "$@"}
                \item The \;\bash|PS3|\; prompt is then displayed and a line is read from the standard input:
                      \begin{itemize}
                          \item[$\circ$] If the line consists of a number corresponding to one of the displayed words, then the value of \;\bash|NAME|\; is set to that word
                          \item[$\circ$] If the line is empty, the words and prompt are displayed again
                          \item[$\circ$] Any other value read causes \;\bash|NAME|\; to be set to null
                          \item[$\circ$] The line read is saved in the variable \;\bash|REPLY|
                      \end{itemize}
                \item It is good practice not to \;\bash|break|\; if the user typed something meaningless!
            \end{itemize}
        \end{onlyenv}
        \begin{onlyenv}<2>
            \begin{lstlisting}[style=MyBash]
                $ select file in *; do
                >     echo "You picked \"${file}\" (${REPLY})"
                >     break
                > done
                |+1) Day_1.tex
                2) Day_2.tex
                3) Day_3.tex
                #? 2
                You picked "Day_2.tex" (2)+|
                $ echo "After select: \"${file}\" (${REPLY})"
                |+After select: "Day_2.tex" (2)+|
                $ unset file
            \end{lstlisting}
        \end{onlyenv}
        \begin{onlyenv}<3>
            \begin{lstlisting}[style=MyBash]
                $ select file in *; do
                >     echo "You picked \"${file}\" (${REPLY})"
                >     break;
                > done
                |+1) Day_1.tex
                2) Day_2.tex
                3) Day_3.tex
                #? dlfagfdsu
                You picked "" (dlfagfdsu)+|
                $ echo "After select: \"${file}\" (${REPLY})"
                |+After select: "" (dlfagfdsu)+|
                $ unset file
            \end{lstlisting}
            \medskip
            \begin{varblock}{alerted}[0.52\textwidth]{}
                It is always important to \alert{\textbf{validate the answer}}!
            \end{varblock}
        \end{onlyenv}
        \begin{onlyenv}<4>
            \begin{lstlisting}[style=MyBash]
                $ select file in *; do
                >     if [[ ${file} != '' ]]; then  # GOOD PRACTICE
                >         echo "You picked \"${file}\" (${REPLY})"
                >         break;
                >     fi
                > done; unset file
                |+1) Day_1.tex
                2) Day_2.tex
                3) Day_3.tex
                #? dlfagfdsu
                #? dgasdf
                #? 4
                #? 231345134
                #? 2
                You picked "Day_2.tex" (2)
            \end{lstlisting}
        \end{onlyenv}
    \end{overlayarea}
\end{frame}
%~~~~~~~~~~~~~~~~~~~~~~~~~~~~~~~~~~~~~~~~~~~~%




















