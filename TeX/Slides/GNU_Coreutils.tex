%-------------------------------%
%  Author: Alessandro Sciarra   %
%    Date: 24 Sep 2020          %
%-------------------------------%

%~~~~~~~~~~~~~~~~~~~~~~~~~~~~~~~~~~~~~~~~~~~~%
\begin{frame}[fragile]{They are not part of bash but they naturally interact with it}
    \vspace{-2mm}
    \begin{varblock}{example}[0.8\textwidth]{GNU coreutils}
        The GNU Core Utilities are the basic file, shell and text manipulation utilities of the GNU operating system.
        These are the core utilities which are expected to exist on every operating system.
    \end{varblock}
    \begin{overlayarea}{\textwidth}{0.6\textheight}
        \only<1-2>{\vspace{-3mm}}
        \begin{varblock}{example}[0.8\textwidth]{The util-linux package}<only@1-2>
            \textbf{util-linux} is a standard package distributed by the Linux Kernel Organization for use as part of the Linux operating system.
        \end{varblock}
        \PrepareURLsymbol[PB]
        \begin{varblock}{alert}[\textwidth]{What should I do here?!}<only@2>
            The \URL*{https://www.gnu.org/software/coreutils/manual/coreutils.pdf}{v9.4 GNU coreutils manual} from September 2023 has \alert{295 pages} and there are \alert{98 available commands}.
            There are more than \alert{100 commands} in the \textbf{util-linux} package.
        \end{varblock}
        \only<2>{\vspace{2mm}\centerline{\large\PQ{Don't panic. Discover, read the manuals and \textbf{learn when you need it}!}}}
        \begin{onlyenv}<3>
            \begin{lstlisting}[style=myBash, style=smaller, numbers=none, aboveskip=3mm]
                |+arch      cut        false     logname  od        runcon  tee       uptime
                base64    date       fmt       ls       paste     seq     test      users
                basename  dd         fold      md5sum   pathchk   shred   timeout   vdir
                cat       df         groups    mkdir    pinky     shuf    touch     wc
                chcon     dir        head      mkfifo   pr        sleep   tr        who
                chgrp     dircolors  hostid    mknod    printenv  sort    true      whoami
                chmod     dirname    hostname  mktemp   printf    split   truncate  yes
                chown     du         id        mv       ptx       stat    tsort
                chroot    echo       install   nice     pwd       stdbuf  tty
                cksum     env        join      nl       readlink  stty    uname
                comm      expand     kill      nohup    realpath  sum     unexpand
                cp        expr       link      nproc    rm        tac     uniq
                csplit    factor     ln        numfmt   rmdir     tail    unlink+|
            \end{lstlisting}
        \end{onlyenv}
        \begin{onlyenv}<4>
            \begin{lstlisting}[style=myBash, style=smaller, numbers=none, aboveskip=3mm, belowskip=-8mm, deletekeywords={env, expr, nice, nohup}]
                arch      cut        false     logname  od        runcon  tee       uptime
                base64    date       fmt       ls       paste     seq     test      users
                basename  dd         fold      md5sum   pathchk   shred   timeout   vdir
                cat       df         groups    mkdir    pinky     shuf    touch     wc
                chcon     dir        head      mkfifo   pr        sleep   tr        who
                chgrp     dircolors  hostid    mknod    printenv  sort    true      whoami
                chmod     dirname    hostname  mktemp   printf    split   truncate  yes
                chown     du         id        mv       ptx       stat    tsort
                chroot    echo       install   nice     pwd       stdbuf  tty
                cksum     env        join      nl       readlink  stty    uname
                comm      expand     kill      nohup    realpath  sum     unexpand
                cp        expr       link      nproc    rm        tac     uniq
                csplit    factor     ln        numfmt   rmdir     tail    unlink
            \end{lstlisting}
            \uncover<4>{\centerline{\ssmall The highlighted \tc{external-color}{\texttt{commands}} are those that have been touched at least ones during the lecture (some more are discovered in the exercises).}}
        \end{onlyenv}
    \end{overlayarea}
    \FrameRemark{If you are interested in exploring how the GNU core utilities are designed, there is an amazing project $\,\to\,$ \URL*{http://www.maizure.org/projects/decoded-gnu-coreutils/}{Decoded: GNU coreutils}}
\end{frame}
%~~~~~~~~~~~~~~~~~~~~~~~~~~~~~~~~~~~~~~~~~~~~%
\begin{frame}{Some GNU core utilities}{\URL[PB]{https://catonmat.net/ftp/gnu-coreutils-cheat-sheet.pdf}{Cheat-sheet}}
    \vspace{-8mm}
    \begin{columns}
        \begin{column}{\dimexpr\paperwidth-10mm}
            \begin{onlyenv}<1>
                \begin{description}[XXX\texttt{unexpand}]
                    \setlength{\itemsep}{1mm}
                    \item[\PB{\texttt{basename}}]
                        Strip directory and, optionally, a suffix from filenames.
                    \item[\PB{\texttt{comm}}]
                        Compare two sorted files line by line.
                    \item[\PB{\texttt{dirname}}]
                        Print a filename with its last non-slash component and trailing slashes removed.
                    \item[\PB{\texttt{expand}}]
                        Convert tabs to spaces.
                    \item[\PB{\texttt{fmt}}]
                        Reformat each paragraph in the file(s), writing to standard output.
                    \item[\PB{\texttt{fold}}]
                        Wrap each input line to fit in specified width.
                    \item[\PB{\texttt{ln}}]
                        Make links between files.
                    \item[\PB{\texttt{nl}}]
                        Write each input file to standard output, with line numbers added.
                    \item[\PB{\texttt{realpath}}]
                        Print the resolved absolute file name.
                    \item[\PB{\texttt{tac}}]
                        Write each input file to standard output in a reversed order (last line first).
                    \item[\PB{\texttt{uniq}}]
                        Report, omit or count repeated lines in a file
                    \item[\PB{\texttt{unexpand}}]
                        Convert spaces to tabs.
                \end{description}
            \end{onlyenv}
        \end{column}
    \end{columns}
\end{frame}
%~~~~~~~~~~~~~~~~~~~~~~~~~~~~~~~~~~~~~~~~~~~~%
\begin{frame}{Some commands from the util-linux package}{The project is available \URL[PB]{https://github.com/karelzak/util-linux}{on GitHub} and a list of the commands on \URL[PB]{https://en.wikipedia.org/wiki/Util-linux}{Wikipedia}}
    \vspace{-1mm}
    \begin{itemize}
        \item This package is much more OS oriented.
        \item Just have a look to it if you are curious, otherwise the following will be enough!
    \end{itemize}
    \vspace{-1mm}
    \begin{varblock}{example}[\textwidth]{You can do better than that!}
        At the end of this lecture you will have learnt to implement even more flexible versions of the following commands.
        To know they exist might still be handy in some occasions, though.
    \end{varblock}
    \vspace{-5mm}
    \begin{columns}
        \begin{column}{\dimexpr\paperwidth-10mm}
            \small
            \begin{description}[xxxxxxxxxxx\texttt{failglob}]
                \setlength{\itemsep}{1mm}
                \item[\PB{\texttt{cal}}]
                    Display a calendar and the date of Easter.
                \item[\PB{\texttt{column}}]
                    Format the input into multiple columns.
                \item[\PB{\texttt{getopt}}]
                    Parse command options.
                \item[\PB{\texttt{rename}}]
                    Rename the specified files by replacing the first occurrence\\ of an expression in their name by a replacement.
            \end{description}
        \end{column}
    \end{columns}
    \PrepareURLsymbol[PB]
    \FrameRemark{There might be \URL*{https://bugs.debian.org/cgi-bin/bugreport.cgi?bug=961301}{an issue} on your OS, since the package \textbf{bsdmainutils} is providing column in Debian and \textbf{util-linux} will not ship column as long as \textbf{bsdmainutils} does.}
\end{frame}
%~~~~~~~~~~~~~~~~~~~~~~~~~~~~~~~~~~~~~~~~~~~~%