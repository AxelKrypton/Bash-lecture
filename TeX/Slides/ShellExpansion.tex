%-------------------------------%
%  Author: Alessandro Sciarra   %
%    Date: 25 Jun 2019          %
%-------------------------------%

%~~~~~~~~~~~~~~~~~~~~~~~~~~~~~~~~~~~~~~~~~~~~%
\begin{frame}{Several kind of expansions}{\URL[PB]{https://www.gnu.org/software/bash/manual/}{Bash manual v5.0 section 3.5}}
    Expansion is performed on the command line after it has been split into tokens:
    \begin{itemize}
        \item Brace expansion\tikzmark{ExpA}
        \item Tilde expansion\tikzmark{ExpB}
        \item Parameter and variable expansion
        \item Arithmetic expansion
        \item Process substitution \tc{Gray!80}{{\tiny~\{Not available if \URL[Gray]{https://www.gnu.org/software/bash/manual/html_node/Bash-POSIX-Mode.html}{Bash in POSIX Mode (item 30)}\}}}
        \item Command substitution\tikzmark{ExpC}
        \item Word splitting\tikzmark{ExpD}
        \item Filename expansion\tikzmark{ExpE}
    \end{itemize}
    \begin{varblock}{alerted}[0.85\textwidth]{Quote Removal}
        After the preceding expansions, all unquoted occurrences of the characters\\
        \alert{\texttt{\textbackslash}}, \alert{\texttt{'}}, and \alert{\texttt{"}}
        that did not result from one of the above expansions are removed.
    \end{varblock}
    \begin{tikzpicture}[remember picture, overlay]
        \coordinate (xPos) at ($(current page.north)!0.3!(current page.north east)$);
        \draw[very thick, decorate, decoration={brace,amplitude=6pt}] (ExpB -| xPos) ++(-8mm,1mm) -- ($(ExpC -| xPos)+(-8mm,-1mm)$)
              coordinate[midway] (second);
        \node[anchor=west] at (second -| xPos) {\MakeEnumerateBox{2}$\quad$\alert{At the same time!}};
        \draw[from, shorter={0mm}{3mm}] (ExpA) -- (xPos |- ExpA) node[anchor=west] {\MakeEnumerateBox{1}};
        \draw[from, shorter={0mm}{3mm}] (ExpD) -- (xPos |- ExpD) node[anchor=west] {\MakeEnumerateBox{3}};
        \draw[from, shorter={0mm}{3mm}] (ExpE) -- (xPos |- ExpE) node[anchor=west] {\MakeEnumerateBox{4}};
    \end{tikzpicture}
\end{frame}
%~~~~~~~~~~~~~~~~~~~~~~~~~~~~~~~~~~~~~~~~~~~~%
\begin{frame}[fragile]{Brace expansion}
    \vspace{-5mm}
    \begin{overlayarea}{\textwidth}{0.7\textheight}
        \begin{itemize}
            \item It comes in two forms:
                \begin{enumerate}
                    \item \bash|[preamble]{comma separated list}[postscript]|
                    \item \bash|[preamble]{X..Y[..increment]}[postscript]|
                \end{enumerate}
            \item<only@1> \bash|[preamble]| and \bash|[postscript]| are optional 
            \item<only@1> \bash|[preamble]| and \bash|[postscript]| might contain other brace expansions
            \item<only@1> Brace expansions can be nested and, if so, they work from the outside in
            \item<only@1> \bash|X| and \bash|Y| are either characters or numbers
            \item<only@1> \bash|increment| is an optional integer (if omitted, it is +1 or -1 as appropriate)
            \item<only@1> If \bash|X| and \bash|Y| are numbers, leading 0 are respected to force each term to have the same width
            \item<only@1> If the brace expansion syntax is not respected, then brace expansion is not performed!
        \end{itemize}
        \vspace{-3mm}
        \begin{varblock}{}[\textwidth]{Result of the expansion}<only@1>
            A space separated list of all combinations of \bash|[preamble]| and \bash|[postscript]| with the elements in the brace.
            In \MakeEnumerateBox{2}, the sequence in braces is at first completed going back to \MakeEnumerateBox{1}.
            Order is respected left to right. Multiple non-nested braces expand to all combinations.
        \end{varblock}
        \begin{onlyenv}<2>
            \begin{lstlisting}[style=MyBash, style=oddnumbers, aboveskip=3mm]
                $ echo {a,b,c}
                |+a b c+|
                $ echo {a,b,c}.tex
                |+a.tex b.tex c.tex+|
                $ echo image.{jpg,png,pdf}
                |+image.jpg image.png image.pdf+|
                $ echo {1..8}
                |+1 2 3 4 5 6 7 8+|
                $ echo {8..1}
                |+8 7 6 5 4 3 2 1+|
                $ echo {1..8..3}
                |+1 4 7+|
                $ echo {a..e}
                |+a b c d e+|
                $ echo {e..a..2}
                |+e c a+|
                $ echo file_{01..3}.pdf
                |+file_01.pdf file_02.pdf file_03.pdf+| #Note leading zeros!
            \end{lstlisting}
        \end{onlyenv}
        \begin{onlyenv}<3>
            \begin{lstlisting}[style=MyBash, style=oddnumbers, aboveskip=3mm, firstnumber=18]
                $ echo {A..C}{0..2}
                |+A0 A1 A2 B0 B1 B2 C0 C1 C2+|
                $ echo {A..C}@|\textvisiblespace|@{0..2} # Here two independent expansions!
                |+A B C 0 1 2+|
                $ echo {in,out}{go,com}ing
                |+ingoing incoming outgoing outcoming+|
                $ echo {{A,E,I,O,U},{0..9}}
                |+A E I O U 0 1 2 3 4 5 6 7 8 9+|
                $ echo {a..z..x}
                |+{a..z..x}+|  # No brace expansion!
                $ aVar=1; echo {$aVar..5}; unset aVar
                |+{1..5}+|     # No brace expansion!
                $ echo {b,1..5}
                |+b 1..5+|     # Not surprising, right?
                $ echo {b,{1..5}}
                |+b 1 2 3 4 5+|
                $ echo {b,{1..5}}} # Which of the last braces is kept?
                |+b} 1} 2} 3} 4} 5}+|
            \end{lstlisting}
        \end{onlyenv}
    \end{overlayarea}
\end{frame}
%~~~~~~~~~~~~~~~~~~~~~~~~~~~~~~~~~~~~~~~~~~~~%
\begin{frame}{Parameter expansion}
    
\end{frame}
%~~~~~~~~~~~~~~~~~~~~~~~~~~~~~~~~~~~~~~~~~~~~%
\begin{frame}{Arithmetic expansion}
    
\end{frame}
%~~~~~~~~~~~~~~~~~~~~~~~~~~~~~~~~~~~~~~~~~~~~%
\begin{frame}{Word splitting}
    
\end{frame}
%~~~~~~~~~~~~~~~~~~~~~~~~~~~~~~~~~~~~~~~~~~~~%
\begin{frame}{Other expansions}
\vspace{-3mm}
    \begin{columns}
        \begin{column}{\dimexpr\paperwidth-10mm}
            \begin{description}[Command substitution:]
                \setlength{\itemsep}{3mm}
                \item[Tilde expansion:]
                    Easy to understand $\to$ \URL[PB]{https://www.gnu.org/software/bash/manual/}{Bash manual v5.0 section 3.5.2}
                \item[Process substitution:]
                    Useful, but not discussed here $\to$ \URL[PB]{https://www.gnu.org/software/bash/manual/}{Bash manual v5.0 section 3.5.6}\\
                    It takes the form \alert{\texttt{<(\ldots)}} or \alert{\texttt{>(\ldots)}}
                \item[Command substitution:]
                    It occurs when a command is enclosed as \alert{\texttt{\$(\ldots)}}\\
                    Do not use the deprecated syntax \tc{Gray!30}{\texttt{\textasciigrave\ldots\textasciigrave}}
                \item[Filename expansion:] Discussed in a separate section about globbing
            \end{description}
        \end{column}
    \end{columns}
    \vspace{3mm}
    \begin{varblock}{example}[0.8\textwidth]{Try to read about them alone}
        The Bash manual is a very good source, you can start reading there!\\
        \URL[PS]{http://mywiki.wooledge.org/BashGuide/InputAndOutput\#Process\_Substitution}{More on process substitution}
    \end{varblock}
\end{frame}
%~~~~~~~~~~~~~~~~~~~~~~~~~~~~~~~~~~~~~~~~~~~~%
