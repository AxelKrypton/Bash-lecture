%-------------------------------%
%  Author: Alessandro Sciarra   %
%    Date: 23 Sep 2020          %
%-------------------------------%

%~~~~~~~~~~~~~~~~~~~~~~~~~~~~~~~~~~~~~~~~~~~~%
\begin{frame}{The Special Parameters}{\URL[PB]{https://www.gnu.org/software/bash/manual/}{Bash manual v5.0 section 3.4.2}}
    \vspace{-8mm}
    \begin{columns}
        \begin{column}{\dimexpr\paperwidth-10mm}
            \begin{description}[\texttt{1 2 \ldots}]
                \setlength{\itemsep}{2pt}
                \item[\texttt{0}]
                    Contains the name, or the path, of the script.
                    Do not rely on it or read the manual carefully!
                \item[\texttt{1 2 \ldots}]
                    \alert{Positional Parameters} contain the arguments passed
                    to the current script or function
                \item[\texttt{*}]
                    Expands to all the words of all the positional parameters\tikzmark{curlyStart}\\
                    Double quoted, it expands to a single string containing them all,\\ separated by the first character of the \bash|IFS| variable
                \item[\texttt{@}]
                    Expands to all the words of all the positional parameters\\
                    Double quoted, it expands to a list of them all as individual words
                \item[\texttt{\#}]
                    Expands to the number of positional parameters that are currently set\tikzmark{curlyEnd}
                \item[\texttt{-}]
                    Expands to the current shell option flags
                \item[\texttt{?}]
                    Expands to the exit code of the most recently completed foreground command
                \item[\texttt{\$}]
                    Expands to the PID (process ID number) of the current shell
                \item[\texttt{!}]
                    Expands to the PID of the command most recently executed in the background
                \item[\texttt{\_}]
                    Expands to the last argument of the last command that was executed
            \end{description}
        \end{column}
    \end{columns}
    \begin{tikzpicture}[remember picture, overlay]
        \begin{scope}[scope on=<2>, PP]
            \draw[very thick, decorate, decoration={brace,amplitude=6pt}] (curlyStart -| curlyEnd) ++(5mm,1mm) -- ($(curlyEnd)+(5mm,-1mm)$) 
                  node[midway, right=3mm, text width=10mm, align=center] {Exercise Sheet 1};
        \end{scope}
    \end{tikzpicture}
\end{frame}
%~~~~~~~~~~~~~~~~~~~~~~~~~~~~~~~~~~~~~~~~~~~~%
