%-------------------------------%
%  Author: Alessandro Sciarra   %
%    Date: 23 Sep 2020          %
%-------------------------------%

%~~~~~~~~~~~~~~~~~~~~~~~~~~~~~~~~~~~~~~~~~~~~%
\begin{frame}{Patterns in Bash}
    \begin{varblock}{}[0.7\textwidth]{Definition}
        A pattern is a string with a special format designed to match filenames, or to check, classify or validate data strings.
    \end{varblock}
    \medskip
    Bash offers three different kinds of \PP{pattern matching}:\\[0.5em]
    \begin{enumerate}
        \item Globs \Remark{Used also in patterns of Parameter Expansion}\tikzmark{curlyStart}
        \item Extended Globs\tikzmark{curlyEnd}
        \item \tc<2>{Gray!50}{Regular\tikzmark{REleft} expressions \Remark{Since Bash v3.0}}\tikzmark{RE}
    \end{enumerate}
    \begin{tikzpicture}[remember picture, overlay]
        \draw[very thick, decorate, decoration={brace,amplitude=3pt}] (curlyStart) ++(8mm,1mm) -- ($(curlyEnd -| curlyStart)+(8mm,-1mm)$)
              node[midway, right=3mm, text width=45mm, align=center, text=PB] (FE) {Pattern matching done in Filename expansion};
        \node (RElabel) at (RE -| FE) {{\color{PT}\scriptsize\{~Cannot be used to select filenames!~\}}};
        \draw[from, shorter={1mm}{1mm}] (RE) -- (RElabel.west);
        \begin{scope}[scope on=<2>]
            \path[from, shorter={1mm}{1mm}] (REleft) edge[out=270, in=180] node[pos=1, anchor=west] {We will discuss this tomorrow with conditional blocks} ++(1cm, -15mm);
        \end{scope}
    \end{tikzpicture}
\end{frame}
%~~~~~~~~~~~~~~~~~~~~~~~~~~~~~~~~~~~~~~~~~~~~%
\begin{frame}{Filename expansion}{Glob patterns}
    \vspace{-3mm}
    \begin{varblock}{}[0.9\textwidth]{How does it work?}
        If a character among \PB{\texttt{*}}, \PB{\texttt{?}}, \PB{\texttt{[}} appears, then the word is regarded as a pattern, and replaced with an alphabetically sorted list of filenames matching the pattern
    \end{varblock}
    \begin{itemize}
        \item Metacharacters which might be used in glob patterns and undergo Filename Expansion:
              \begin{description}
                  \item[\texttt{*}]
                      Matches any string, including the null string
                  \item[\texttt{?}]
                      Matches any single character
                  \item[\texttt{[\ldots]}]
                      Matches any one of the enclosed characters
              \end{description}
        \item Globs are implicitly \textbf{anchored at both ends}, i.e.\ \alert{a glob must match \textbf{a whole string}}
        \item \PP{When matching filenames, the \texttt{*} and \texttt{?} characters cannot match a slash (\texttt{/})}
        \item When matching patterns, the \texttt{/} restriction is removed
    \end{itemize}
    \PrepareURLsymbol[PB]
    \FrameRemark{Filename expansion can be deactivated: $\;$\URL*{https://www.gnu.org/software/bash/manual/html_node/The-Set-Builtin.html}{\texttt{set -f}}}
\end{frame}
%~~~~~~~~~~~~~~~~~~~~~~~~~~~~~~~~~~~~~~~~~~~~%
\begin{frame}[fragile]{The \texttt{[\ldots]} pattern}
    \vspace{-1mm}
    \begin{itemize}
        \item There are some special rules which might be interesting to know
        \item A pair of characters separated by a hyphen denotes a range expression\\
              $\to\;$ any character that falls between those two characters, inclusive, is matched
        \item If the first character following the \PB{\texttt{[}} is a \PP{\texttt{!}} or a \PP{\texttt{\^{}}} then any character not enclosed is matched
        \item A \PP{\texttt{-}} may be matched by including it as the first or last character in the set
        \item A \PP{\texttt{]}} may be matched by including it as the first character in the set
        \item Within \PB{\texttt{[\ldots]}}, character classes can be specified using the syntax \PB{\texttt{[:class:]}},\\
              where \texttt{class} is one of the \URL[PB]{http://tldp.org/LDP/GNU-Linux-Tools-Summary/html/x11655.htm}{character classes} defined in the POSIX standard:
              \begin{lstlisting}[style=MyBash, numbers=none, xleftmargin=2mm, xrightmargin=15mm, aboveskip=3mm, belowskip=-5mm]
                  |+alnum   ascii   cntrl   graph   print   space   word+|
                  |+alpha   blank   digit   lower   punct   upper   xdigit+|
              \end{lstlisting}
              For example, \PB{\texttt{[[:digit:]]}} or \PB{\texttt{[[:blank:]]}}
    \end{itemize}
    \FrameRemark{The sorting order of characters in range expressions is determined by the current locale and the values of the \bash|LC_COLLATE| and \bash|LC_ALL| shell variables, if set.}
\end{frame}
%~~~~~~~~~~~~~~~~~~~~~~~~~~~~~~~~~~~~~~~~~~~~%
\begin{frame}[fragile]{Globs examples}
    \vspace{-2mm}
    \begin{onlyenv}<1>
        \begin{lstlisting}[style=MyBash, style=oddnumbers, xleftmargin=3mm, xrightmargin=3mm]
            $ ls
            |+Bash.pdf   Day_1.tex   Day_2.tex   Day_3.tex   Day_extra.tex+|
            $ echo *
            |+Bash.pdf Day_1.tex Day_2.tex Day_3.tex Day_extra.tex+|
            $ echo Day*
            |+Day_1.tex Day_2.tex Day_3.tex Day_extra.tex+|
            $ echo *.pdf
            |+Bash.pdf+|
            $ echo *_?.tex
            |+Day_1.tex Day_2.tex Day_3.tex+|
            $ echo *_[12]*
            |+Day_1.tex Day_2.tex+|
            $  echo *_[0-9]*
            |+Day_1.tex Day_2.tex Day_3.tex+|
            $ echo *_[[:digit:]]*
            |+Day_1.tex Day_2.tex Day_3.tex+|
            $ echo '*'  # As already said, quotes prevent globbing
            |+*+|
            $ echo "*_[[:digit:]]*"
            |+*_[[:digit:]]*+|
        \end{lstlisting}
    \end{onlyenv}
    \begin{onlyenv}<2>
        \begin{lstlisting}[style=MyBash, style=oddnumbers, xleftmargin=3mm, xrightmargin=3mm, firstnumber=20]
            $ aVar='Day3_Day2_Day1'; echo "${aVar#*[12]}"; unset aVar
            |+_Day1+|
            $ aVar='Day3_Day2_Day1'; echo "${aVar%%[123]*}"; unset aVar
            |+Day+|
            $ echo /*/bin   # For filenames, * does not match /
            |+/usr/bin+|
            $ echo /*/*/bin
            |+/usr/local/bin+|
            $ aVar='/home/sciarra/Documents'; echo "${aVar##*/}"
            |+Documents+|       # For strings, it does!
            $ echo "${aVar%/*}"; unset aVar
            |+/home/sciarra+|
            # This different behaviour is quite natural, indeed
        \end{lstlisting}
    \end{onlyenv}
    \begin{onlyenv}<3>
        \begin{lstlisting}[style=MyBash, style=oddnumbers, xleftmargin=3mm, xrightmargin=3mm, firstnumber=32]
            # Bash performs Filename Expansions AFTER Word Splitting
            #  => filenames generated by a glob will not be split!
            $ touch "a b.txt"; ls -1
            |+a b.txt+|
            $ rm *; ls # Here * is expanded into "a b.txt"
                       #  => rm run correctly with one argument only!
            $ touch "a b.txt"; rm $(ls) # WRONG!
            |+rm: cannot remove 'a': No such file or directory+|
            |+rm: cannot remove 'b.txt': No such file or directory+|
            # Please, do not be tempted by using quotes here...
            $ rm "$(ls)" # AAAARGHH! It works, but try this:
            $ touch "a b.txt" c.txt; rm "$(ls)"
            |+rm: cannot remove 'a b.txt'$'\n''c.txt': No such file or dir+|
        \end{lstlisting}
    \end{onlyenv}
    \vspace{2mm}
    \begin{varblock}{quote}[0.9\textwidth]{Good practice:}[Greg's Wiki]<only@2-3>
        You should always use globs instead of \textnormal{\bash|ls|} (or similar) to enumerate files.
        Globs will always expand safely and minimise the risk for bugs.
        You can sometimes end up with some very weird filenames.
        Most scripts aren't tested against all the odd cases that they may end up being used with.
        Don't let your script be one of those!\\[-0.7em] ~
    \end{varblock}
\end{frame}
%~~~~~~~~~~~~~~~~~~~~~~~~~~~~~~~~~~~~~~~~~~~~%
\begin{frame}[fragile]{Extended Globs}
    \vspace{-5mm}
    \begin{varblock}{}[0.92\textwidth]{This feature is turned off by default (in scripts)}
        You can turn it on with the \bash|shopt| command, which is used to toggle \textbf{sh}ell \textbf{op}tions:
        \begin{lstlisting}[style=MyBash, numbers=none, belowskip=-5mm, aboveskip=2mm]
            $ shopt -s extglob
        \end{lstlisting}
    \end{varblock}
    \medskip
    In the following description, a pattern-list is a list of one or more patterns separated by a \texttt{|}:
    \begin{description}[\texttt{?(pattern-list)}xx]
        \item[\texttt{?(pattern-list)}] Matches zero or one occurrence of the given patterns
        \item[\texttt{*(pattern-list)}] Matches zero or more occurrences of the given patterns
        \item[\texttt{+(pattern-list)}] Matches one or more occurrences of the given patterns
        \item[\texttt{@(pattern-list)}] Matches one of the given patterns
        \item[\texttt{!(pattern-list)}] Matches anything except one of the given patterns
    \end{description}
    \medskip
    \begin{onlyenv}<2>
        \begin{lstlisting}[style=MyBash, style=oddnumbers]
            $ ls
            |+names.txt   tokyo.jpg   california.bmp+|
            $ echo |+!(*jpg|*bmp)+|
            |+names.txt+|
        \end{lstlisting}
    \end{onlyenv}
\end{frame}










