%-------------------------------%
%  Author: Alessandro Sciarra   %
%    Date: 23 Sep 2020          %
%-------------------------------%

%~~~~~~~~~~~~~~~~~~~~~~~~~~~~~~~~~~~~~~~~~~~~%
\begin{frame}[label={Quotes}]{Types of quoting}
    \vspace{-4mm}
    \begin{description}
        \item[Single quotes] Everything inside single quotes becomes a literal string\\[-0.3em]
                             \Remark{The only character that you can't safely enclose in single quotes is a single quote.}
        \item[Double quotes] Performed actions:
                             \begin{itemize}
                                 \item Every substitution that begins with a dollar sign \texttt{\$}
                                 \item Backslash escaping
                                 \item \tc{Gray!30}{The legacy \texttt{\textasciigrave\ldots\textasciigrave} (backticks) command substitution}
                             \end{itemize}
                             \alert{No word splitting or filename expansion is performed!}
        \item[Backticks] \tc{Gray!30}{\texttt{\textasciigrave\ldots\textasciigrave} is the legacy command substitution syntax}\\
                         \alert{Deprecated in favor of \texttt{\$(\ldots)}}
        \item[Backslash] Putting \texttt{\textbackslash} in front of a metacharacter removes its special meaning\\[-0.3em]
                         \Remark{This works inside double quotes, or in the absence of quotes. It does not work inside single quotes.}
        \item[\texttt{\$'\ldots'}] A Bash extension that prevents everything except backslash escaping\\
                                   It also allows special backslash escape sequences like \texttt{\textbackslash{}n} for newline%, \t for tab, and \xnn for bytes specified in hexadecimal. 
        \item[\texttt{\$"\ldots"}] Bash extension used for \URL[PP]{http://mywiki.wooledge.org/BashFAQ/098}{localisation support}, not covered here
    \end{description}
    \FrameRemark{This is the short story. The verbose version is available at the end of \URL[PB]{http://mywiki.wooledge.org/Quotes}{this page}.}
\end{frame}
%~~~~~~~~~~~~~~~~~~~~~~~~~~~~~~~~~~~~~~~~~~~~%
\begin{frame}[fragile]{Examples about quoting}
    \begin{onlyenv}<1>
        \begin{lstlisting}[style=MyBash, style=oddnumbers]
            $ echo I    am  Clark       Kent
            |+I am Clark Kent+|
            $ echo I"    "am"  "Clark"       "Kent        # Not nice
            |+I    am  Clark       Kent+|
            $ echo I\ \ \ \ am\ \ Clark\ \ \ \ \ \ \ Kent # Really!?
            |+I    am  Clark       Kent+|
            $ echo 'I    am  Clark       Kent'
            |+I    am  Clark       Kent+|
            $ echo "I    am  Clark       Kent"
            |+I    am  Clark       Kent+|
        \end{lstlisting}
        \begin{lstlisting}[style=MyBash, style=oddnumbers, firstnumber=10, aboveskip=12mm]
            $ echo 'PATH contains ${PATH}'
            |+PATH contains ${PATH}+|
            $ echo "PATH contains ${PATH}"
            |+PATH contains /home/sciarra/.local/bin:/usr/bin:/bin+|
            $ echo "PATH contains \${PATH}"
            |+PATH contains ${PATH}+|
        \end{lstlisting}
    \end{onlyenv}
    \begin{onlyenv}<2>
        \renewcommand\thelstnumber{\ifnum\value{lstnumber}=11\relax\else\arabic{lstnumber}\fi}
        \begin{lstlisting}[style=MyBash, style=oddnumbers, firstnumber=16]
            $ ls *.tex
            |+Day_1.tex   Day_2.tex   Day_3.tex+|
            $ ls '*.tex'
            |+ls: cannot access '*.tex': No such file or directory+|
            $ ls "*.tex"
            |+ls: cannot access '*.tex': No such file or directory+|
            $ echo "Hello\nWorld"
            Hello\nWorld
            $ echo $'Hello\nWorld'
            Hello
            @|\renewcommand\thelstnumber{}|@World
        \end{lstlisting}
    \end{onlyenv}
    \begin{varblock}{alerted}[0.9\textwidth]{I'm Too Lazy to Read, Just Tell Me What to Do}<2>
        \begin{lstlisting}[style=MyBash, numbers=none, belowskip=-6mm,aboveskip=2mm]
            $ cp $file $destination         # WRONG
            $ cp -- "$file" "$destination"  # Right
        \end{lstlisting}
        When in doubt, \textbf{double-quote every expansion} in your shell commands.\\
        Generally use \textbf{double quotes} unless it makes more sense to use single quotes.
    \end{varblock}
\end{frame}
%~~~~~~~~~~~~~~~~~~~~~~~~~~~~~~~~~~~~~~~~~~~~%
\begin{frame}[fragile]{Special characters}{Just a brief overview}
    \vspace{-6mm}
    \begin{columns}
        \begin{column}{\dimexpr\paperwidth-10mm}
            \begin{description}[\texttt{>  >>  <}]
                \setlength{\itemsep}{1mm}
                \item<only@1>[\texttt{\textvisiblespace}]
                    A \alert{white}\tikzmark{whitespace}\alert{space} is used by Bash to determine where words begin and end \\
                    The first word is the command name and additional words become arguments
                \item<only@1>[\texttt{\$}]
                    It introduces various types of \alert{expansion}:
                    \begin{tabular}{ll}
                        \PB{$\bullet\;$} parameter expansion   & \texttt{\$\{var\}}          \\
                        \PB{$\bullet\;$} command substitution  & \texttt{\$(command)}      \\
                        \PB{$\bullet\;$} arithmetic expansion  & \texttt{\$((expression))} \\
                    \end{tabular}
                \item<only@1>[\texttt{' '}] \alert{Single quotes}\tikzmark{curlyStart}
                \item<only@1>[\texttt{" "}] \alert{Double quotes}
                \item<only@1>[\texttt{\textbackslash}] \alert{Escape symbol}\tikzmark{curlyEnd}
                \item<only@1>[\texttt{\#}]
                    It introduced a \alert{comment} that extends to the end of the line. \\
                    Text after it is ignored by the shell. \Remark{In single and double quotes, \texttt{\#} is just a \texttt{\#}.}
                \item<only@2>[\texttt{=}]
                    The \alert{assignment} symbol is used to assign a value to a variable \\
                    Whitespace is not allowed on either side of the \texttt{=} character
                \item<only@2>[\texttt{[[ ]]}]
                    This \alert{testing keyword} allows to evaluate a conditional expression \\
                    to determine whether it is ``true'' or ``false''
                \item<only@2>[\texttt{!}]
                    The \alert{negate keyword} reverses a test or an exit status
                \item<only@2>[\texttt{>  >>  <}]
                    \alert{Redirection} of a command's output or input to a file
                \item<only@2>[\texttt{|}]
                    The \alert{pipeline} sends the output from one command to the input of another command
                \item<only@2>[\texttt{;}]
                    \alert{Command separator} of multiple commands that are on the same line 
                \item<only@2>[\texttt{\{ \}}]
                    An \alert{inline group} allows to treat multiple commands as if they were one command
                    Convenient to be uses when Bash syntax requires only one command
                \item<only@2>[\texttt{( )}]
                    Another way to group commands, but in a \alert{subshell group} \\
                    However, commands in \texttt{( )} are executed in a subshell (a new process) \\
                    and this is often the way to avoid side effects on the current shell
                \item<only@3-4>[\texttt{(( ))}]
                     Within an \alert{arithmetic expression}, mathematical oper\tikzmark{operators}ators are used for calculations \\
                     They can be used for\\
                     \begin{tabular}{ll}
                        \PB{$\bullet\;$} variable assignments   & \texttt{(( a = 1 + 4 ))}  \\
                        \PB{$\bullet\;$} tests evaluation       & \texttt{(( a < b ))}      \\
                    \end{tabular}
                \item<only@3-4>[\texttt{\$(( ))}]
                    Comparable to \texttt{(( ))}, but the expression is replaced with the result of its evaluation
                \item<only@3-4>[\texttt{*  ?}]
                    \alert{Globbing} characters are wildcards which match parts of filenames
                \item<only@3-4>[\texttt{\~}]
                    The tilde is a representation of a \alert{home directory}\\
                    When alone or followed by a \texttt{/}, it means the current user's home directory \\
                    Otherwise, a username must be specified: \bash|ls ~john/|
                \item<only@3-4>[\texttt{\&}]
                    When used at the end of a command, run the command in the \alert{background} \\
                    The shell does not wait for it to complete
            \end{description}
        \end{column}
    \end{columns}
    \begin{varblock}{}[0.9\textwidth]{More details later}<only@4>
        \PB{We will come back to these special instructions with details and examples} 
    \end{varblock}
    \begin{tikzpicture}[remember picture, overlay]
        \begin{scope}[scope on=<1>]
            \node[font=\ssmall, above = 4mm of whitespace, anchor=west, xshift=1cm] (ws) {\{a space, tab, newline, carriage return, vertical tab or form feed\}};
            \path[to] (ws.west) edge[out=180, in=90] ([yshift=2mm]whitespace);
            \draw[very thick, decorate, decoration={brace,amplitude=6pt}] (curlyStart -| curlyEnd) ++(5mm,1mm) -- ($(curlyEnd)+(5mm,-1mm)$) 
                  node[midway, right=3mm, text width=25mm, align=center] {Already discussed on slide~\ref{Quotes}};
        \end{scope}
        \begin{scope}[scope on=<3-4>]
            \node[font=\ssmall, above = 4mm of operators, anchor=west, xshift=1cm] (ops) {the 4 characters\,\,\fbox{\texttt{+~~~-~~~*~~~/}}};
            \path[to] (ops.west) edge[out=180, in=90] ([yshift=2mm]operators);
        \end{scope}
    \end{tikzpicture}
\end{frame}
%~~~~~~~~~~~~~~~~~~~~~~~~~~~~~~~~~~~~~~~~~~~~%
\begin{frame}{Deprecated special character}
    \begin{columns}
        \begin{column}{\dimexpr\paperwidth-10mm}
            \begin{description}[\texttt{>  >>  <}]
                \setlength{\itemsep}{1mm}
                \color{Gray!30}
                \item[\texttt{\textasciigrave\ \textasciigrave}]
                    Command substitution - \alert{use \texttt{\$( )} instead}\tikzmark{backticks}
                \item[\texttt{[ ]}]
                    An alias for the old test command\\
                    Commonly used in POSIX shell scripts\\
                    \alert{Lacks many features of \texttt{[[ ]]}}\tikzmark{test}
                \item[\texttt{\$[ ]}]
                    Arithmetic expression - \alert{use \texttt{\$(( ))} instead} \\
                    Simply do not use it, it dates back to 1990s!!!
            \end{description}
        \end{column}
    \end{columns}
    \bigskip
    \begin{varblock}{}[0.9\textwidth]{Legacy code and portability}
        Unless you have very special requirement, try to stick to modern Bash features!
    \end{varblock}
    \begin{tikzpicture}[remember picture, overlay, every node/.style={anchor=east, xshift=-1cm, font=\ssmall}]
        \node at (current page.east |- backticks) (A) {\URL[PB]{http://mywiki.wooledge.org/BashFAQ/082}{Advance reading}};
        \node at (current page.east |- test) (B) {\URL[PB]{http://mywiki.wooledge.org/BashFAQ/031}{Advanced reading}};
        \path[from, shorter={5mm}{5mm}, thin, Gray!30] (backticks) edge (A) (test) edge (B);
    \end{tikzpicture}
    \FrameRemark{At the end of the lecture, you might come back to \URL[PP]{https://wiki.bash-hackers.org/scripting/obsolete}{this page} and read about obsolete features!}
\end{frame}
%~~~~~~~~~~~~~~~~~~~~~~~~~~~~~~~~~~~~~~~~~~~~%
