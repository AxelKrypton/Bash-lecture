%-------------------------------%
%  Author: Alessandro Sciarra   %
%    Date: 23 Sep 2020          %
%-------------------------------%

%~~~~~~~~~~~~~~~~~~~~~~~~~~~~~~~~~~~~~~~~~~~~%
\begin{frame}{The Bash script flow}
    \begin{tikzpicture}[overlay, remember picture]
        \pgfmathsetmacro{\cubex}{1}
        \pgfmathsetmacro{\cubey}{1}
        \pgfmathsetmacro{\cubez}{1}
        \coordinate (O) at ($(current page.center)-(0.5,0.3)$);
        \draw[PP, fill=PP!40] (O) ++(\cubex,\cubey,0) -- ++(0,0,-\cubez) -- ++(0,-\cubey,0) -- ++(0,0,\cubez) -- cycle;
        \draw[PP, fill=PP!10] (O) ++(\cubex,\cubey,0) -- ++(-\cubex,0,0) -- ++(0,0,-\cubez) -- ++(\cubex,0,0) -- cycle;
        \draw[PP] (O) ++(0,\cubey,-\cubez) -- ++(0,-\cubey,0);
        \draw[PP, fill=PP!30] (O)                     -- ++(\cubex,0,0) -- ++(0,\cubey,0) -- ++(-\cubex,0,0) -- cycle;
        \draw[PP, fill=PP!10] (O) ++(\cubex,\cubey/6,-\cubez/4) -- ++(0,0,-\cubez/2) -- ++(0,\cubey/4,0) -- ++(0,0,\cubez/2) -- cycle;
        \node[PB] at ($(O)+(\cubex/2,\cubey/2,0)$) {Script};
        % Note the NEED of extra group around \uncover in node! --> https://tex.stackexchange.com/a/9130/128737
        \path[from] ($(O)+(\cubex/2,\cubey,-\cubez/2)$) edge[out=90, in=0]   node[pos=1, anchor=east] {\alert<2>{Command-line arguments}} ++(-1.5,0.5,0)
                                                        edge[out=90, in=270] node[pos=1, anchor=345]  {\alert<2>{Environment variables}} ++(-1.0,1.3,-0.2)
                                                        edge[out=90, in=270] node[pos=1, anchor= south, inner sep=1pt]  {File(s){\uncover<2>{$^{\star}$}}} ++(0.1,1.5,-1.5)
                                                        edge[out=90, in=180] node[pos=1, anchor=west] {File descriptors{\uncover<2>{$^{\star}$}}} ++(1.8,1.0,-0.3);
        \path[to] ($(O)+(\cubex,\cubey/24*7,-\cubez/2)$) edge[out=0, in=180] node[pos=1, anchor=west] {File(s){\uncover<2>{$^{\star}$}}} ++(3.0,-0.5,0)
                                                         edge[out=0, in=180] node[pos=1, anchor=west] {File descriptors{\uncover<2>{$^{\star}$}}} ++(2.2,-1.5,0)
                                                         edge[out=0, in=180] node[pos=1, anchor=west] {\alert<2>{Environment variables}} ++(1.4,-2.5,0);
    \end{tikzpicture}
    \vspace{35mm}
    \begin{columns}[c]
        \begin{column}{0.08\textwidth}
        \end{column}
        \begin{column}{0.4\textwidth}
            \begin{varblock}{example}[\textwidth]{Think before doing!}
                It is part of the design-phase to keep in mind all these possibilities and take advantage of them.
            \end{varblock}
        \end{column}
        \begin{column}{0.52\textwidth}
        \end{column}
    \end{columns}
    \FrameRemark[2]{$^{\star}$ We will discuss all these in detail tomorrow.}
\end{frame}
%~~~~~~~~~~~~~~~~~~~~~~~~~~~~~~~~~~~~~~~~~~~~%
\begin{frame}[fragile]{Command-line parameters}
    \vspace{-5mm}
    \begin{overlayarea}{\textwidth}{0.8\textheight}
        \begin{itemize}
            \item They are accessible via \PB{\texttt{\$1}}, \PB{\texttt{\$2}}, etc.
            \item After the 9th one, you need curly braces: \PB{\texttt{\$\{10\}}}, \PB{\texttt{\$\{11\}}}, etc.
            \item It is possible to refer to all of them via \PB{\texttt{\$@}} and \PB{\texttt{\$*}}
            \item When you refer to all of them, especially to pass them over, use \PS{\texttt{"\$@"}}\\
                  $\to\,$ the double quotes are \alert{crucial} to preserve the parameters without splitting them!
            \item The \bash|shift| bulit-in is remarkably handy when parsing command-line parameters\\
                  $\to\,$ it destroys \PB{\texttt{\$1}} and it maps \PB{\texttt{\$2}} into \PB{\texttt{\$1}}, \PB{\texttt{\$3}} into \PB{\texttt{\$2}} and so on
        \end{itemize}
        \begin{onlyenv}<1>
            \begin{lstlisting}[style=MyBash]
                $ set -- Hello my "nice world"
                $ printf '%s\n' "$@"
                |+Hello
                my
                nice world+|
                $ printf '%s\n' $@   # <-- Probably, AAAAARGH!
                |+Hello
                my
                nice
                world+|
            \end{lstlisting}
        \end{onlyenv}
        \begin{onlyenv}<2>
            \begin{lstlisting}[style=MyBash, firstnumber=11]
                $ echo "$1"; shift
                |+Hello+|
                $ echo "$1"; shift
                |+my+|
                $ echo "$1"; shift
                |+nice world+|
                $ echo _"$1"_
                |+__+|
                $ set -- Hello my "nice world"; shift 2@|\tikzmark{two}|@; echo "$1"
                |+nice world+|
            \end{lstlisting}
        \end{onlyenv}
    \end{overlayarea}
    \begin{tikzpicture}[remember picture, overlay]
        \draw[from, visible on=<2>, red] (two) ++(-0.09,0.2) -- ++(0,1) node[pos=1, anchor=south] {Cool!};
    \end{tikzpicture}
\end{frame}
%~~~~~~~~~~~~~~~~~~~~~~~~~~~~~~~~~~~~~~~~~~~~%
\begin{frame}[fragile]{Environment variables (I)}
    \vspace{-3mm}
    \begin{itemize}
        \item Every program inherits certain information from its parent process {\tiny\{~resources, privileges and restrictions~\}}
        \item One of those resources is a set of variables called \PP{Environment Variables}
        \item Traditionally, environment variables have names that are all capital letters, such as \bash|PATH|
        \item When you run a command in Bash, you have the option of specifying a temporary environment change which only takes effect for the duration of that command
        \item This is done by putting \PB{\texttt{VAR=value}} in front of the command
    \end{itemize}
    \begin{lstlisting}[style=MyBash]
        $ ls /tpm
        |+ls: cannot access '/tpm': No such file or directory+|
        $ LANGUAGE=de_DE.utf-8 ls /tpm  # @|\URL[red]{https://askubuntu.com/a/544728}{Read more about LANGUAGE}[background-color]|@
        |+ls: Zugriff auf '/tpm' nicht möglich: Datei oder Verzeichnis nicht gefunden+|
        $ VERBOSE=1 make
    \end{lstlisting}
    \begin{varblock}{example}[0.9\textwidth]{Good practice}
        Don't use all-capital variable names in your scripts, unless they are environment variables.
        Use lower-case or mixed-case variable names, to avoid accidents.
    \end{varblock}
\end{frame}
%~~~~~~~~~~~~~~~~~~~~~~~~~~~~~~~~~~~~~~~~~~~~%
\begin{frame}[fragile]{Environment variables (II)}
    \vspace{-3mm}
    \begin{itemize}
        \item In a script, you can use environment variables just like any other variable
              \begin{lstlisting}[style=MyBash, numbers=none, aboveskip=2mm, belowskip=-4mm, xrightmargin=30mm]
                  if [[ ${EDITOR} ]]; then
                      ${EDITOR}
                  else
                      emacs
                  fi
              \end{lstlisting}
        \item To change the environment \alert{for your child processes} to inherit, use the \bash|export| builtin
              \begin{lstlisting}[style=MyBash, numbers=none, aboveskip=2mm, xrightmargin=30mm]
                  export PATH=${HOME}/.local/bin:${PATH}
              \end{lstlisting}
    \end{itemize}
    \begin{varblock}{alerted}[0.9\textwidth]{Remember!}
        The tricky part here is that your environment changes are \alert{\textbf{only inherited by your descendants}}.
        You can't change the environment of a program that is already running, or of a program that you don't run. 
    \end{varblock}
    \PP{\centerline{\ldots{}mmmh, and if I needed it?}}
\end{frame}
%~~~~~~~~~~~~~~~~~~~~~~~~~~~~~~~~~~~~~~~~~~~~%
\begin{frame}[fragile]{Environment variables (III)}
    \vspace{-3mm}
    \begin{itemize}
        \item Sourcing a script, will execute it in the current environment/shell
              \begin{lstlisting}[style=MyBash, numbers=none, aboveskip=2mm, belowskip=-6mm, xrightmargin=20mm]
                  $ cat script.bash
                  |+#!/bin/sh
                  cd /tmp+|
                  $ pwd; ./script.bash; pwd
                  |+/home/sciarra/Documents+|
                  |+/home/sciarra/Documents+|
                  $ pwd; source script.bash; pwd # 'source' as '.'
                  |+/home/sciarra/Documents+|
                  |+/tmp+|
              \end{lstlisting}
              Indeed, this is what you do e.g.\ when you add code in the \PB{\texttt{\$\{HOME\}/.bashrc}} file
    \end{itemize}
    \begin{varblock}{}[0.9\textwidth]{Splitting a large script in several files}
        Although a script should not be huge, it is important sometimes to split it into pieces for handier development.
        This can be done using the \bash|source| builtin, delegating to the main (executable) script to source all secondary files.
        We will come back to this point when we introduce functions.
    \end{varblock}
\end{frame}



