%-------------------------------%
%  Author: Alessandro Sciarra   %
%    Date: 28 Sep 2020          %
%-------------------------------%

%~~~~~~~~~~~~~~~~~~~~~~~~~~~~~~~~~~~~~~~~~~~~%
\begin{frame}{The GNU Readline Library}{\URL[PB]{https://tiswww.case.edu/php/chet/readline/rltop.html}{Official website}}
    \vspace{-3mm}
    \begin{varblock}{}[0.9\textwidth]{What is it for?}
        The GNU Readline library provides a set of functions for use by applications that \alert{allow users to edit command lines as they are typed in}.
        Both Emacs and vi editing modes are available.
        The Readline library includes additional functions to maintain a list of previously-entered command lines, to recall and perhaps reedit those lines.
    \end{varblock}
    \vspace{2mm}
    \begin{itemize}[<2>]
        \setlength{\itemsep}{2mm}
        \item Documentation can be also found in the \URL[PB]{https://www.gnu.org/software/bash/manual/}{Bash manual v5.0 sections 8.1 to 8.5}.
        \item It provide lots of functionality, some of which you might already be used to.
        \item \PS{If you use bash in your terminal, take advantage of it, please!}
        \item A great deal of run-time behaviour is changeable with \alert{45 variables}. \Remark{\PB{Section 8.3.1}}
        \item There are \alert{\textasciitilde170 Readline commands} that may be bound to key sequences. \Remark{\PB{Section 8.4}}
    \end{itemize}
\end{frame}
%~~~~~~~~~~~~~~~~~~~~~~~~~~~~~~~~~~~~~~~~~~~~%
\begin{frame}{The Readline init file}
    \begin{itemize}
        \setlength{\itemsep}{2mm}
        \item The Readline library comes with a set of Emacs-like keybindings installed by default.
        \item It is possible to use a different set of keybindings, though.
        \item Any user can customise programs that use Readline by putting commands in an inputrc file, conventionally in his home directory.
        \item The name of this file is taken from the value of the shell variable \bash|INPUTRC|.
              If that variable is unset, the default is \texttt{\textasciitilde/.inputrc}.
              If that file does not exist or cannot be read, the ultimate default is \texttt{/etc/inputrc}.
        \item When a program which uses the Readline library starts up, the init file is read, and the key bindings are set.
        \item The \texttt{CTRL-x CTRL-r} command re-reads this init file, thus incorporating any changes that you might have made to it.
    \end{itemize}
\end{frame}
%~~~~~~~~~~~~~~~~~~~~~~~~~~~~~~~~~~~~~~~~~~~~%
\begin{frame}[fragile]{The \bash|bind| builtin}{Interacting with the Readline library}
    \begin{onlyenv}<1>
        \begin{lstlisting}[style=myBash, style=smaller, numbers=none]
            $ help bind

            |+bind: bind [-lpsvPSVX] [-m keymap] [-f filename] [-q name] [-u name]
                       [-r keyseq] [-x keyseq:shell-command]
                       [keyseq:readline-function or readline-command]

                Set Readline key bindings and variables.

                Bind a key sequence to a Readline function or a macro, or set a
                Readline variable. The non-option argument syntax is equivalent to
                that found in ~/.inputrc, but must be passed as a single argument:+|
                e.g., bind '"\C-x\C-r": re-read-init-file'.
        \end{lstlisting}
    \end{onlyenv}
    \begin{onlyenv}<2>
        \begin{lstlisting}[style=myBash, style=smaller, numbers=none]
            $ help bind

            [...]

             |+Options:+|

               |+-l                 List names of functions.
               -P                 List function names and bindings.
               -p                 List functions and bindings in a form that can be
                                  reused as input.
               -S                 List key sequences that invoke macros and their values
               -s                 List key sequences that invoke macros and their values
                                  in a form that can be reused as input.
               -V                 List variable names and values
               -v                 List variable names and values in a form that can
                                  be reused as input.
               -q  function-name  Query about which keys invoke the named function.
               -u  function-name  Unbind all keys which are bound to the named function.
               -r  keyseq         Remove the binding for KEYSEQ.
               -f  filename       Read key bindings from FILENAME.+|
               # Plus few more
        \end{lstlisting}
    \end{onlyenv}
\end{frame}
%~~~~~~~~~~~~~~~~~~~~~~~~~~~~~~~~~~~~~~~~~~~~%
\begin{frame}[fragile]{An extremely useful example}
    \begin{overlayarea}{\textwidth}{0.16\textheight}
        \vspace{-6mm}
        \begin{varblock}{example}[0.6\textwidth]{If you want to affect bash only}<only@3>
            You can put the following block in your \texttt{\textasciitilde/.bashrc} file and use then the \bash|bind| builtin.
        \end{varblock}
        \begin{varblock}{quote}[0.7\textwidth]{You can act globally instead}<only@4>
            \textnormal{To affect all programs that use the GNU Readline library, you can put the following block in your \texttt{\textasciitilde/.inputrc} file and load it.}
        \end{varblock}
        \begin{varblock}{alert}[0.55\textwidth]{\textbf{No more need of} \texttt{CTRL-r}}<only@5>
            You can try to type something and press the arrow up and see how it is.
        \end{varblock}
    \end{overlayarea}
    \begin{tikzpicture}[remember picture, overlay, scope on=<2>]
        \node[label={[text=PT, font=\LARGE, label distance=1mm]90:\textbf{Warning:} Once you try it, you'll never go back!}] at ($(current page.center)-(0,8mm)$) {\includegraphics[width=0.5\textwidth]{MatrixPills}};
    \end{tikzpicture}
    \begin{onlyenv}<3>
        \begin{lstlisting}[style=myBash, numbers=none]
            #If session is interactive do some useful binding
            if [[ -t 1 ]]; then
                #To bind page-up and page-down to search in history
                # (command starting by what typed)
                bind '"\e[5~": history-search-backward'
                bind '"\e[6~": history-search-forward'
                #To bind up and down arrow to search in history
                # (anything containing what typed)
                bind '"\e[A": history-substring-search-backward'
                bind '"\e[B": history-substring-search-forward'
                #Move right and left one word with CTRL+@|\,\tc{red}{\ding{222}}|@ and CTRL+@|\,\tc{red}{\reflectbox{\ding{222}}}|@
                bind '"\e[1;5D": backward-word'
                bind '"\e[1;5C": forward-word'
            fi
        \end{lstlisting}
    \end{onlyenv}
    \begin{onlyenv}<4-5>
        \begin{lstlisting}[style=myBash, numbers=none, commentstyle=\color{Gray}]
            #To use global settings as well
            $include /etc/inputrc

            #To bind page-up and page-down to search in history
            # (command starting by what typed)
            |+"\e[5~": history-search-backward+|
            |+"\e[6~": history-search-forward+|
            #To bind up and down arrow to search in history
            # (anything containing what typed)
            |+"\e[A": history-substring-search-backward+|
            |+"\e[B": history-substring-search-forward+|
            #Move right and left one word with CTRL+@|\,\tc{Gray}{\ding{222}}|@ and CTRL+@|\,\tc{Gray}{\reflectbox{\ding{222}}}|@
            |+"\e[1;5D": backward-word+|
            |+"\e[1;5C": forward-word+|
        \end{lstlisting}
    \end{onlyenv}
    \FrameRemark[4-5]{Remember, you can reload the Readline init file via \texttt{CTRL-x CTRL-r} or via \,\bash|bind -f| \texttt{\textasciitilde/.inputrc} (removed commands will not be cleaned).}
\end{frame}
%~~~~~~~~~~~~~~~~~~~~~~~~~~~~~~~~~~~~~~~~~~~~%
